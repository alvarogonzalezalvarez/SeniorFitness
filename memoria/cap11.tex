% -*-cap2.tex-*-
% Este fichero es parte de la plantilla LaTeX para
% la realización de Proyectos Final de Carrera, protejido
% bajo los términos de la licencia GFDL.
% Para más información, la licencia completa viene incluida en el
% fichero fdl-1.3.tex

% Copyright (C) 2009 Pablo Recio Quijano 

En este capítulo se detallan los objetivos alcanzados, las enseñanzas obtenidas por parte del desarrollador del proyecto y las posibles mejoras futuras que puedan implementarse en el sistema.

\section{Objetivos alcanzados}

A pesar de que a lo largo del desarrollo del proyecto han ido surgiendo nuevas ideas y cambios con respecto a los planteamientos iniciales, podemos concluir que se han conseguido alcanzar los objetivos principales y las ideas base que se definieron durante la definición de la aplicación, dado que ésta nos permite de forma sencilla llevar un control y seguimiento sobre la condición física funcional de personas mayores, siempre basándonos en las pruebas y criterios definidos en la batería de pruebas Senior Fitness Test.\\

\section{Lecciones aprendidas}

Durante el desarrollo del proyecto he podido adquirir nuevos conocimientos y refrescar otros tantos que estudié durante la realización de la carrera años atrás y que, debido a que las metodologías de trabajo actualmente están en constante evolución en el mundo laboral y que éstas se van adaptando a las necesidades de la empresa, no había tenido ocasión de poner en práctica. Algunos ejemplos podrían ser la realización de los diagramas de secuencia y otros conceptos relacionados con UML.\\

Es importante destacar también que es la primera vez que me enfrento al desarrollo de una aplicación móvil Android completa. Anteriormente había realizado de forma autodidacta pequeños tutoriales y ejercicios, pero no ha sido hasta la realización de este proyecto cuando he podido obtener y consolidar los conocimientos necesarios para, desde 0, crear una aplicación haciendo uso de Android.\\

Por otro lado también me gustaría dejar constancia de todo lo que me llevo aprendido en relación al tema que abarca el proyecto, lo cual me ha llevado a tener que indagar sobre el mundo e-health para contextualizar el uso y la finalidad principal de la aplicación.\\

En definitiva, puedo concluir que el afrontar este proyecto ha supuesto un reto tanto personal como intelectual.

\section{Trabajo futuro}

A lo largo del desarrollo del proyecto se han barajado ciertas opciones o alternativas que, al no haber sido implementadas en la versión actual de la aplicación, podrían ser aplicadas en futuras versiones o mejoras que se añadan al proyecto. Además, es indispensable contar con el hecho de que aparecerán posibles ampliaciones futuras en cada una de las facetas en las que se divide el sistema. A continuación listamos algunas:

\begin{itemize}
\item Estudiar la posibilidad de que, para los tests de flexibilidad de brazos y flexibilidad de piernas, a través de la cámara o de algún sensor de proximidad del dispositivo, se pueda obtener la distancia de separación entre los dedos de la mano con respecto a la punta del pie para el caso del test de flexibilidad de piernas, o la distancia de separación entre los dedos de ambas manos para el caso del test de flexibilidad de brazos.
\item Para el test de Agilidad, estudiar la posibilidad de añadir un contador de pasos que, a través del dispositivo, calcule aproximadamente la distancia recorrida y la persona sepa cuando volver hacia su sitio sin necesidad de colocar un cono a una distancia determinada.
\item Añadir un selector de sesiones completadas que permita seleccionar las que se deseen para comparar entre las mismas los resultados obtenidos en cada una de ellas.
\item En la pantalla de información de los tests que hacen uso del dispositivo para detectar el movimiento, añadir un botón que permita hacer una prueba del movimiento requerido para comprobar que la persona se familiariza con el mismo antes de realizar el test.
\item Actualmente el APK de la aplicación se encuentra disponible para su descarga en el repositorio GitHub ya mencionado en el presente documento. Sería interesante publicar la aplicación en Google Play Store para facilitar su instalación y distribución.
\item La interfaz gráfica de la aplicación se encuentra en español. Para futuras líneas de actuación del proyecto sería interesante que el sistema estuviera disponible en otros idiomas, como por ejemplo en inglés.
\end{itemize}


