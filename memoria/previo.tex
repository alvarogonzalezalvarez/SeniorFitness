% -*-previo.tex-*-
% Este fichero es parte de la plantilla LaTeX para
% la realización de Proyectos Final de Carrera, protejido
% bajo los términos de la licencia GFDL.
% Para más información, la licencia completa viene incluida en el
% fichero fdl-1.3.tex

% Copyright (C) 2009 Pablo Recio Quijano 

\section*{Agradecimientos}

Me gustaria agradecer y/o dedicar este texto a ...

\cleardoublepage

\section*{Licencia} % Por ejemplo GFDL, aunque puede ser cualquiera

Este documento ha sido liberado bajo Licencia GFDL 1.3 (GNU Free
Documentation License). Se incluyen los términos de la licencia en
inglés al final del mismo.\\

Copyright (c) 2017 Álvaro González Álvarez.\\

Permission is granted to copy, distribute and/or modify this document under the
terms of the GNU Free Documentation License, Version 1.3 or any later version
published by the Free Software Foundation; with no Invariant Sections, no
Front-Cover Texts, and no Back-Cover Texts. A copy of the license is included in
the section entitled "GNU Free Documentation License".\\

\cleardoublepage

\section*{Notación y formato}

Aquí incluiremos los aspectos relevantes a la notación y el formato a
lo largo del documento. Para simplificar podemos generar comandos
nuevos que nos ayuden a ello, ver \texttt{comandos.sty} para más
información. 

Cuando nos refiramos a un programa en concreto, utilizaremos la
notación: \\ \programa{emacs}.\\

Cuando nos refiramos a un comando, o función de un lenguaje, usaremos
la notación: \\ \comando{quicksort}.\\

\cleardoublepage

\section*{Resumen}

Es un hecho que el ejercicio físico puede paliar las limitaciones que va imponiendo el proceso de envejecimiento en las personas, pero este debe ser individualizado a las características de la persona mayor, y es por ello que es necesaria la valoración de la condición física de ésta. La Senior Fitness Test (SFT) es una batería de pruebas para tal valoración, y es una de las pocas que está adaptada a los mayores.\\

Esta batería evalúa la condición física funcional, entendiendo por este término: \textit{la capacidad física para desarrollar actividades normales de la vida diaria de forma segura, con independencia y sin una excesiva fatiga} (Rikli y Jones, 2001). Los parámetros de condición física que incluye dicha batería son: fuerza muscular (miembros superiores e inferiores), resistencia aeróbica, flexibilidad (miembros superiores e inferiores) y agilidad.\\

La aplicación Android que se presenta en este documento tiene como empresa darle al usuario la posibilidad de realizar los ejercicios que componen la SFT para cuantas personas mayores el usuario desee registrar en la aplicación, así como almacenar y llevar un control sobre los resultados obtenidos para cada una de ellas. Las numerosas características y sensores de los que disponen actualmente los dispositivos móviles facilitan en gran medida la monitorización de cada uno de los ejercicios y además reducen los materiales necesarios para obtener y registrar los resultados.\\

\textbf{Palabras claves}: Android, Aplicación, Senior Fitness Test, Persona Mayor, Ejercicio Físico