% -*-cap2.tex-*-
% Este fichero es parte de la plantilla LaTeX para
% la realización de Proyectos Final de Carrera, protejido
% bajo los términos de la licencia GFDL.
% Para más información, la licencia completa viene incluida en el
% fichero fdl-1.3.tex

% Copyright (C) 2009 Pablo Recio Quijano 

\section{Descripción}

Con la aplicación descrita en el presente documento, el usuario puede registrar en la misma cuantas personas mayores desee. Una vez exista una persona registrada en el sistema, podrá realizar la batería de pruebas de valoración de la condición física en la que consiste la Senior Fitness Test, sirviéndole de ayuda el propio dispositivo móvil donde se encuentre instalada la aplicación.\\

Para ello, dependiendo de la prueba que se haya seleccionado para realizar entre las disponibles, la aplicación hará uso por ejemplo del sensor de gravedad del dispositivo móvil para determinar si el ejercicio se ha realizado de forma satisfactoria (y por tanto contabilizar la repetición), o bien mostrar en pantalla un cronómetro que se deberá usar si la prueba así lo requiere, entre otros casos.\\

Una vez realizada cada prueba para la persona mayor en cuestión, el sistema da la posibilidad de almacenar el resultado obtenido por la aplicación, pudiéndose consultar posteriormente en el historial de sesiones realizadas por dicha persona.

\section{Características de la aplicación}

La aplicación ofrece una alternativa libre y gratuita para poder llevar un registro y control de la condición física funcional de personas mayores, pudiéndose consultar los resultados obtenidos y sus estadísticas asociadas en cualquier lugar y momento. Además, es fácilmente adaptable a nuevos tests o pruebas que se deseen ir incluyendo a la aplicación en un futuro.

\subsection{Componentes de la aplicación}

En esta sección se hará una pequeña descripción de los distintos componentes que conforman la aplicación.

\subsubsection{Usuario}

Usuario de la aplicación, que será quien realice el registro de las personas mayores en el sistema y tenga acceso a las estadísticas y resultados obtenidos para cada una de ellas.

\subsubsection{Personas mayores}

Son los sujetos que se registran en el sistema y quienes realizan las sesiones de ejercicios correspondientes a los tests que componen la batería Senior Fitness Test.

\subsubsection{Tests}

Los tests son las pruebas de valoración de la condición física en las que consiste la batería de pruebas denominada Senior Fitness Test. Los resultados obtenidos en cada uno de los tests se podrán almacenar para la persona mayor que haya realizado el ejercicio.

\subsubsection{Sesiones}

Una sesión es el conjunto de los diferentes tests o pruebas que componen la batería Senior Fitness Test, y se identifican por el día y la hora en el que se comienza con ella. Se considera que una persona ha completado una sesión cuando ya ha realizado cada uno de los tests de la batería de pruebas y por tanto ha obtenido un resultado para cada una de ellas. En caso de que quede algún test por realizar, la sesión se considera en progreso, y por tanto no se marcará como completada hasta obtener el resultado de todas y cada una de las pruebas restantes.

\subsubsection{Resultados}

Se entiende por resultado el valor obtenido por una persona mayor en un determinado test de una determinada sesión. Dependiendo del test, un resultado puede ser un número de repeticiones, un tiempo en segundos o bien una distancia medida en centímetros.

\subsubsection{Estadísticas}

Una vez se ha completado una sesión, desde el historial de sesiones realizadas por la persona mayor en cuestión, se pueden consultar las estadísticas calculadas en función de los resultados obtenidos en cada uno de los tests de dicha sesión.

\subsection{Pruebas de valoración de la condición física}

A continuación se listarán las diferentes pruebas o tests de valoración de la condición física que actualmente se han implementado en la aplicación, con una breve descripción para cada una de ellas.

\subsubsection{Fuerza de piernas (F\_Pna)}

Número de veces que es capaz de sentarse y levantarse de una silla durante un tiempo definido (30 segundos por defecto) con los brazos en cruz y colocados sobre el pecho.

\figura{fuerza_piernas.png}{scale=1}{Fuerza de piernas}{fuerza_piernas}{H}

\subsubsection{Fuerza de brazos (F\_Br)}

Número de flexiones de brazo completas, sentado en una silla, que realiza durante un tiempo definido (30 segundos por defecto) sujetando una pesa de 3 libras (2,27 kg) para mujeres y 5 libras (3,63 kg) para hombres.

\figura{fuerza_brazos.png}{scale=1}{Fuerza de brazos}{fuerza_brazos}{H}

\subsubsection{Resistencia aeróbica (Resist)}

Número de veces que levanta la rodilla hasta una altura equivalente al punto medio entre la rótula y la cresta ilíaca durante un tiempo definido (2 minutos por defecto). Se contabiliza una vez por cada ciclo (derecha-izquierda).

\figura{resistencia_aerobica.png}{scale=1}{Resistencia aeróbica}{resistencia_aerobica}{H}

\subsubsection{Flexibilidad de piernas (Flex\_Pna)}

Sentado en el borde de una silla, estirar la pierna y las manos intentan alcanzar los dedos del pie que está con una flexión de tobillo de 90 grados. Se mide la distancia entre la punta de los dedos de la mano y la punta del pie (positiva si los dedos de la mano sobrepasan los dedos del pie o negativa si los dedos de la manos no alcanzan a tocar los dedos del pie).

\figura{flexibilidad_piernas.png}{scale=1}{Flexibilidad de piernas}{flexibilidad_piernas}{H}

\subsubsection{Flexibilidad de brazos (Flex\_Br)}

Una mano se pasa por encima del mismo hombro y la otra pasa a tocar la parte media de la espalda intentando que ambas manos se toquen. Se mide la distancia entre la punta de los dedos de cada mano (positiva si los dedos de la mano se superponen o negativa si no llegan a tocarse los dedos de la mano).

\figura{flexibilidad_brazos.png}{scale=1}{Flexibilidad de brazos}{flexibilidad_brazos}{H}

\subsubsection{Agilidad (Agil)}

Partiendo de sentado, tiempo que tarda en levantarse caminar hasta un cono situado a 2,44 metros, girar y volver a sentarse.

\figura{agilidad.png}{scale=1}{Agilidad}{agilidad}{H}

\subsection{Evaluación de los resultados}

A continuación mostraremos los valores de referencias que se toman como intervalos normales en cada una de las pruebas según el sexo y la edad de la persona. La aplicación comparará los resultados obtenidos en cada uno de los tests con respecto a estos valores, para determinar si el resultado que ha obtenido la persona que realiza la sesión está por debajo, dentro o por encima de los intervalos normales.

\subsubsection{Intervalo normal en mujeres según la edad}

\figura{intervalos_mujeres.png}{scale=0.6}{Intervalo normal de valores según la edad para mujeres}{intervalos_mujeres}{H}

\subsubsection{Intervalo normal en hombres según la edad}

\figura{intervalos_hombres.png}{scale=0.6}{Intervalo normal de valores según la edad para hombres}{intervalos_hombres}{H}

