% -*-cap2.tex-*-
% Este fichero es parte de la plantilla LaTeX para
% la realización de Proyectos Final de Carrera, protejido
% bajo los términos de la licencia GFDL.
% Para más información, la licencia completa viene incluida en el
% fichero fdl-1.3.tex

% Copyright (C) 2009 Pablo Recio Quijano 

En este capítulo se describen todos los aspectos relacionados con el análisis de requisitos del sistema: catálogo de requisitos, modelo conceptual, así como la solución propuesta.

\section{Especificación de requisitos del sistema}

En esta sección se detallan todos los tipos de requisitos necesarios para satisfacer los objetivos y características descritas en las secciones anteriores del presente documento.\\

\subsection{Requisitos de interfaces externas}

En este apartado se describirán los requisitos de conexión del software y el hardware, así como la interfaz de usuario.\\

Sobre la conexión entre el software y el hardware se encarga el SDK de Android, por lo que al ser un sistema preestablecido, no será necesario realizar el diseño ni el análisis; tan solo haremos uso de él.\\

A continuación, pasamos a definir la interfaz de la aplicación. Todas las ventanas de la aplicación se adaptarán a la resolución nativa del dispositivo en que se ejecute.\\

El usuario utilizará la pantalla táctil para interactuar con la interfaz. Por defecto, bastará con una pulsación típica para activar el evento correspondiente. En caso contrario, se mencionará explícitamente.\\

Las diferentes pantallas que modelarán la aplicación serán:

\begin{itemize}
\item \textbf{Pantalla principal}: Es la pantalla inicial, y tiene como función ofrecer al usuario el listado de personas mayores que ha registrado en la aplicación, así como la posibilidad de acceder a las siguientes pantallas:
\begin{itemize}
\item Pantalla de registro de personas
\item Pantalla de selección de persona para comenzar o continuar con una sesión
\item Pantalla de detalle de persona
\end{itemize}
\item \textbf{Registro de personas}: Permite al usuario registrar a personas en el sistema. La información introducida será validada y almacenada en la base de datos. Además dará la posibilidad de tomar una foto con la cámara o bien seleccionar una imagen de la galería para asignarla al usuario. Una vez seleccionada la imagen, debe mostrar al usuario una pantalla de edición que permita recortar la imagen con la proporción de aspecto usada por la aplicación.
\item \textbf{Selección de persona para comenzar o continuar con una sesión}: Permite al usuario seleccionar una persona de entre todas las registradas en la aplicación. Una vez seleccionada, llevará a la pantalla de selección de test.
\item \textbf{Detalle de persona}: Muestra al usuario los datos de la persona que se está consultando, así como un listado de las sesiones realizadas por el mismo, encabezado por aquella sesión que esté en progreso en caso de existir. Además da la posibilidad al usuario de eliminar a la persona, así como de acceder a la pantalla de estadísticas de cualquier sesión ya realizada o a la pantalla de selección de test para aquella sesión que esté en progreso.
\item \textbf{Selección de test}: Muestra al usuario el listado de tests que la persona seleccionada debe realizar. Se indicarán aquellos tests que ya se hayan completado y el resultado obtenido, diferenciándolos de aquellos que estén pendientes de hacer. Además, al lado de cada test aparecerá un botón de información, que de ser pulsado mostrará una pantalla de información sobre el test a realizar.
\item \textbf{Información de test}: Muestra al usuario una breve explicación que indica en qué consiste el ejercicio que se debe realizar para completar el test, así como instrucciones para su correcta realización con el dispositivo móvil en caso de ser necesario. A esta pantalla se accederá desde la pantalla de selección de test.
\item \textbf{Estadísticas de sesión}: Muestra al usuario los resultados obtenidos en los diferentes tests de una sesión ya realizada por una persona, así como estadísticas sobre la misma. A esta pantalla se accederá desde la pantalla de detalle de persona.
\item \textbf{Resultados}: Esta pantalla es la que se muestra al usuario inmediatamente después de que la persona que esté realizando una sesión haya terminado el último test de la misma que tuviese pendiente. En la pantalla se indicarán los resultados obtenidos en la sesión para cada uno de los tests realizados.
\item \textbf{Fuerza de piernas}: En esta pantalla se lleva a cabo la realización del test Fuerza de piernas (F\_Pna). Permite al usuario especificar el tiempo (en segundos) que durará la prueba y unos botones que servirán para iniciar/parar/resetear la prueba. Durante la realización del ejercicio irá mostrando en pantalla el número de repeticiones realizadas, así como una cuenta atrás con el tiempo restante. Una vez haya finalizado el tiempo, mostrará un botón que permitirá almacenar el resultado obtenido para la persona que ha realizado el ejercicio. A esta pantalla se accederá desde la pantalla de selección de test.
\item \textbf{Fuerza de brazos}: En esta pantalla se lleva a cabo la realización del test Fuerza de brazos (F\_Br). Permite al usuario especificar el tiempo (en segundos) que durará la prueba y unos botones que servirán para iniciar/parar/resetear la prueba. Durante la realización del ejercicio irá mostrando en pantalla el número de repeticiones realizadas, así como una cuenta atrás con el tiempo restante. Una vez haya finalizado el tiempo, mostrará un botón que permitirá almacenar el resultado obtenido para la persona que ha realizado el ejercicio. A esta pantalla se accederá desde la pantalla de selección de test.
\item \textbf{Resistencia aeróbica}: En esta pantalla se lleva a cabo la realización del test Resistencia aeróbica (Resist). Permite al usuario especificar el tiempo (en segundos) que durará la prueba y unos botones que servirán para iniciar/parar/resetear la prueba. Durante la realización del ejercicio irá mostrando en pantalla el número de repeticiones realizadas, así como una cuenta atrás con el tiempo restante. Una vez haya finalizado el tiempo, mostrará un botón que permitirá almacenar el resultado obtenido para la persona que ha realizado el ejercicio. A esta pantalla se accederá desde la pantalla de selección de test.
\item \textbf{Flexibilidad de piernas}: En esta pantalla se debe introducir la distancia en centímetros obtenida tras la realización del text Flexibilidad de piernas (Flex\_Pna), y debe contener un botón que tras pulsarse almacenará en el sistema el valor para la persona que esté realizando la sesión. A esta pantalla se accederá desde la pantalla de selección de test.
\item \textbf{Flexibilidad de brazos}: En esta pantalla se debe introducir la distancia en centímetros obtenida tras la realización del text Flexibilidad de brazos (Flex\_Br), y debe contener un botón que tras pulsarse almacenará en el sistema el valor para la persona que esté realizando la sesión. A esta pantalla se accederá desde la pantalla de selección de test.
\item \textbf{Agilidad}: Esta pantalla mostrará al usuario un cronómetro con los controles correspondientes para iniciarlo, pausarlo y resetearlo. Además mostrará un botón que tras pulsarse almacenará en el sistema el tiempo obtenido (en segundos) para la persona que esté realizando el ejercicio. A esta pantalla se accederá desde la pantalla de selección de test.
\end{itemize}

\subsection{Requisitos funcionales}

Los requisitos funcionales se han agrupado en diferentes subsistemas, cada uno de éstos contienen los requisitos establecidos que permitirán cumplir los objetivos y características descritas en las secciones anteriores y que se detallarán más adelante.\\

Para describir los distintos comportamientos que tendrá el sistema, usaremos el lenguaje de modelado de sistemas UML, que representa los requisitos funcionales del sistema, centrado en qué hace y no cómo lo hace.

\subsubsection{4.1.2.1. Diagramas de casos de uso}

Para no sobrecargar el diagrama de casos de uso, se ha optado por dividirlos de la siguiente forma:
\begin{itemize}
\item Un diagrama de casos de uso que mostrará de forma global las funcionalidades de la aplicación.
\item Un diagrama de casos de uso para cada uno de los diferentes tests que componen la Senior Fitness Test.
\end{itemize}

A continuación se muestran los diagramas de casos de uso comentados:

\figura{casosdeusoUML.png}{scale=0.6}{Diagrama UML: Casos de uso de la aplicación}{casosdeuso_uml}{H}

\figura{casosdeusoUMLfuerza_resistencia.png}{scale=0.6}{Diagrama UML: Casos de uso test fuerza de piernas}{casosdeuso_fuerza_piernas}{H}

\figura{casosdeusoUMLfuerza_resistencia.png}{scale=0.6}{Diagrama UML: Casos de uso test fuerza de brazos}{casosdeuso_fuerza_brazos}{H}

\figura{casosdeusoUMLfuerza_resistencia.png}{scale=0.6}{Diagrama UML: Casos de uso test resistencia aeróbica}{casosdeuso_resistencia_aerobica}{H}

\figura{casosdeusoUMLagilidad.png}{scale=0.6}{Diagrama UML: Casos de uso test agilidad}{casosdeuso_agilidad}{H}

\figura{casosdeusoUMLflexibilidad.png}{scale=0.6}{Diagrama UML: Casos de uso test flexibilidad de piernas}{casosdeuso_agilidad}{H}

\figura{casosdeusoUMLflexibilidad.png}{scale=0.6}{Diagrama UML: Casos de uso test flexibilidad de brazos}{casosdeuso_agilidad}{H}

\subsubsection{4.1.2.2. Descripción de casos de uso}

A continuación se detallarán los casos de uso empleados en cada uno de los subsistemas.

\subsection{Requisitos no funcionales}

En este apartado se detallarán los requisitos no funcionales del sistema.

\begin{table}[H]
\label{RNF01}
\begin{center}
\begin{tabular}{| l | p{10cm} |}
\hline
RNF01 & Usabilidad.\\
\hline
Descripción & El sistema debe ser usable y disponer de una interfaz intuitiva, adaptable y fácil de manejar para un usuario de nivel básico.\\
\hline
Importancia & Vital.\\
\hline
\end{tabular}
\end{center}
\caption{RNF01}
\end{table} 

\begin{table}[H]
\label{RNF02}
\begin{center}
\begin{tabular}{| l | p{10cm} |}
\hline
RNF02 & Portabilidad.\\
\hline
Descripción & Debe ser posible usar la aplicación en diferentes dispositivos móviles Android, independientemente del tamaño de la pantalla de los mismos, gracias al diseño adaptable empleado en las interfaces de usuario.\\
\hline
Importancia & Vital.\\
\hline
\end{tabular}
\end{center}
\caption{RNF02}
\end{table}

\begin{table}[H]
\label{RNF03}
\begin{center}
\begin{tabular}{| l | p{10cm} |}
\hline
RNF03 & Escabilidad.\\
\hline
Descripción & El sistema debe responder de manera óptima y eficiente a futuras mejoras en el desarrollo de la aplicación, sin que éstas comprometan el estado de la misma. El código debe ser mantenible y fácilmente ampliable para futuras versiones.\\
\hline
Importancia & Vital.\\
\hline
\end{tabular}
\end{center}
\caption{RNF03}
\end{table}

\begin{table}[H]
\label{RNF04}
\begin{center}
\begin{tabular}{| l | p{10cm} |}
\hline
RNF04 & Seguridad\\
\hline
Descripción & Cuando la aplicación necesite hacer uso de algunas características del dispositivo móvil que puedan comprometer la privacidad/seguridad de los datos del usuario, como puede ser la cámara o el acceso de lectura/escritura a los archivos de la memoria del dispositivo, siempre solicitará permiso al usuario antes de hacer uso de las mismas. En caso de que no se conceda el permiso, la aplicación seguirá siendo funcional aunque no se pueda hacer uso de las características para las que no se ha concedido el permiso.\\
\hline
Importancia & Vital.\\
\hline
\end{tabular}
\end{center}
\caption{RNF04}
\end{table} 

\begin{table}[H]
\label{RNF05}
\begin{center}
\begin{tabular}{| l | p{10cm} |}
\hline
RNF05 & Rendimiento\\
\hline
Descripción & El rendimiento de la aplicación de ser tal que permita un desempeño agradable y suave durante el uso de la misma. Los tiempos de respuesta al insertar/recuperar información en la base de datos serán cortos y deberá minimizarse la utilización de recursos cuando sea posible para ahorrar batería.\\
\hline
Importancia & Vital.\\
\hline
\end{tabular}
\end{center}
\caption{RNF05}
\end{table}

\begin{table}[H]
\label{RNF06}
\begin{center}
\begin{tabular}{| l | p{10cm} |}
\hline
RNF06 & Diseño\\
\hline
Descripción & Como se ha indicado en el requisito anterior, minimizar la utilización de recursos y el tiempo de respuesta debe primar sobre cualquier factor. Sin embargo, se debe tener en cuenta que los dispositivos móviles Android (Java) pueden perder el contexto de la aplicación que ejecutan, por lo que el contenido debe ser recuperable. Será de valor añadido que el paquete .APK final tenga el menor tamaño posible para abarcar aquellos teléfonos con menos memoria secundaria.\\
\hline
Importancia & Vital.\\
\hline
\end{tabular}
\end{center}
\caption{RNF06}
\end{table}

\subsection{Reglas de negocio}

El sistema en su totalidad está desarrollado para ser software libre bajo licencia GNU GPL. Ello permite que el código sea accesible a cualquier desarrollador que quiera incrementar su funcionalidad y contribuir a la libre creación de contenidos. Así mismo, las herramientas y tecnologías empleadas también son de libre uso.

\subsection{Requisitos de información}

A continuación se detallarán los requisitos de información, que describen cómo gestiona el sistema la información que se va a almacenar, y son los siguientes:

\section{Modelo conceptual de datos}

En esta sección se muestra el diagrama conceptual de datos UML de la aplicación, con el que se visualizarán a modo de primer vistazo las clases que posteriormente se diseñarán e implementarán para cubrir las funcionalidades impuestas en el proyecto.\\

No será hasta el capítulo siguiente cuando se desarrolle todo lo que abarca al diseño del sistema, incluido el diseño de clases UML que se genera a partir del conceptual que se muestra a continuación:

\section{Modelo de comportamiento del sistema}

Este modelo de comportamiento especifica como debe actuar el sistema. El sistema es el que engloba todos los objetos, y el modelo consta de dos partes:

\begin{itemize}
\item Diagramas de secuencias del sistema: Muestran la secuencia de eventos entre el usuario y el sistema.
\item Contrato de las operaciones del sistema: Describen el efecto que producen las operaciones en el sistema.
\end{itemize}

\subsection{Diagramas de secuencia y contrato de las operaciones del sistema}

A continuación se mostrarán los diagramas de secuencia. Para evitar contenidos duplicados, se omitirán las operaciones que ya hayan sido explicadas con anterioridad.














