% -*-cap2.tex-*-
% Este fichero es parte de la plantilla LaTeX para
% la realización de Proyectos Final de Carrera, protejido
% bajo los términos de la licencia GFDL.
% Para más información, la licencia completa viene incluida en el
% fichero fdl-1.3.tex

% Copyright (C) 2009 Pablo Recio Quijano 

En este capítulo se describen todos los aspectos relacionados con el análisis de requisitos del sistema: catálogo de requisitos, modelo conceptual, así como la solución propuesta.

\section{Especificación de requisitos del sistema}

En esta sección se detallan todos los tipos de requisitos necesarios para satisfacer los objetivos y características descritas en las secciones anteriores del presente documento.\\

\subsection{Requisitos de interfaces externas}

En este apartado se describirán los requisitos de conexión del software y el hardware, así como la interfaz de usuario.\\

Sobre la conexión entre el software y el hardware se encarga el SDK de Android, por lo que al ser un sistema preestablecido, no será necesario realizar el diseño ni el análisis; tan solo haremos uso de él.\\

A continuación, pasamos a definir la interfaz de la aplicación. Todas las ventanas de la aplicación se adaptarán a la resolución nativa del dispositivo en que se ejecute.\\

El usuario utilizará la pantalla táctil para interactuar con la interfaz. Por defecto, bastará con una pulsación típica para activar el evento correspondiente. En caso contrario, se mencionará explícitamente.\\

Las diferentes pantallas que modelarán la aplicación serán:

\begin{itemize}
\item \textbf{Pantalla principal}: Es la pantalla inicial, y tiene como función ofrecer al usuario el listado de personas mayores que ha registrado en la aplicación, así como la posibilidad de acceder a las siguientes pantallas:
\begin{itemize}
\item Pantalla de registro de personas
\item Pantalla de selección de persona para comenzar o continuar con una sesión
\item Pantalla de detalle de persona
\end{itemize}
\item \textbf{Registro de personas}: Permite al usuario registrar a personas en el sistema. La información introducida será validada y almacenada en la base de datos. Además dará la posibilidad de tomar una foto con la cámara o bien seleccionar una imagen de la galería para asignarla al usuario. Una vez seleccionada la imagen, debe mostrar al usuario una pantalla de edición que permita recortar la imagen con la proporción de aspecto usada por la aplicación.
\item \textbf{Selección de persona para comenzar o continuar con una sesión}: Permite al usuario seleccionar una persona de entre todas las registradas en la aplicación. Una vez seleccionada, llevará a la pantalla de selección de test.
\item \textbf{Detalle de persona}: Muestra al usuario los datos de la persona que se está consultando, así como un listado de las sesiones realizadas por el mismo, encabezado por aquella sesión que esté en progreso en caso de existir. Además da la posibilidad al usuario de eliminar a la persona, así como de acceder a la pantalla de estadísticas de cualquier sesión ya realizada o a la pantalla de selección de test para aquella sesión que esté en progreso.
\item \textbf{Selección de test}: Muestra al usuario el listado de tests que la persona seleccionada debe realizar. Se indicarán aquellos tests que ya se hayan completado y el resultado obtenido, diferenciándolos de aquellos que estén pendientes de hacer. Además, al lado de cada test aparecerá un botón de información, que de ser pulsado mostrará una pantalla de información sobre el test a realizar.
\item \textbf{Información de test}: Muestra al usuario una breve explicación que indica en qué consiste el ejercicio que se debe realizar para completar el test, así como instrucciones para su correcta realización con el dispositivo móvil en caso de ser necesario. A esta pantalla se accederá desde la pantalla de selección de test.
\item \textbf{Estadísticas de sesión}: Muestra al usuario los resultados obtenidos en los diferentes tests de una sesión ya realizada por una persona, así como estadísticas sobre la misma. A esta pantalla se accederá desde la pantalla de detalle de persona.
\item \textbf{Resultados}: Esta pantalla es la que se muestra al usuario inmediatamente después de que la persona que esté realizando una sesión haya terminado el último test de la misma que tuviese pendiente. En la pantalla se indicarán los resultados obtenidos en la sesión para cada uno de los tests realizados.
\item \textbf{Fuerza de piernas}: En esta pantalla se lleva a cabo la realización del test Fuerza de piernas (F\_Pna). Permite al usuario especificar el tiempo (en segundos) que durará la prueba y unos botones que servirán para iniciar/parar/resetear la prueba. Durante la realización del ejercicio irá mostrando en pantalla el número de repeticiones realizadas, así como una cuenta atrás con el tiempo restante. Una vez haya finalizado el tiempo, mostrará un botón que permitirá almacenar el resultado obtenido para la persona que ha realizado el ejercicio. A esta pantalla se accederá desde la pantalla de selección de test.
\item \textbf{Fuerza de brazos}: En esta pantalla se lleva a cabo la realización del test Fuerza de brazos (F\_Br). Permite al usuario especificar el tiempo (en segundos) que durará la prueba y unos botones que servirán para iniciar/parar/resetear la prueba. Durante la realización del ejercicio irá mostrando en pantalla el número de repeticiones realizadas, así como una cuenta atrás con el tiempo restante. Una vez haya finalizado el tiempo, mostrará un botón que permitirá almacenar el resultado obtenido para la persona que ha realizado el ejercicio. A esta pantalla se accederá desde la pantalla de selección de test.
\item \textbf{Resistencia aeróbica}: En esta pantalla se lleva a cabo la realización del test Resistencia aeróbica (Resist). Permite al usuario especificar el tiempo (en segundos) que durará la prueba y unos botones que servirán para iniciar/parar/resetear la prueba. Durante la realización del ejercicio irá mostrando en pantalla el número de repeticiones realizadas, así como una cuenta atrás con el tiempo restante. Una vez haya finalizado el tiempo, mostrará un botón que permitirá almacenar el resultado obtenido para la persona que ha realizado el ejercicio. A esta pantalla se accederá desde la pantalla de selección de test.
\item \textbf{Flexibilidad de piernas}: En esta pantalla se debe introducir la distancia en centímetros obtenida tras la realización del text Flexibilidad de piernas (Flex\_Pna), y debe contener un botón que tras pulsarse almacenará en el sistema el valor para la persona que esté realizando la sesión. A esta pantalla se accederá desde la pantalla de selección de test.
\item \textbf{Flexibilidad de brazos}: En esta pantalla se debe introducir la distancia en centímetros obtenida tras la realización del text Flexibilidad de brazos (Flex\_Br), y debe contener un botón que tras pulsarse almacenará en el sistema el valor para la persona que esté realizando la sesión. A esta pantalla se accederá desde la pantalla de selección de test.
\item \textbf{Agilidad}: Esta pantalla mostrará al usuario un cronómetro con los controles correspondientes para iniciarlo, pausarlo y resetearlo. Además mostrará un botón que tras pulsarse almacenará en el sistema el tiempo obtenido (en segundos) para la persona que esté realizando el ejercicio. A esta pantalla se accederá desde la pantalla de selección de test.
\end{itemize}

\subsection{Requisitos funcionales}

Los requisitos funcionales se han agrupado en diferentes subsistemas, cada uno de éstos contienen los requisitos establecidos que permitirán cumplir los objetivos y características descritas en las secciones anteriores y que se detallarán más adelante.\\

Para describir los distintos comportamientos que tendrá el sistema, usaremos el lenguaje de modelado de sistemas UML, que representa los requisitos funcionales del sistema, centrado en qué hace y no cómo lo hace.

\subsubsection{4.1.2.1. Diagramas de casos de uso}

Para no sobrecargar el diagrama de casos de uso, se ha optado por dividirlos de la siguiente forma:
\begin{itemize}
\item Un diagrama de casos de uso que mostrará de forma global las funcionalidades de la aplicación.
\item Un diagrama de casos de uso para cada uno de los diferentes tests que componen la Senior Fitness Test.
\end{itemize}

A continuación se muestran los diagramas de casos de uso comentados:

\figura{casosdeusoUML.png}{scale=0.6}{Diagrama UML: Casos de uso de la aplicación}{casosdeuso_uml}{H}

\figura{casosdeusoUMLfuerza_resistencia.png}{scale=0.6}{Diagrama UML: Casos de uso test fuerza de piernas}{casosdeuso_fuerza_piernas}{H}

\figura{casosdeusoUMLfuerza_resistencia.png}{scale=0.6}{Diagrama UML: Casos de uso test fuerza de brazos}{casosdeuso_fuerza_brazos}{H}

\figura{casosdeusoUMLfuerza_resistencia.png}{scale=0.6}{Diagrama UML: Casos de uso test resistencia aeróbica}{casosdeuso_resistencia_aerobica}{H}

\figura{casosdeusoUMLagilidad.png}{scale=0.6}{Diagrama UML: Casos de uso test agilidad}{casosdeuso_agilidad}{H}

\figura{casosdeusoUMLflexibilidad.png}{scale=0.6}{Diagrama UML: Casos de uso test flexibilidad de piernas}{casosdeuso_agilidad}{H}

\figura{casosdeusoUMLflexibilidad.png}{scale=0.6}{Diagrama UML: Casos de uso test flexibilidad de brazos}{casosdeuso_agilidad}{H}

\subsubsection{4.1.2.2. Descripción de casos de uso}

A continuación se detallarán los casos de uso empleados en cada uno de los subsistemas.

\subsubsection{4.1.2.2.1. Subsistema de registro de personas mayores}

\begin{table}[H]
\label{CU01}
\begin{center}
\begin{tabular}{| l | p{10cm} |}
\hline
CU01 & Registrar persona.\\
\hline
Descripción & La persona se registra en el sistema indicando sus datos.\\
\hline
Precondición & El usuario debe encontrarse en la pantalla de registro de personas.\\
\hline
Secuencia normal & 1. El usuario pulsa sobre el botón de registro de persona desde la pantalla principal.
\newline 2. Se muestra al usuario la pantalla de registro de persona.
\newline 3. El usuario hace o selecciona una foto de la persona e introduce el DNI, su nombre, apellidos, fecha de nacimiento y género y pulsa sobre Guardar.
\newline 4. El sistema registra la persona y redirige a la pantalla principal de la aplicación.\\
\hline
Secuencia alternativa & 4a. Se ha dejado uno o varios campos vacíos. Vuelve al paso 3.
\newline 4b. Se ha introducido un DNI de persona existente. Vuelve al paso 3.\\
\hline
Postcondición & Se registra a la persona en el sistema exitosamente y redirige a la pantalla principal de la aplicación.\\
\hline
Importancia & Vital.\\
\hline
\end{tabular}
\end{center}
\caption{CU01}
\end{table} 

\subsubsection{4.1.2.2.2. Subsistema de gestión de personas mayores}

\begin{table}[H]
\label{CU02}
\begin{center}
\begin{tabular}{| l | p{10cm} |}
\hline
CU02 & Ver detalle de persona.\\
\hline
Descripción & El usuario visualiza los datos y sesiones realizadas por la persona seleccionada.\\
\hline
Precondición & Debe existir al menos una persona registrada en la aplicación y el usuario debe haber pulsado sobre una persona del listado de personas registradas de la pantalla principal.\\
\hline
Secuencia normal & 1. El usuario pulsa sobre una de las personas listadas en la pantalla principal.
\newline 2. Se muestra al usuario la pantalla de detalle de persona, donde se encuentran los datos de la misma y un listado de las sesiones realizadas.\\
\hline
Secuencia alternativa & No hay.\\
\hline
Postcondición & Se visualizan los datos de la persona seleccionada y se muestra el listado de sesiones realizadas y la sesión que actualmente tenga en progreso la persona, en caso de que tuviese alguna.\\
\hline
Importancia & Vital.\\
\hline
\end{tabular}
\end{center}
\caption{CU02}
\end{table} 

\begin{table}[H]
\label{CU03}
\begin{center}
\begin{tabular}{| l | p{10cm} |}
\hline
CU03 & Eliminar persona.\\
\hline
Descripción & El usuario elimina la persona seleccionada del sistema, así como todos los datos y resultados asociados a la misma.\\
\hline
Precondición & El usuario debe estar en la pantalla de detalle de usuario, por lo que debe existir al menos una persona registrada en la aplicación y el usuario debe haber pulsado sobre una persona del listado de personas registradas de la pantalla principal.\\
\hline
Secuencia normal & 1. El usuario pulsa sobre una de las personas listadas en la pantalla principal.
\newline 2. Se muestra al usuario la pantalla de detalle de persona, donde se encuentran los datos de la misma y un listado de las sesiones realizadas.
\newline 3. El usuario pulsa sobre el icono de la papelera, situado en la esquina superior izquierda.
\newline 4. Se muestra al usuario un popup de confirmación, donde el usuario puede confirmar o cancelar la acción.
\newline 5. El usuario confirma la acción y el sistema elimina la persona seleccionada y retorna al usuario a la pantalla principal de la aplicación.\\
\hline
Secuencia alternativa & 5a. El usuario cancela la acción. Vuelve al paso 2.\\
\hline
Postcondición & Se elimina la persona seleccionada del sistema exitosamente, así como los datos y resultados asociados a la misma.\\
\hline
Importancia & Vital.\\
\hline
\end{tabular}
\end{center}
\caption{CU03}
\end{table} 

\subsubsection{4.1.2.2.3. Subsistema de gestión de sesiones}

\begin{table}[H]
\label{CU04}
\begin{center}
\begin{tabular}{| l | p{10cm} |}
\hline
CU04 & Comenzar sesión.\\
\hline
Descripción & El usuario pulsa sobre el botón de la pesa situado en la esquina superior derecha de la pantalla principal y la aplicación muestra el listado de personas registradas en el sistema para que el usuario seleccione la persona que comenzará la sesión.\\
\hline
Precondición & Debe existir al menos una persona registrada en la aplicación y el usuario debe haber pulsado sobre el botón de la pesa situado en la esquina superior derecha de la pantalla principal.\\
\hline
Secuencia normal & 1. El usuario pulsa sobre el icono de la pesa situado en la esquina superior derecha de la pantalla principal.
\newline 2. Se muestra la pantalla el listado de personas registradas en la aplicación.\\
\hline
Secuencia alternativa & 2a. Si no hay ninguna persona registrada en la aplicación, se muestra un mensaje indicando que debe existir al menos una persona registrada en el sistema.\\
\hline
Postcondición & Se muestra el listado de personas registradas en la aplicación para que el usuario seleccione la persona que comenzará la sesión.\\
\hline
Importancia & Vital.\\
\hline
\end{tabular}
\end{center}
\caption{CU04}
\end{table}

\begin{table}[H]
\label{CU05}
\begin{center}
\begin{tabular}{| l | p{10cm} |}
\hline
CU05 & Seleccionar persona.\\
\hline
Descripción & El usuario pulsa sobre uno de los usuarios listados y, en caso de que el usuario no tenga ningúna sesión en progreso, se creará la sesión en el sistema para el usuario seleccionado. Por último se mostrará la pantalla de selección de test.\\
\hline
Precondición & El usuario se debe encontrar en la pantalla de selección de usuario y debe existir al menos una persona registrada en la aplicación.\\
\hline
Secuencia normal & 1. El usuario selecciona una de las personas listadas.
\newline 2. Se crea la sesión en el sistema para la persona seleccionada en caso de que el usuario no tuviese ninguna sesión en progreso.
\newline 3. Se muestra la pantalla de selección de test.\\
\hline
Secuencia alternativa & 2a. La persona seleccionada ya tiene una sesión creada y en progreso. Salta al punto 3, mostrándose los resultados obtenidos en aquellos tests que ya hayan sido completados para la sesión que tiene en progreso.\\
\hline
Postcondición & Se crea la sesión en el sistema satisfactoriamente para la persona seleccionada en caso de que no tuviese ninguna sesión en progreso y se muestra al usuario la pantalla de selección de test.\\
\hline
Importancia & Vital.\\
\hline
\end{tabular}
\end{center}
\caption{CU05}
\end{table}

\begin{table}[H]
\label{CU06}
\begin{center}
\begin{tabular}{| l | p{10cm} |}
\hline
CU06 & Seleccionar test.\\
\hline
Descripción & El usuario pulsa sobre uno de los test listados que no estén ya completados y se mostrará la pantalla de realización del test correspondiente.\\
\hline
Precondición & El usuario debe encontrarse en la pantalla de selección de test y la persona seleccionada debe tener una sesión creada y no completada.\\
\hline
Secuencia normal & 1. El usuario selecciona uno de los tests que no estén ya completados.
\newline 2. Se muestra la pantalla de realización del test correspondiente.\\
\hline
Secuencia alternativa & 1a. El usuario selecciona un test que ya estaba completado. Se muestra un mensaje en pantalla indicando que el test ya ha sido realizado y el usuario permanece en la pantalla de selección de test.\\
\hline
Postcondición & Se accede satisfactoriamente a la pantalla de realización del test.\\
\hline
Importancia & Vital.\\
\hline
\end{tabular}
\end{center}
\caption{CU06}
\end{table}

\begin{table}[H]
\label{CU07}
\begin{center}
\begin{tabular}{| l | p{10cm} |}
\hline
CU07 & Ver información de test.\\
\hline
Descripción & El usuario pulsa sobre el icono de información situado a la derecha de cada test en la pantalla de selección de test. El sistema entonces muestra una pantalla con explicaciones sobre cómo realizar el test para el que se está consultando la información.\\
\hline
Precondición & El usuario debe encontrarse en la pantalla de selección de test y la persona seleccionada debe tener una sesión creada y no completada.\\
\hline
Secuencia normal & 1. El usuario pulsa sobre el botón de información que aparece a la derecha de los tests no completados en la pantalla de selección de test.
\newline 2. Se muestra la pantalla de información del test, con las explicaciones pertinentes sobre como realizar el ejercicio.\\
\hline
Secuencia alternativa & No hay.\\
\hline
Postcondición & Se accede satisfactoriamente a la pantalla de información del test.\\
\hline
Importancia & Vital.\\
\hline
\end{tabular}
\end{center}
\caption{CU07}
\end{table}

\begin{table}[H]
\label{CU08}
\begin{center}
\begin{tabular}{| l | p{10cm} |}
\hline
CU08 & Reanudar sesión\\
\hline
Descripción & El usuario pulsa sobre la sesión en progreso que aparece listada en la pantalla de detalle de persona, mostrándose la pantalla de selección de test.\\
\hline
Precondición & El usuario debe encontrarse en la pantalla de detalle de usuario y la persona seleccionada debe tener una sesión creada y no completada.\\
\hline
Secuencia normal & 1. El usuario pulsa sobre la sesión en progreso listada para el usuario del que estamos consultando el detalle.
\newline 2. Se muestra la pantalla de selección de test, diferenciando aquellos que ya estén completados de los pendientes y mostrando los resultados para los que ya están realizados por el usuario para esa sesión.\\
\hline
Secuencia alternativa & No hay.\\
\hline
Postcondición & Se accede satisfactoriamente a la pantalla de selección de test y se muestran los resultados de aquellos que ya estuviesen completados para la sesión, en caso de que los hubiese.\\
\hline
Importancia & Vital.\\
\hline
\end{tabular}
\end{center}
\caption{CU08}
\end{table}

\subsubsection{4.1.2.2.4. Subsistema de resultados y cálculos de estadísticas}

\begin{table}[H]
\label{CU09}
\begin{center}
\begin{tabular}{| l | p{10cm} |}
\hline
CU09 & Ver estadísticas de sesión\\
\hline
Descripción & El usuario pulsa sobre cualquier sesión ya completada de las que aparecen listadas en la pantalla de detalle de persona, mostrándose la pantalla de estadísticas de la sesión.\\
\hline
Precondición & El usuario debe encontrarse en la pantalla de detalle de usuario y la persona seleccionada debe tener al menos una sesión ya completada.\\
\hline
Secuencia normal & 1. El usuario pulsa sobre cualquiera de las sesiones listadas como completadas para el usuario del que estamos consultando el detalle.
\newline 2. Se muestra la pantalla de estadísticas de la sesión completada seleccionada en el paso anterior.\\
\hline
Secuencia alternativa & No hay.\\
\hline
Postcondición & Se accede satisfactoriamente a la pantalla de estadísticas de la sesión completada seleccionada en la pantalla de detalle de usuario.\\
\hline
Importancia & Vital.\\
\hline
\end{tabular}
\end{center}
\caption{CU09}
\end{table}

\begin{table}[H]
\label{CU10}
\begin{center}
\begin{tabular}{| l | p{10cm} |}
\hline
CU10 & Mostrar resultados\\
\hline
Descripción & Cuando el usuario realiza el último test que quede por completar de la sesión, se mostrará al usuario la pantalla de resultados, que mostrará los resultados obtenidos en cada uno de los tests durante la sesión que se acaba de completar.\\
\hline
Precondición & El usuario debe haber completado todos los tests de la sesión que en ese momento tuviese en progreso.\\
\hline
Secuencia normal & 1. El usuario completa el último test que tuviese pendiente de la sesión que tuviese en progreso en el momento de la realización del test indicado.
\newline 2. Una vez obtenido el resultado del último test, se muestra la pantalla de resultados de todos los tests completados para la sesión que recién se ha acabado de completar.\\
\hline
Secuencia alternativa & 1a. El test que se ha completado no era el último y había algún test más pendiente para la sesión actual. Se vuelve a la pantalla de selección de test.\\
\hline
Postcondición & Se muestra satisfactoriamente la pantalla de resultados de la sesión, mostrándose los valores obtenidos para cada uno de los tests que se han completado.\\
\hline
Importancia & Vital.\\
\hline
\end{tabular}
\end{center}
\caption{CU10}
\end{table}

\subsubsection{4.1.2.2.5. Tests de valoración de la condición física}

\begin{table}[H]
\label{CU11}
\begin{center}
\begin{tabular}{| l | p{10cm} |}
\hline
CU11 & Definir duración de test\\
\hline
Descripción & El usuario define el tiempo (en segundos) que durará el test que se va a realizar. Este caso de uso solo aplica para aquellos tests que usan una cuenta atrás.\\
\hline
Precondición & El usuario debe encontrarse en la pantalla de realización de alguno de los siguientes tests: fuerza de brazos, fuerza de piernas o resistencia aeróbica.\\
\hline
Secuencia normal & 1. El usuario pulsa sobre el tiempo predefinido por defecto (30 segundos).
\newline 2. Se abre el teclado numérico del dispositivo móvil.
\newline 3. El usuario borra el valor que ya esté especificado, introduce el número de segundos deseado y acepta.\\
\hline
Secuencia alternativa & No hay.\\
\hline
Postcondición & Se actualiza satisfactoriamente el tiempo de duración de la prueba, mostrándose en la pantalla el número de segundos introducido por el usuario.\\
\hline
Importancia & Vital.\\
\hline
\end{tabular}
\end{center}
\caption{CU11}
\end{table}

\begin{table}[H]
\label{CU12}
\begin{center}
\begin{tabular}{| l | p{10cm} |}
\hline
CU12 & Iniciar test\\
\hline
Descripción & El usuario pulsa sobre el botón START, activándose la detección de movimiento en el dispositivo móvil y comenzando la cuenta atrás. Además aparecen en pantalla los botones STOP y RESET.\\
\hline
Precondición & El usuario debe encontrarse en la pantalla de realización de alguno de los siguientes tests: fuerza de brazos, fuerza de piernas o resistencia aeróbica.\\
\hline
Secuencia normal & 1. El usuario coloca el dispositivo móvil en el brazo o en la pierna (según sea necesario) de la persona que va a realizar el test.
\newline 2. El usuario pulsa sobre el botón START.\\
\hline
Secuencia alternativa & No hay.\\
\hline
Postcondición & Se activa la detección de movimiento en el dispositivo móvil y comienza la cuenta atrás. El botón START desaparece y en su lugar aparecen los botones STOP y RESET.\\
\hline
Importancia & Vital.\\
\hline
\end{tabular}
\end{center}
\caption{CU12}
\end{table}

\begin{table}[H]
\label{CU13}
\begin{center}
\begin{tabular}{| l | p{10cm} |}
\hline
CU13 & Detectar movimiento\\
\hline
Descripción & El dispositivo móvil, colocado en el brazo o en la pierna con el/la que se realizará los ejercicios, detectará el movimiento realizado y contará las repeticiones que se hayan realizado de forma correcta, emitiendo un sonido por cada repetición contabilizada.\\
\hline
Precondición & El usuario debe encontrarse en la pantalla de realización de alguno de los siguientes tests: fuerza de brazos, fuerza de piernas o resistencia aeróbica. Debe tener colocado el móvil en el brazo o en la pierna, según el test a realizar y debe haber pulsado sobre el botón START, encontrándose el test en progreso y sin que haya finalizado la cuenta atrás.\\
\hline
Secuencia normal & 1. La persona que esté realizando el ejercicio y tenga el dispositivo móvil colocado en la extremidad con la que esté realizando la actividad, hace el movimiento requerido por el test mientras la cuenta atrás siga en progreso.\\
\hline
Secuencia alternativa & 1a. El movimiento realizado no es correcto. El sistema no contabiliza la repetición y no emite ningún sonido, manteniéndose inalterado el número de repeticiones que se muestra en pantalla.
\newline 1b. La persona realiza el movimiento cuando la cuenta atrás ha finalizado. El sistema no contabiliza la repetición y no emite ningún sonido, dado que al haber acabado la cuenta atrás el dispositivo móvil ya no detecta el movimiento realizado.\\
\hline
Postcondición & El sistema contabiliza la repetición, aumentando en 1 el número de repeticiones mostradas en pantalla y emitiendo un sonido.\\
\hline
Importancia & Vital.\\
\hline
\end{tabular}
\end{center}
\caption{CU13}
\end{table}

\begin{table}[H]
\label{CU14}
\begin{center}
\begin{tabular}{| l | p{10cm} |}
\hline
CU14 & Guardar resultado\\
\hline
Descripción & El usuario pulsa sobre el botón GUARDAR y el sistema almacena en la base de datos el resultado obtenido en el test para la sesión en progreso y para la persona que lo haya realizado.\\
\hline
Precondición & El usuario debe encontrarse en la pantalla de realización del test. Para que el botón GUARDAR aparezca, el test debe haberse completado. Según el tipo de test, se considerará completado cuando se cumplan una de las siguientes condiciones: que la cuenta atrás haya terminado, que el cronómetro se haya parado por el usuario o que se haya introducido la distancia medida para los tests de flexibilidad.\\
\hline
Secuencia normal & 1. Se finaliza el ejercicio o se completa la información requerida por el test.
\newline 2. El sistema muestra el botón GUARDAR.
\newline 3. El usuario pulsa sobre el botón GUARDAR.
\newline 4. El sistema almacena en la base de datos el resultado del test para la sesión en progreso y la persona que lo haya realizado y lleva a la pantalla de selección de test, donde se muestra el resultado obtenido junto al test recién completado.\\
\hline
Secuencia alternativa & 1a. El test aún no se ha completado. Volver al paso 1.
\newline 4a. El test recién completado era el último que quedaba pendiente de la sesión. En lugar de llevar a la pantalla de selección de test, lleva a la pantalla de resultados de la sesión.\\
\hline
Postcondición & El sistema almacena satisfactoriamente en la base de datos el resultado del test para la sesión en progreso y la persona que lo haya realizado y lleva a la pantalla de selección de test, donde se muestra el resultado recién almacenado.\\
\hline
Importancia & Vital.\\
\hline
\end{tabular}
\end{center}
\caption{CU14}
\end{table}

\begin{table}[H]
\label{CU15}
\begin{center}
\begin{tabular}{| l | p{10cm} |}
\hline
CU15 & Parar test\\
\hline
Descripción & La realización del test se para, de forma que el dispositivo móvil deja de detectar el movimiento y la cuenta atrás se detiene. Se muestra el botón RESET para volver los contadores a 0 si así se desea.\\
\hline
Precondición & El usuario debe encontrarse en la pantalla de realización de alguno de los siguientes tests: fuerza de brazos, fuerza de piernas o resistencia aeróbica y el test debe estar en progreso para que el botón STOP se muestre.\\
\hline
Secuencia normal & 1. El usuario pulsa sobre el botón STOP.\\
\hline
Secuencia alternativa & No hay.\\
\hline
Postcondición & Tanto la cuenta atrás como la detección de movimiento del dispositivo móvil se paran. Se permanece en la pantalla de realización de test, mostrando valores obtenidos hasta el momento y apareciendo el botón RESET.\\
\hline
Importancia & Vital.\\
\hline
\end{tabular}
\end{center}
\caption{CU15}
\end{table}

\begin{table}[H]
\label{CU16}
\begin{center}
\begin{tabular}{| l | p{10cm} |}
\hline
CU16 & Resetear test\\
\hline
Descripción & Restablece todos los contadores y temporizadores a sus valores iniciales, volviéndose a mostrar la pantalla como en su estado inicial.\\
\hline
Precondición & El usuario debe encontrarse en la pantalla de realización de alguno de los siguientes tests: fuerza de brazos, fuerza de piernas o resistencia aeróbica. El test debe estar en progreso o parado (tras haber estado en progreso) para que el botón RESET se muestre.\\
\hline
Secuencia normal & 1. El usuario pulsa sobre el botón RESET.\\
\hline
Secuencia alternativa & No hay.\\
\hline
Postcondición & La cuenta atrás se inicializa y las repeticiones contabilizadas hasta el momento se ponen a 0. La detección de movimiento del dispositivo móvil se deshabilita. Se permanece en la pantalla de realización de test, con todos los contadores mostrando los valores iniciales y apareciendo únicamente el botón START.\\
\hline
Importancia & Vital.\\
\hline
\end{tabular}
\end{center}
\caption{CU16}
\end{table}

\begin{table}[H]
\label{CU17}
\begin{center}
\begin{tabular}{| l | p{10cm} |}
\hline
CU17 & Iniciar cronómetro\\
\hline
Descripción & El usuario pulsa sobre el botón START, activándose el cronómetro. Además aparecen en pantalla los botones STOP y RESET.\\
\hline
Precondición & El usuario debe encontrarse en la pantalla de realización del test de Agilidad y el cronómetro no debe estar funcionando.\\
\hline
Secuencia normal & 1. El usuario pulsa sobre el botón START.\\
\hline
Secuencia alternativa & No hay.\\
\hline
Postcondición & El cronómetro comienza a funcionar y se va mostrando el tiempo transcurrido en pantalla. El botón START desaparece y en su lugar aparecen los botones STOP y RESET.\\
\hline
Importancia & Vital.\\
\hline
\end{tabular}
\end{center}
\caption{CU17}
\end{table}

\begin{table}[H]
\label{CU18}
\begin{center}
\begin{tabular}{| l | p{10cm} |}
\hline
CU18 & Parar cronómetro\\
\hline
Descripción & El cronómetro se para, mostrando en pantalla el tiempo transcurrido hasta el momento. Se muestra el botón RESET para volver los contadores a 0 si así se desea y el botón GUARDAR.\\
\hline
Precondición & El usuario debe encontrarse en la pantalla de realización del test de Agilidad y el cronómetro debe estar funcionando.\\
\hline
Secuencia normal & 1. El usuario pulsa sobre el botón STOP.\\
\hline
Secuencia alternativa & No hay.\\
\hline
Postcondición & Se para el cronómetro y se muestra en pantalla el tiempo registrado hasta el momento. Además aparecen en pantalla el botón RESET, por si se quiere resetear el cronómetro, y el botón GUARDAR, por si se quiere guardar el tiempo obtenido durante la realización del ejercicio.\\
\hline
Importancia & Vital.\\
\hline
\end{tabular}
\end{center}
\caption{CU18}
\end{table}

\begin{table}[H]
\label{CU19}
\begin{center}
\begin{tabular}{| l | p{10cm} |}
\hline
CU19 & Resetear cronómetro\\
\hline
Descripción & Restablece el cronómetro a 0 y se ocultan todos los botones, volviéndose a mostrar la pantalla como en su estado inicial.\\
\hline
Precondición & El usuario debe encontrarse en la pantalla de realización del test de Agilidad y el cronómetro debe estar en funcionamiento o parado (tras haber estado en funcionamiento) para que el botón RESET se muestre.\\
\hline
Secuencia normal & 1. El usuario pulsa sobre el botón RESET.\\
\hline
Secuencia alternativa & No hay.\\
\hline
Postcondición & El cronómetro se inicializa a 0. Se permanece en la pantalla de realización de test pero volviendo a su estado inicial y apareciendo únicamente el botón START.\\
\hline
Importancia & Vital.\\
\hline
\end{tabular}
\end{center}
\caption{CU19}
\end{table}

\begin{table}[H]
\label{CU20}
\begin{center}
\begin{tabular}{| l | p{10cm} |}
\hline
CU20 & Introducir distancia\\
\hline
Descripción & El usuario introduce la medida obtenida tras realizar el test de flexibilidad de brazos o de piernas a la persona que realiza la sesión.\\
\hline
Precondición & El usuario debe encontrarse en la pantalla de realización de alguno de los siguientes tests: flexibilidad de brazos o flexibilidad de piernas.\\
\hline
Secuencia normal & 1. El usuario pulsa sobre el input donde introducirá la distancia en centímetros.
\newline 2. Se abre el teclado numérico del dispositivo móvil.
\newline 3. El usuario introduce la distancia obtenida y acepta.\\
\hline
Secuencia alternativa & No hay.\\
\hline
Postcondición & Se muestra satisfactoriamente en la pantalla la distancia introducida y se muestra el botón GUARDAR para almacenar en el sistema el valor introducido para la sesión y la persona que ha realizado el test.\\
\hline
Importancia & Vital.\\
\hline
\end{tabular}
\end{center}
\caption{CU20}
\end{table}

\subsection{Requisitos no funcionales}

En este apartado se detallarán los requisitos no funcionales del sistema.

\begin{table}[H]
\label{RNF01}
\begin{center}
\begin{tabular}{| l | p{10cm} |}
\hline
RNF01 & Usabilidad.\\
\hline
Descripción & El sistema debe ser usable y disponer de una interfaz intuitiva, adaptable y fácil de manejar para un usuario de nivel básico.\\
\hline
Importancia & Vital.\\
\hline
\end{tabular}
\end{center}
\caption{RNF01}
\end{table} 

\begin{table}[H]
\label{RNF02}
\begin{center}
\begin{tabular}{| l | p{10cm} |}
\hline
RNF02 & Portabilidad.\\
\hline
Descripción & Debe ser posible usar la aplicación en diferentes dispositivos móviles Android, independientemente del tamaño de la pantalla de los mismos, gracias al diseño adaptable empleado en las interfaces de usuario.\\
\hline
Importancia & Vital.\\
\hline
\end{tabular}
\end{center}
\caption{RNF02}
\end{table}

\begin{table}[H]
\label{RNF03}
\begin{center}
\begin{tabular}{| l | p{10cm} |}
\hline
RNF03 & Escabilidad.\\
\hline
Descripción & El sistema debe responder de manera óptima y eficiente a futuras mejoras en el desarrollo de la aplicación, sin que éstas comprometan el estado de la misma. El código debe ser mantenible y fácilmente ampliable para futuras versiones.\\
\hline
Importancia & Vital.\\
\hline
\end{tabular}
\end{center}
\caption{RNF03}
\end{table}

\begin{table}[H]
\label{RNF04}
\begin{center}
\begin{tabular}{| l | p{10cm} |}
\hline
RNF04 & Seguridad\\
\hline
Descripción & Cuando la aplicación necesite hacer uso de algunas características del dispositivo móvil que puedan comprometer la privacidad/seguridad de los datos del usuario, como puede ser la cámara o el acceso de lectura/escritura a los archivos de la memoria del dispositivo, siempre solicitará permiso al usuario antes de hacer uso de las mismas. En caso de que no se conceda el permiso, la aplicación seguirá siendo funcional aunque no se pueda hacer uso de las características para las que no se ha concedido el permiso.\\
\hline
Importancia & Vital.\\
\hline
\end{tabular}
\end{center}
\caption{RNF04}
\end{table} 

\begin{table}[H]
\label{RNF05}
\begin{center}
\begin{tabular}{| l | p{10cm} |}
\hline
RNF05 & Rendimiento\\
\hline
Descripción & El rendimiento de la aplicación de ser tal que permita un desempeño agradable y suave durante el uso de la misma. Los tiempos de respuesta al insertar/recuperar información en la base de datos serán cortos y deberá minimizarse la utilización de recursos cuando sea posible para ahorrar batería.\\
\hline
Importancia & Vital.\\
\hline
\end{tabular}
\end{center}
\caption{RNF05}
\end{table}

\begin{table}[H]
\label{RNF06}
\begin{center}
\begin{tabular}{| l | p{10cm} |}
\hline
RNF06 & Diseño\\
\hline
Descripción & Como se ha indicado en el requisito anterior, minimizar la utilización de recursos y el tiempo de respuesta debe primar sobre cualquier factor. Sin embargo, se debe tener en cuenta que los dispositivos móviles Android (Java) pueden perder el contexto de la aplicación que ejecutan, por lo que el contenido debe ser recuperable. Será de valor añadido que el paquete .APK final tenga el menor tamaño posible para abarcar aquellos teléfonos con menos memoria secundaria.\\
\hline
Importancia & Vital.\\
\hline
\end{tabular}
\end{center}
\caption{RNF06}
\end{table}

\subsection{Reglas de negocio}

El sistema en su totalidad está desarrollado para ser software libre bajo licencia GNU GPL. Ello permite que el código sea accesible a cualquier desarrollador que quiera incrementar su funcionalidad y contribuir a la libre creación de contenidos. Así mismo, las herramientas y tecnologías empleadas también son de libre uso.

\subsection{Requisitos de información}

A continuación se detallarán los requisitos de información, que describen cómo gestiona el sistema la información que se va a almacenar, y son los siguientes:

\begin{table}[H]
\label{RIN01}
\begin{center}
\begin{tabular}{| l | p{10cm} |}
\hline
RIN01 & Información de la persona.\\
\hline
Descripción & El sistema debe almacenar la información sobre las personas mayores registradas en la aplicación.\\
\hline
Datos específicos & DNI de la persona.
\newline Nombre de la persona.
\newline Apellidos de la persona.
\newline Género de la persona.
\newline Fecha de nacimiento de la persona.
\newline Foto asociada a la persona.\\
\hline
Importancia & Vital.\\
\hline
\end{tabular}
\end{center}
\caption{RIN01}
\end{table} 

\begin{table}[H]
\label{RIN02}
\begin{center}
\begin{tabular}{| l | p{10cm} |}
\hline
RIN02 & Información de la sesión.\\
\hline
Descripción & El sistema debe almacenar la información correspondiente a las sesiones de ejercicio que realizan las personas registradas en la aplicación.\\
\hline
Datos específicos & Identificador de la persona que realiza la sesión.
\newline Estado de la sesión.
\newline Fecha de inicio de la sesión.\\
\hline
Importancia & Vital.\\
\hline
\end{tabular}
\end{center}
\caption{RIN02}
\end{table} 

\begin{table}[H]
\label{RIN03}
\begin{center}
\begin{tabular}{| l | p{10cm} |}
\hline
RIN03 & Información del test.\\
\hline
Descripción & El sistema debe almacenar la información correspondiente a los diferentes tests que realizarán las personas y que componen la batería de pruebas Senior Fitness Test (y por tanto cada una de las sesiones).\\
\hline
Datos específicos & Nombre del test.
\newline Descripción del test.\\
\hline
Importancia & Vital.\\
\hline
\end{tabular}
\end{center}
\caption{RIN03}
\end{table} 

\begin{table}[H]
\label{RIN04}
\begin{center}
\begin{tabular}{| l | p{10cm} |}
\hline
RIN04 & Información del resultado.\\
\hline
Descripción & El sistema debe almacenar los resultados que las personas obtienen en cada uno de los tests que componen la batería de pruebas Senior Fitness Test, y que estarán asociados a la sesión que la persona haya realizado en una fecha concreta.\\
\hline
Datos específicos & Identificador de la persona que ha realizado el test.
\newline Identificador de la sesión en la que la persona ha realizado el test.
\newline Identificador del test realizado.
\newline Resultado obtenido en el test.
\newline Fecha de obtención del resultado.\\
\hline
Importancia & Vital.\\
\hline
\end{tabular}
\end{center}
\caption{RIN04}
\end{table} 

\section{Modelo conceptual de datos}

En esta sección se muestra el diagrama conceptual de datos UML de la aplicación, con el que se visualizarán a modo de primer vistazo las clases que posteriormente se diseñarán e implementarán para cubrir las funcionalidades impuestas en el proyecto.\\

En la siguiente imagen podemos ver el diagrama Entidad-Relación que representa las entidades relevantes del sistema de información implementado, así como sus interrelaciones:

\figura{modeloconceptual_er.png}{scale=0.6}{Análisis: Modelo conceptual Entidad-Relación}{modeloconceptual_er}{H}

\section{Modelo de comportamiento del sistema}

Este modelo de comportamiento especifica como debe actuar el sistema. El sistema es el que engloba todos los objetos, y el modelo consta de dos partes:

\begin{itemize}
\item Diagramas de secuencias del sistema: Muestran la secuencia de eventos entre el usuario y el sistema.
\item Contrato de las operaciones del sistema: Describen el efecto que producen las operaciones en el sistema.
\end{itemize}

\subsection{Diagramas de secuencia y contrato de las operaciones del sistema}

A continuación se mostrarán los diagramas de secuencia. No todos los posibles diagramas de secuencia aparecerán, sino que nos centraremos en los más importantes. Además, para evitar contenidos duplicados, se omitirán las operaciones que ya hayan sido explicadas con anterioridad.














