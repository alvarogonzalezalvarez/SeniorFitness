% -*-memoria.tex-*-
% Este fichero es parte de la plantilla LaTeX para
% la realización de Proyectos Final de Carrera, protejido
% bajo los términos de la licencia GFDL.
% Para más información, la licencia completa viene incluida en el
% fichero fdl-1.3.tex

% Copyright (C) 2009 Pablo Recio Quijano 

%-------------------------------------------------------
% ---- Plantilla para libros / memorias PFC -----

% Realizada por Pablo Recio Quijano y Noelia Sales Montes 
% Formato de portada y primera página tomado del PFC de
% Francisco Javier Vázquez Púa, en su proyecto 'libgann'
% -------------------------------------------------------

\documentclass[a4paper,11pt]{book}

\usepackage{./estilos/estiloBase} % Basicamente son todas las
                                  % librerias usadas. En caso de que
                                  % falten librerias se van añadiendo
                                  % al fichero.
\usepackage{./estilos/colores}  % Algunos colores ya generados, para
                                % los algunos estilos más avanzados.
\usepackage{./estilos/comandos} % Algunos comandos personalizados

\graphicspath{{./imagenes/}} % Indicamos la ruta donde se encuentran
                             % las imagenes, para ahorrarnos la ruta
                             % completa, y solo modificar aquí si en
                             % un momento dado lo movemos

\begin{document}

% Renombramos las figuras y las tablas
\renewcommand{\figurename}{Figura}
\renewcommand{\listfigurename}{Indice de figuras}
\renewcommand{\tablename}{Tabla}
\renewcommand{\listtablename}{Indice de tablas}

\pagestyle{empty}
\input{portada.tex}
\cleardoublepage

\input{primerahoja.tex}
\cleardoublepage
\pagestyle{plain}

\frontmatter % Introducción, índices ...

% -*-previo.tex-*-
% Este fichero es parte de la plantilla LaTeX para
% la realización de Proyectos Final de Carrera, protejido
% bajo los términos de la licencia GFDL.
% Para más información, la licencia completa viene incluida en el
% fichero fdl-1.3.tex

% Copyright (C) 2009 Pablo Recio Quijano 

\section*{Agradecimientos}

Me gustaria agradecer y/o dedicar este texto a ...

\cleardoublepage

\section*{Licencia} % Por ejemplo GFDL, aunque puede ser cualquiera

Este documento ha sido liberado bajo Licencia GFDL 1.3 (GNU Free
Documentation License). Se incluyen los términos de la licencia en
inglés al final del mismo.\\

Copyright (c) 2017 Álvaro González Álvarez.\\

Permission is granted to copy, distribute and/or modify this document under the
terms of the GNU Free Documentation License, Version 1.3 or any later version
published by the Free Software Foundation; with no Invariant Sections, no
Front-Cover Texts, and no Back-Cover Texts. A copy of the license is included in
the section entitled "GNU Free Documentation License".\\

\cleardoublepage

\section*{Notación y formato}

Aquí incluiremos los aspectos relevantes a la notación y el formato a
lo largo del documento. Para simplificar podemos generar comandos
nuevos que nos ayuden a ello, ver \texttt{comandos.sty} para más
información. 

Cuando nos refiramos a un programa en concreto, utilizaremos la
notación: \\ \programa{emacs}.\\

Cuando nos refiramos a un comando, o función de un lenguaje, usaremos
la notación: \\ \comando{quicksort}.\\

\cleardoublepage

\section*{Resumen}

Es un hecho que el ejercicio físico puede paliar las limitaciones que va imponiendo el proceso de envejecimiento en las personas, pero este debe ser individualizado a las características de la persona mayor, y es por ello que es necesaria la valoración de la condición física de ésta. La Senior Fitness Test (SFT) es una batería de pruebas para tal valoración, y es una de las pocas que está adaptada a los mayores.\\

Esta batería evalúa la condición física funcional, entendiendo por este término: \textit{la capacidad física para desarrollar actividades normales de la vida diaria de forma segura, con independencia y sin una excesiva fatiga} (Rikli y Jones, 2001) \cite{mitt04}. Los parámetros de condición física que incluye dicha batería son: fuerza muscular (miembros superiores e inferiores), resistencia aeróbica, flexibilidad (miembros superiores e inferiores) y agilidad.\\

La aplicación Android que se presenta en este documento tiene como empresa darle al usuario la posibilidad de realizar los ejercicios que componen la SFT para cuantas personas mayores el usuario desee registrar en la aplicación, así como almacenar y llevar un control sobre los resultados obtenidos para cada una de ellas. Las numerosas características y sensores de los que disponen actualmente los dispositivos móviles facilitan en gran medida la monitorización de cada uno de los ejercicios y además reducen los materiales necesarios para obtener y registrar los resultados.\\

\textbf{Palabras claves}: Android, Aplicación, Senior Fitness Test, Persona Mayor, Ejercicio Físico

\tableofcontents
\listoffigures
\listoftables

\mainmatter % Contenido en si ...

\chapter{Introducción}
% -*-cap1.tex-*-
% Este fichero es parte de la plantilla LaTeX para
% la realización de Proyectos Final de Carrera, protejido
% bajo los términos de la licencia GFDL.
% Para más información, la licencia completa viene incluida en el
% fichero fdl-1.3.tex

% Copyright (C) 2009 Pablo Recio Quijano 

A continuación, se describe la motivación del presente proyecto y su alcance, y justificaremos la elección de un proyecto como el presente. También se incluye un glosario de términos y la organización del resto de la presente documentación.

\section{Motivación}

Actualmente, si tomamos como referencia la población española, las personas mayores de 65 años representan el 18,4\% del número total de habitantes (INE, 2016) \cite{website:ine}, y se espera que este porcentaje continue aumentando en las próximas décadas. Por tanto, es esperable encontrarnos cada vez con mayor número de personas mayores y a su vez con más limitaciones físicas, que vendrán dadas principalmente por el irreversible proceso de envejecimiento, y muchas de las cuales pueden paliarse con el ejercicio físico. Sin embargo, para conseguir los máximos beneficios del ejercicio para cada persona es de vital relevancia conocer la condición física del mayor para la correcta prescripción de ejercicio, y es en este punto donde entra en juego la Senior Fitness Test (SFT), una batería de pruebas específicamente diseñada para evaluar la condición física funcional de las personas mayores.\\

Partiendo de esta base y teniendo en cuenta que actualmente es cada vez más frecuente el uso de dispositivos móviles por una amplia gama de usuarios y que el rango de edad de los mismos crece sin detenerse, se determinó que sería interesante desarrollar una aplicación móvil que, con la ayuda de los numerosos sensores de los que disponen los dispositivos móviles de hoy en día, sirviese como herramienta para realizar las diferentes pruebas físicas que componen la SFT, y además diese la posibilidad de registrar y consultar los resultados obtenidos para cada sujeto en las diferentes sesiones realizadas.\\

De esta forma cualquier profesional dentro del sector de Ciencias de la Salud, de la Actividad Física o del Deporte que esté interesado en llevar un seguimiento de la condición física funcional de personas mayores, puede hacer uso de la aplicación para que ésta contabilice de forma autónoma (sin necesidad de tener que contar a ojo ni con la ayuda de un cronómetro) las repeticiones realizadas y los tiempos obtenidos por los sujetos para cada uno de los diferentes ejercicios de la batería de pruebas, además de quedar registrados en el perfil de cada persona para su estudio.\\

Debido a todo lo anteriormente mencionado y a que el aprendizaje de las tecnologías empleadas resulta muy útil de cara al mercado laboral por tratarse de tecnologías punteras en la actualidad, puede justificarse un proyecto como el presente como Proyecto de Fin de Carrera.

\section{Alcance}

El alcance del proyecto consiste en la realización de una aplicación Android que, mediante los sensores que contienen los dispositivos móviles actuales tales como el acelerómetro, el giroscopio o el sensor de gravedad, sea capaz de detectar la correcta realización de los ejercicios en los que consiste la SFT.\\

Para que los sensores del dispositivo móvil sean capaces de realizar las lecturas necesarias para el correcto cálculo del movimiento realizado durante la ejecución del ejercicio, es necesario que el usuario siga las instrucciones dadas por la aplicación para cada uno de los diferentes ejercicios físicos. Por ejemplo, puede ser necesario que el dispositivo móvil se coloque en la extremidad (brazo o pierna) con la que se realizará la actividad.\\

Durante la realización de las pruebas que hagan uso de los sensores comentados, el dispositivo móvil dará feedback sobre la correcta realización de los ejercicios, emitiendo un sonido con cada repetición realizada de forma satisfactoria y en el mismo instante en el que se contabiliza, sin que sea necesario observar la pantalla del dispositivo para conocer si se ha realizado adecuadamente el ejercicio.\\

Una vez finalizado cualquiera de los tests, la aplicación da la posibilidad de almacenar el resultado para la persona mayor que esté realizando la batería de pruebas, dándose por concluida la sesión cuando se haya completado el último test que quede pendiente de realizar de la SFT.\\

El usuario puede consultar en todo momento los resultados obtenidos por cada uno de los sujetos que han ido realizando las pruebas, tanto para la sesión que esté en progreso (con algún test pendiente de finalizar) como para aquellas sesiones que ya se completaron con anterioridad, para las cuales se mostrarán las correspondientes estadísticas.\\

La aplicación es de código libre bajo la licencia GNU GPL así como las técnicas y herramientas empleadas para su desarrollo. Su uso será en dispositivos móviles Android.

\section{Glosario de términos}

Esta sección contiene una lista ordenada alfabéticamente de los principales términos, acrónimos y abreviaturas específicos del dominio del problema:

\begin{itemize}
\item \textbf{Android}: Android es un sistema operativo basado en el núcleo Linux. Fue diseñado principalmente para dispositivos móviles con pantalla táctil, como teléfonos inteligentes, tablets y también para relojes inteligentes, televisores y automóviles.
\item \textbf{API}: La interfaz de programación de aplicaciones es un conjunto de subrutinas, funciones y procedimientos (o métodos, en la programación orientada a objetos) que ofrece cierta biblioteca para ser utilizado por otro software como una capa de abstracción.
\item \textbf{APK}: Un archivo con extensión .apk (Android Application Package, significado en español: Aplicación empaquetada de Android) es un paquete para el sistema operativo Android. Este formato es una variante del formato JAR de Java y se usa para distribuir e instalar componentes empaquetados para la plataforma Android.
\item \textbf{Aplicación}: Es un tipo de programa informático diseñado como herramienta para permitir a un usuario realizar uno o diversos tipos de trabajos.
\item \textbf{Condición física funcional}: Capacidad física para desarrollar actividades normales de la vida diaria de forma segura, con independencia y sin una excesiva fatiga (Rikli y Jones, 2001).
\item \textbf{e-health} / \textbf{eHealth} / \textbf{eSalud}: Es el término con el que se define al conjunto de Tecnologías de la Información y la Comunicación (TICs) que, a modo de herramientas, se emplean en el entorno sanitario en materia de prevención, diagnóstico, tratamiento, seguimiento, así como en la gestión de la salud, ahorrando costes al sistema sanitario y mejorando la eficacia de este.
\item \textbf{Framework}: Es un conjunto estandarizado de conceptos, prácticas y criterios para enfocar un tipo de problemática particular que sirve como referencia, para enfrentar y resolver nuevos problemas de índole similar. Un framework proporciona bibliotecas para acceder a bases de datos, estructuras para plantillas y gestión de sesiones, y con frecuencia facilitan la reutilización de código.
\item \textbf{Git}: Es un software de control de versiones diseñado por Linus Torvalds, pensando en la eficiencia y la confiabilidad del mantenimiento de versiones de aplicaciones cuando éstas tienen un gran número de archivos de código fuente.
\item \textbf{GPL}: La Licencia Pública General de GNU es la licencia de derecho de autor más ampliamente usada en el mundo del software y garantiza a los usuarios finales (personas, organizaciones, compañías) la libertad de usar, estudiar, compartir (copiar) y modificar el software.
\item \textbf{IDE}: Un entorno de desarrollo integrado es una aplicación informática que proporciona servicios integrales para facilitarle al desarrollador o programador el desarrollo de software.
\item \textbf{INE}: El Instituto Nacional de Estadística (INE) es un organismo autónomo de España encargado de la coordinación general de los servicios estadísticos de la Administración General del Estado y la vigilancia, control y supervisión de los procedimientos técnicos de los mismos. Entre los trabajos que realiza, destacan las estadísticas sobre la demografía, economía, y sociedad españolas.
\item \textbf{JAR}: Es un tipo de archivo que permite ejecutar aplicaciones escritas en el lenguaje Java.
\item \textbf{Java}: Java es un lenguaje de programación de propósito general, concurrente, orientado a objetos que fue diseñado específicamente para tener tan pocas dependencias de implementación como fuera posible.
\item \textbf{Prueba} /  \textbf{ Test de valoración de la condición física}: Son las pruebas que se realizan con la finalidad de medir y valorar las diferentes cualidades físicas básicas en sus diferentes facetas. La medición y valoración de estas cualidades nos informa del estado actual del sujeto. Este dato es fundamental para la programación del entrenamiento, ya que nos indica si hay que trabajar de forma específica alguna de ellas, en función de los objetivos a alcanzar, o por el contrario hay que trabajar de forma general.
\item \textbf{Repositorio}: Es un sitio centralizado donde se almacena y mantiene información digital, habitualmente bases de datos o archivos informáticos.
\item \textbf{Senior Fitness Test (SFT)}: Es una batería de pruebas de valoración de la condición física diseñada por Rikli y Jones. Surgió por la necesidad de crear una herramienta que nos permitiese valorar la condición física de los mayores con seguridad así como de forma práctica.
\item \textbf{SQL}: Es un lenguaje específico del dominio que da acceso a un sistema de gestión de bases de datos relacionales que permite especificar diversos tipos de operaciones en ellos.
\item \textbf{SQLite}: SQLite es un motor de bases de datos muy popular en la actualidad por ofrecer características tan interesantes como su pequeño tamaño, no necesitar servidor, precisar poca configuración, ser transaccional y ser de código libre. Android incorpora de serie todas las herramientas necesarias para la creación y gestión de bases de datos SQLite, y entre ellas una completa API para llevar a cabo de manera sencilla todas las tareas necesarias.
\item \textbf{UML}: El Lenguaje Unificado de Modelado es el lenguaje de modelado de sistemas de software más conocido y utilizado en la actualidad. Es un lenguaje gráfico para visualizar, especificar, construir y documentar un Sistema.
\item \textbf{XML}: El Lenguaje de Marcas Extensible es un meta-lenguaje que permite definir lenguajes de marcas desarrollado por el World Wide Web Consortium (W3C) utilizado para almacenar datos en forma legible.
\end{itemize}

\section{Organización del Documento}

La estructura del presente documento es la siguiente:

\begin{itemize}
\item \textbf{Introducción}: Apartado introductorio sobre la motivación para desarrollar este proyecto, su alcance, así como la estructuración de este documento.
\item \textbf{Descripción general}: Descripción más amplia sobre el proyecto, así como todas las características relevantes que tendrá.
\item \textbf{Planificación}: En este apartado se detalla la planificación realizada para llevar a cabo el proyecto, así como las distintas etapas en las que está compuesto el mismo.
\item \textbf{Análisis de requisitos}: En este apartado se detallan los objetivos del sistema, actores del sistema, requisitos funcionales, requisitos no funcionales, reglas de negocio, requisitos de información y se representará el modelo conceptual de datos del proyecto.
\item \textbf{Diseño del sistema}: El diseño del sistema viene dado por una arquitectura general del sistema, patrones de diseño, diseño físico y lógico de datos, el diseño de la interfaz de usuario y los diagramas de clases y de secuencia.
\item \textbf{Implementación del sistema}: En este apartado se detallan los aspectos más importantes sobre la implementación del sistema: entorno de construcción y el código fuente.
\item \textbf{Pruebas y validaciones}: Pruebas realizada a la aplicación, con el fin de comprobar su correcto funcionamiento y cumplimiento de las expectativas.
\item \textbf{Conclusiones}: Conclusiones obtenidas sobre el proyecto desarrollado, visión futura del mismo y experiencia adquirida.
\item \textbf{Herramientas utilizadas}: Herramientas software utilizadas a lo largo del desarrollo del proyecto.
\item \textbf{Manual de instalación}: Apartado donde se indica como realizar la instalación de la aplicación en el sistema.
\item \textbf{Manual de usuario}: Manual de usuario para el correcto uso de la aplicación.
\item \textbf{Bibliografía}: Referencias consultadas durante la realización del proyecto.
\item \textbf{Licencia GNU GFDL}: Texto completo sobre la licencia GNU GFDL en inglés.
\end{itemize}



\chapter{Descripción general del proyecto}
% -*-cap2.tex-*-
% Este fichero es parte de la plantilla LaTeX para
% la realización de Proyectos Final de Carrera, protejido
% bajo los términos de la licencia GFDL.
% Para más información, la licencia completa viene incluida en el
% fichero fdl-1.3.tex

% Copyright (C) 2009 Pablo Recio Quijano 

\section{Descripción}

Descripción de la aplicación.

\section{Características de la aplicación}

Aquí ponemos las características de la aplicación.

\subsection{Componentes de la aplicación}

Aquí ponemos los componentes de la aplicación.

\subsection{Pruebas de valoración de la condición física}

Aquí cada una de las pruebas físicas que es posible realizar.

\chapter{Planificación}
% -*-cap2.tex-*-
% Este fichero es parte de la plantilla LaTeX para
% la realización de Proyectos Final de Carrera, protejido
% bajo los términos de la licencia GFDL.
% Para más información, la licencia completa viene incluida en el
% fichero fdl-1.3.tex

% Copyright (C) 2009 Pablo Recio Quijano 

La planificación realizada para el desarrollo del proyecto está dividida en varias partes, tal y como pasamos a describir a continuación.

\section{Fase inicial}

La primera fase consistió en plantear la idea del proyecto con la ayuda de Raquel Ureña. Tras plantear varios enfoques, se decidió realizar este proyecto debido a las motivaciones escritas anteriormente.\\

También se pensó en qué lenguaje de programación se desarrollaría el proyecto, así como las principales bibliotecas que se usarían durante la realización del mismo, priorizando siempre opciones libres. Debido a que ya tenía suficientes conocimientos de Java, finalmente se optó por utilizar el SDK oficial de Android (Java) por su documentación, comunidad y robustez.

\section{Fase de análisis}

Esta etapa está dividida principalmente en las dos partes siguientes:

\begin{itemize}
\item \textbf{Especificación de los requisitos}: Estudio de los diferentes requisitos que deberá cumplir la aplicación.
\item \textbf{Recursos necesarios}: Recursos necesarios que deberemos usar para llevar a cabo el desarrollo del proyecto.
\end{itemize}

\section{Fase de aprendizaje}

Aunque ya conocía Java en profundidad debido a mi profesión actual, hasta ahora no me había embarcado en el desarrollo de una aplicación Android, por lo que tuve que dedicarle tiempo a aprender algunas reglas de diseño, determinadas sintaxis del lenguaje y API de librerías, así como familiarizarme con el entorno de desarrollo Android Studio, que aunque tiene algunas similitudes respecto a Eclipse (IDE que uso a diario en mi puesto de trabajo), también tiene muchas otras diferencias.\\

Esta fase se caracterizó por intentar entender código ya escrito así como leer tutoriales y documentación oficial. Se podría dividir en varias etapas principales:

\begin{itemize}
\item \textbf{Aprendizaje del SDK de Android:} Las aplicaciones Android siguen un diseño específico basado en Activities, así como un sistema de layout propio para la capa de presentación. Dado que no tenía experiencia previa, me basé en la documentación oficial, la cual me ayudó a que el código evolucionase favorablemente y fuese fácil de mantener. 
\item \textbf{Estudio de los sensores existentes en los dispositivos móviles actuales:} En los dispositivos móviles de hoy en día existen numerosos sensores que se pueden usar para la detección de movimientos. Dado que en ciertos tests la aplicación debe ser capaz de detectar la correcta realización de los ejercicios, fue necesario hacer un estudio sobre los diferentes sensores disponibles (como pueden ser el giroscopio, acelerómetro o el sensor de gravedad) para determinar cual de ellos usar.
\item \textbf{Aprendizaje del entorno de desarrollo:} Estudio del entorno de desarrollo y de la operativa a seguir para la programación de los diferentes módulos de la aplicación, así como para el testeo, detección y depuración de errores.
\item \textbf{Aprendizaje de otras bibliotecas:} Para algunas funcionalidades fue necesario el aprendizaje de algunas bibliotecas adicionales, como puede ser por ejemplo SQLite para la persistencia de datos en Android.
\item \textbf{Aprendizaje del arte:} No estaba muy familiarizado con el retoque de imágenes, pero a partir de tutoriales y pruebas ensayo-error aprendí el manejo básico de los programas que usé para la edición de los iconos e imágenes usadas en la aplicación.
\end{itemize}

\section{Fase de desarrollo}

Tras la consecución de las etapas anteriores se comenzó el desarrollo del proyecto. Esta etapa del desarrollo es la más extensa de todas y me fue posible llevarla a cabo gracias a los prototipos y pruebas de concepto que iba implementando durante el aprendizaje de los aspectos detallados en el punto anterior.

\subsection{Metodología de desarrollo}

Para el desarrollo del presente proyecto se ha optado por el empleo del modelo de desarrollo iterativo e incremental. Se consideró que esta estrategia de desarrollo sería la idónea dado que, gracias al empleo de las iteraciones o tareas, nos sería de gran utilidad a la hora de hacer frente a la inclusión de nuevos requisitos a medida que fuera avanzando el desarrollo del proyecto.\\

Por tanto, el empleo de esta metodología implicará que el proceso de desarrollo se divida en iteraciones que abarquen a un segmento del proyecto que ya es funcional. Al ser éste también un modelo incremental, las futuras iteraciones evolucionarán el desarrollo anterior, incluyendo mejoras y respondiendo a los requisitos del sistema que se van cumplimentando y añadiendo a posteriori. Es importante mencionar que cada iteración tratará a un requisito del sistema, y que se tendrán en cuenta primeramente los requisitos básicos que permitan tener una versión funcional del sistema en las primeras iteraciones.\\

El cliente de este modo va obteniendo en cada iteración una versión más avanzada del proyecto pero siempre funcional, por lo que no tendrá que esperar al final del desarrollo de la aplicación para poder hacer uso del sistema.\\

El desarrollo iterativo e incremental permite que, al dividir el desarrollo del proyecto en tareas más pequeñas, sea más flexible la respuesta ante cambios y nuevos requisitos que imponga el cliente. Del mismo modo, será más sencillo realizar las pruebas y consideraciones que se hagan tras cada iteración completada, pues si éstas se realizaran una vez se desarrolle todo el proyecto (como ocurre con otras metodologías como el desarrollo en cascada), la aparición de un error en el diseño implicaría un coste temporal importante comparado con el de detectar un error en una parte o módulo específico del proyecto.

\subsection{Planificación del desarrollo}

Una vez decidida la metodología de desarrollo a emplear en el proyecto, se procede a especificar cada una de las iteraciones que compondrán el desarrollo de éste.\\

Antes de desarrollar cada iteración, cabe mencionar que existe una etapa previa a la del cumplimiento de las iteraciones en la que se han de especificar los objetivos del sistema, así como las diferentes tecnologías que se usarán y el tiempo que se estima para cada una de las iteraciones asignadas.\\

Así pues, las iteraciones que se contemplarán en el desarrollo del proyecto son las siguientes:

\begin{itemize}
\item \textbf{Iteración 1: Implementación de la prueba Fuerza de brazos (F\_Br):} En esta iteración se implementa el test \textit{Fuerza de brazos} de la batería de pruebas Senior Fitness Test. Una vez finalizada esta iteración, la aplicación es capaz de contar, con ayuda del sensor de gravedad, las repeticiones que el usuario hace durante la realización del ejercicio y durante un tiempo determinado (en segundos) introducido por el usuario. Además, al ser el primer test que se implementa, el desarrollo se hace de forma que sea posible reutilizar mediante herencia los métodos y funciones encargados de detectar las repeticiones y dar feedback al usuario para el resto de tests que se quieran implementar.
\item \textbf{Iteración 2: Implementación de la prueba Fuerza de piernas (F\_Pna):} En esta iteración se implementa el test \textit{Fuerza de piernas} de la batería de pruebas Senior Fitness Test. La amplitud del desarrollo en esta iteración se reduce, dado que se heredan funcionalidades de la clase padre implementada en la iteración anterior. Una vez finalizada esta iteración, la aplicación ya es capaz de contar las repeticiones que el usuario hace durante la realización de dos tipos de ejercicios diferentes de la SFT.
\item \textbf{Iteración 3: Implementación de la prueba Resistencia aeróbica (Resist):} En esta iteración se implementa el test \textit{Resistencia aeróbica} de la batería de pruebas Senior Fitness Test. Al igual que en la iteración anterior, es posible heredar de la clase padre implementada en la primera iteración. Una vez finalizada esta iteración, la aplicación ya es capaz de contar las repeticiones que el usuario hace durante la realización de tres tipos de ejercicios diferentes de la SFT.
\item \textbf{Iteración 4: Implementación de la prueba Agilidad (Agil):} En esta iteración se implementa el test \textit{Agilidad} de la batería de pruebas Senior Fitness Test. En este caso, al ser un tipo de ejercicio diferente en el que es necesario implementar un cronómetro y por tanto no es necesario contar repeticiones, no es posible heredar directamente de la actividad padre, por lo que es necesario hacer modificaciones sobre la misma para que pueda ser reutilizable por ejercicios que requieran un cronómetro, además de hacer la implementación propia del test.
\item \textbf{Iteración 5: Implementación de la prueba Flexibilidad de piernas (Flex\_Pna):} En esta iteración se implementa el test \textit{Flexibilidad de piernas} de la batería de pruebas Senior Fitness Test. En este caso, al ser un tipo de ejercicio diferente en el que es necesario obtener una medida en centímetros y por tanto no es necesario contar repeticiones ni hacer uso de un cronómetro, no es posible heredar directamente de la actividad padre, por lo que es necesario hacer modificaciones sobre la misma para que pueda ser reutilizable por ejercicios de este tipo, además de hacer la implementación propia del test.
\item \textbf{Iteración 6: Implementación de la prueba Flexibilidad de brazos (Flex\_Br):} En esta iteración se implementa el test \textit{Flexibilidad de brazos} de la batería de pruebas Senior Fitness Test. La amplitud del desarrollo se reduce en este caso con respecto a la iteración anterior, dado que en la misma se había adaptado la clase padre heredable para este tipo de ejercicios en los que el resultado a obtener es una medida en centímetros.
\item \textbf{Iteración 7: Desarrollo del subsistema de almacenamiento de la información:} Llegados a este punto, una vez implementados todos las pruebas de la SFT, se considera necesario empezar con la implementación del sistema encargado de almacenar en una base de datos la información y los resultados obtenidos para cada uno de los tests que se realicen, así como el diseño y el despliegue de la propia estructura. Al finalizar esta iteración solo se almacenan los resultados para el usuario del dispositivo móvil, sin posibilidad de hacer los tests a cualquier persona mayor que se desee.
\item \textbf{Iteración 8: Desarrollo del subsistema de registro de usuarios/personas mayores:} En esta tarea se implementa el sistema de registro de personas mayores en la aplicación, añadiendo una pantalla de registro básica para la obtención, validación y almacenamiento de los datos de cada persona en el sistema. También se implementa la funcionalidad para acceder a la cámara del dispositivo por si se quiere tomar una foto de la persona durante el registro, o bien elegir una imagen de la galería.
\item \textbf{Iteración 9: Desarrollo del subsistema de gestión de personas mayores:} En esta tarea se implementa la gestión de personas mayores registradas en la app. En la pantalla principal de la aplicación se muestra un listado de las personas registradas en la aplicación (con la información básica para identificar cada una de ellas, así como una foto de la persona en caso de habérsela tomado con el dispositivo móvil durante el registro). Si se toca sobre una persona del listado, se accede a la pantalla de detalle de usuario donde se da la posibilidad de eliminar al usuario del sistema. También se mostrarán las sesiones realizadas por el usuario.
\item \textbf{Iteración 10: Desarrollo del subsistema de gestión de sesiones:} En esta tarea se implementa la gestión de sesiones, esto es, la posibilidad de conocer qué tests ha realizado cada usuario y cuales les queda pendiente por realizar para dar por completada una sesión, mostrando en todo momento los resultados obtenidos en los tests ya realizados.
\item \textbf{Iteración 11: Desarrollo del subsistema de resultados y cálculos de estadísticas:} En esta iteración se implementa la pantalla de resultados que se muestra una vez que el usuario ha completado una sesión (es decir, ha realizado todos los tests de la SFT). Además, también se hace la implementación del cálculo de estadísticas mostradas cuando se accede a las sesiones realizadas por el usuario desde la pantalla de detalle del mismo.
\item \textbf{Iteración 12: Desarrollo de posibles mejoras de la aplicación:} En este punto se procederá a solventar los errores y mejoras que pudieran demandar las últimas pruebas aplicadas sobre todo el sistema, así como la inclusión de nuevas funcionalidades o tests que se soliciten.
\item \textbf{Iteración 13: Documentación del proyecto.}
\end{itemize}

Para tener una referencia del tiempo estimado para el desarrollo del proyecto y el tiempo real empleado, se muestra a continuación una tabla que representa dichos datos.\\

\begin{table}[H]
\label{tiempo}
\begin{center}
\begin{tabular}{| l | l | l |}
\hline
Etapa & Tiempo estimado & Tiempo real\\
\hline \hline
Fase previa & 30 días & 28 días\\
\hline
Iteración 1 & 15 días & 20 días\\
\hline
Iteración 2 & 5 días & 7 días\\
\hline
Iteración 3 & 5 días & 4 días\\
\hline
Iteración 4 & 10 días & 6 días\\
\hline
Iteración 5 & 10 días & 3 días\\
\hline
Iteración 6 & 5 días & 3 días\\
\hline
Iteración 7 & 15 días & 18 días\\
\hline
Iteración 8 & 15 días & 10 días\\
\hline
Iteración 9 & 10 días & 12 días\\
\hline
Iteración 10 & 15 días & 19 días\\
\hline
Iteración 11 & 15 días & 10 días\\
\hline
Iteración 12 & 10 días & 8 días\\
\hline
Iteración 13 & 20 días & 22 días\\
\hline \hline
Total & 180 días & 170 días\\
\hline
\end{tabular}
\end{center}
\caption{Tiempo estimado y tiempo real empleado en el proyecto}
\end{table}

En el último punto de este capítulo se encuentra el diagrama de Gantt inicial diseñado para el proyecto, en el cual se representa de manera gráfica el tiempo estimado que se le dedicará a cada una de las fases del desarrollo del proyecto.

\section{Pruebas y correcciones}

Una de las etapas más importantes del desarrollo de cualquier proyecto. Esta etapa se realizó en paralelo a la de desarrollo ya que, conforme se implementaron nuevas funcionalidades, se iban probando exhaustivamente bajo el contexto de las distintas situaciones que pudieran darse, hasta obtener el comportamiento esperado.\\

Además, conforme se iban entregando versiones funcionales de la aplicación en cada iteración a Raquel Ureña, ella también ejercía de tester, dándome feedback muy importante para mejorar el proyecto.

\section{Redacción de la memoria}

La redacción de la memoria se ha realizado conforme se iba avanzando en el desarrollo del proyecto pero, tras la finalización de éste, se le ha dedicado más tiempo a su cumplimentación, corrigiendo puntos que han requerido modificaciones o que finalmente no se han adecuado al producto final.

\section{Diagrama de Gantt}

Para la documentación del proyecto y en particular para lo concerniente al tiempo en que se realizarán cada una de las tareas descritas en cada iteración, se hace uso del modelo de diagrama de Gantt. Este modelo representa el tiempo de dedicación prevista para diferentes tareas o actividades a lo largo de un tiempo total determinado.\\

A continuación se mostrará el diagrama de Gantt inicial con el que se estimó el tiempo que iban a ocupar cada una de las fases e iteraciones del presente proyecto, tal y como se indicó en la sección 3.4.2 del presente capítulo.

 \figura{seniorfitness_gantt.png}{scale=0.5}{Diagrama de Gantt inicial del proyecto}{seniorfitness_gantt}{H}










\chapter{Análisis de requisitos}
% -*-cap2.tex-*-
% Este fichero es parte de la plantilla LaTeX para
% la realización de Proyectos Final de Carrera, protejido
% bajo los términos de la licencia GFDL.
% Para más información, la licencia completa viene incluida en el
% fichero fdl-1.3.tex

% Copyright (C) 2009 Pablo Recio Quijano 

En este capítulo se describen todos los aspectos relacionados con el análisis de requisitos del sistema: catálogo de requisitos, modelo conceptual, así como la solución propuesta.

\section{Especificación de requisitos del sistema}

En esta sección se detallan todos los tipos de requisitos necesarios para satisfacer los objetivos y características descritas en las secciones anteriores del presente documento.\\

\subsection{Requisitos de interfaces externas}

En este apartado se describirán los requisitos de conexión del software y el hardware, así como la interfaz de usuario.\\

Sobre la conexión entre el software y el hardware se encarga el SDK de Android, por lo que al ser un sistema preestablecido, no será necesario realizar el diseño ni el análisis; tan solo haremos uso de él.\\

A continuación, pasamos a definir la interfaz de la aplicación. Todas las ventanas de la aplicación se adaptarán a la resolución nativa del dispositivo en que se ejecute.\\

El usuario utilizará la pantalla táctil para interactuar con la interfaz. Por defecto, bastará con una pulsación típica para activar el evento correspondiente. En caso contrario, se mencionará explícitamente.\\

Las diferentes pantallas que modelarán la aplicación serán:

\begin{itemize}
\item \textbf{Pantalla principal}: Es la pantalla inicial, y tiene como función ofrecer al usuario el listado de personas mayores que ha registrado en la aplicación, así como la posibilidad de acceder a las siguientes pantallas:
\begin{itemize}
\item Pantalla de registro de personas
\item Pantalla de selección de persona para comenzar o continuar con una sesión
\item Pantalla de detalle de persona
\end{itemize}
\item \textbf{Registro de personas}: Permite al usuario registrar a personas en el sistema. La información introducida será validada y almacenada en la base de datos. Además dará la posibilidad de tomar una foto con la cámara o bien seleccionar una imagen de la galería para asignarla al usuario. Una vez seleccionada la imagen, debe mostrar al usuario una pantalla de edición que permita recortar la imagen con la proporción de aspecto usada por la aplicación.
\item \textbf{Selección de persona para comenzar o continuar con una sesión}: Permite al usuario seleccionar una persona de entre todas las registradas en la aplicación. Una vez seleccionada, llevará a la pantalla de selección de test.
\item \textbf{Detalle de persona}: Muestra al usuario los datos de la persona que se está consultando, así como un listado de las sesiones realizadas por el mismo, encabezado por aquella sesión que esté en progreso en caso de existir. Además da la posibilidad al usuario de eliminar a la persona, así como de acceder a la pantalla de estadísticas de cualquier sesión ya realizada o a la pantalla de selección de test para aquella sesión que esté en progreso.
\item \textbf{Selección de test}: Muestra al usuario el listado de tests que la persona seleccionada debe realizar. Se indicarán aquellos tests que ya se hayan completado y el resultado obtenido, diferenciándolos de aquellos que estén pendientes de hacer. Además, al lado de cada test aparecerá un botón de información, que de ser pulsado mostrará una pantalla de información sobre el test a realizar.
\item \textbf{Información de test}: Muestra al usuario una breve explicación que indica en qué consiste el ejercicio que se debe realizar para completar el test, así como instrucciones para su correcta realización con el dispositivo móvil en caso de ser necesario. A esta pantalla se accederá desde la pantalla de selección de test.
\item \textbf{Estadísticas de sesión}: Muestra al usuario los resultados obtenidos en los diferentes tests de una sesión ya realizada por una persona, así como estadísticas sobre la misma. A esta pantalla se accederá desde la pantalla de detalle de persona.
\item \textbf{Resultados}: Esta pantalla es la que se muestra al usuario inmediatamente después de que la persona que esté realizando una sesión haya terminado el último test de la misma que tuviese pendiente. En la pantalla se indicarán los resultados obtenidos en la sesión para cada uno de los tests realizados.
\item \textbf{Fuerza de piernas}: En esta pantalla se lleva a cabo la realización del test Fuerza de piernas (F\_Pna). Permite al usuario especificar el tiempo (en segundos) que durará la prueba y unos botones que servirán para iniciar/parar/resetear la prueba. Durante la realización del ejercicio irá mostrando en pantalla el número de repeticiones realizadas, así como una cuenta atrás con el tiempo restante. Una vez haya finalizado el tiempo, mostrará un botón que permitirá almacenar el resultado obtenido para la persona que ha realizado el ejercicio. A esta pantalla se accederá desde la pantalla de selección de test.
\item \textbf{Fuerza de brazos}: En esta pantalla se lleva a cabo la realización del test Fuerza de brazos (F\_Br). Permite al usuario especificar el tiempo (en segundos) que durará la prueba y unos botones que servirán para iniciar/parar/resetear la prueba. Durante la realización del ejercicio irá mostrando en pantalla el número de repeticiones realizadas, así como una cuenta atrás con el tiempo restante. Una vez haya finalizado el tiempo, mostrará un botón que permitirá almacenar el resultado obtenido para la persona que ha realizado el ejercicio. A esta pantalla se accederá desde la pantalla de selección de test.
\item \textbf{Resistencia aeróbica}: En esta pantalla se lleva a cabo la realización del test Resistencia aeróbica (Resist). Permite al usuario especificar el tiempo (en segundos) que durará la prueba y unos botones que servirán para iniciar/parar/resetear la prueba. Durante la realización del ejercicio irá mostrando en pantalla el número de repeticiones realizadas, así como una cuenta atrás con el tiempo restante. Una vez haya finalizado el tiempo, mostrará un botón que permitirá almacenar el resultado obtenido para la persona que ha realizado el ejercicio. A esta pantalla se accederá desde la pantalla de selección de test.
\item \textbf{Flexibilidad de piernas}: En esta pantalla se debe introducir la distancia en centímetros obtenida tras la realización del text Flexibilidad de piernas (Flex\_Pna), y debe contener un botón que tras pulsarse almacenará en el sistema el valor para la persona que esté realizando la sesión. A esta pantalla se accederá desde la pantalla de selección de test.
\item \textbf{Flexibilidad de brazos}: En esta pantalla se debe introducir la distancia en centímetros obtenida tras la realización del text Flexibilidad de brazos (Flex\_Br), y debe contener un botón que tras pulsarse almacenará en el sistema el valor para la persona que esté realizando la sesión. A esta pantalla se accederá desde la pantalla de selección de test.
\item \textbf{Agilidad}: Esta pantalla mostrará al usuario un cronómetro con los controles correspondientes para iniciarlo, pausarlo y resetearlo. Además mostrará un botón que tras pulsarse almacenará en el sistema el tiempo obtenido (en segundos) para la persona que esté realizando el ejercicio. A esta pantalla se accederá desde la pantalla de selección de test.
\end{itemize}

\subsection{Requisitos funcionales}

Los requisitos funcionales se han agrupado en diferentes subsistemas, cada uno de éstos contienen los requisitos establecidos que permitirán cumplir los objetivos y características descritas en las secciones anteriores y que se detallarán más adelante.\\

Para describir los distintos comportamientos que tendrá el sistema, usaremos el lenguaje de modelado de sistemas UML, que representa los requisitos funcionales del sistema, centrado en qué hace y no cómo lo hace.

\subsubsection{4.1.2.1. Diagramas de casos de uso}

Para no sobrecargar el diagrama de casos de uso, se ha optado por dividirlos de la siguiente forma:
\begin{itemize}
\item Un diagrama de casos de uso que mostrará de forma global las funcionalidades de la aplicación.
\item Un diagrama de casos de uso para cada uno de los diferentes tests que componen la Senior Fitness Test.
\end{itemize}

A continuación se muestran los diagramas de casos de uso comentados:

\figura{casosdeusoUML.png}{scale=0.6}{Diagrama UML: Casos de uso de la aplicación}{casosdeuso_uml}{H}

\figura{casosdeusoUMLfuerza_resistencia.png}{scale=0.6}{Diagrama UML: Casos de uso test fuerza de piernas}{casosdeuso_fuerza_piernas}{H}

\figura{casosdeusoUMLfuerza_resistencia.png}{scale=0.6}{Diagrama UML: Casos de uso test fuerza de brazos}{casosdeuso_fuerza_brazos}{H}

\figura{casosdeusoUMLfuerza_resistencia.png}{scale=0.6}{Diagrama UML: Casos de uso test resistencia aeróbica}{casosdeuso_resistencia_aerobica}{H}

\figura{casosdeusoUMLagilidad.png}{scale=0.6}{Diagrama UML: Casos de uso test agilidad}{casosdeuso_agilidad}{H}

\figura{casosdeusoUMLflexibilidad.png}{scale=0.6}{Diagrama UML: Casos de uso test flexibilidad de piernas}{casosdeuso_agilidad}{H}

\figura{casosdeusoUMLflexibilidad.png}{scale=0.6}{Diagrama UML: Casos de uso test flexibilidad de brazos}{casosdeuso_agilidad}{H}

\subsubsection{4.1.2.2. Descripción de casos de uso}

A continuación se detallarán los casos de uso empleados en cada uno de los subsistemas.

\subsubsection{4.1.2.2.1. Subsistema de registro de personas mayores}

\begin{table}[H]
\label{CU01}
\begin{center}
\begin{tabular}{| l | p{10cm} |}
\hline
CU01 & Registrar persona.\\
\hline
Descripción & La persona se registra en el sistema indicando sus datos.\\
\hline
Precondición & El usuario debe encontrarse en la pantalla de registro de personas.\\
\hline
Secuencia normal & 1. El usuario pulsa sobre el botón de registro de persona desde la pantalla principal.
\newline 2. Se muestra al usuario la pantalla de registro de persona.
\newline 3. El usuario hace o selecciona una foto de la persona e introduce el DNI, su nombre, apellidos, fecha de nacimiento y género y pulsa sobre Guardar.
\newline 4. El sistema registra la persona y redirige a la pantalla principal de la aplicación.\\
\hline
Secuencia alternativa & 4a. Se ha dejado uno o varios campos vacíos. Vuelve al paso 3.
\newline 4b. Se ha introducido un DNI de persona existente. Vuelve al paso 3.\\
\hline
Postcondición & Se registra a la persona en el sistema exitosamente y redirige a la pantalla principal de la aplicación.\\
\hline
Importancia & Vital.\\
\hline
\end{tabular}
\end{center}
\caption{CU01}
\end{table} 

\subsubsection{4.1.2.2.2. Subsistema de gestión de personas mayores}

\begin{table}[H]
\label{CU02}
\begin{center}
\begin{tabular}{| l | p{10cm} |}
\hline
CU02 & Ver detalle de persona.\\
\hline
Descripción & El usuario visualiza los datos y sesiones realizadas por la persona seleccionada.\\
\hline
Precondición & Debe existir al menos una persona registrada en la aplicación y el usuario debe haber pulsado sobre una persona del listado de personas registradas de la pantalla principal.\\
\hline
Secuencia normal & 1. El usuario pulsa sobre una de las personas listadas en la pantalla principal.
\newline 2. Se muestra al usuario la pantalla de detalle de persona, donde se encuentran los datos de la misma y un listado de las sesiones realizadas.\\
\hline
Secuencia alternativa & No hay.\\
\hline
Postcondición & Se visualizan los datos de la persona seleccionada y se muestra el listado de sesiones realizadas y la sesión que actualmente tenga en progreso la persona, en caso de que tuviese alguna.\\
\hline
Importancia & Vital.\\
\hline
\end{tabular}
\end{center}
\caption{CU02}
\end{table} 

\begin{table}[H]
\label{CU03}
\begin{center}
\begin{tabular}{| l | p{10cm} |}
\hline
CU03 & Eliminar persona.\\
\hline
Descripción & El usuario elimina la persona seleccionada del sistema, así como todos los datos y resultados asociados a la misma.\\
\hline
Precondición & El usuario debe estar en la pantalla de detalle de usuario, por lo que debe existir al menos una persona registrada en la aplicación y el usuario debe haber pulsado sobre una persona del listado de personas registradas de la pantalla principal.\\
\hline
Secuencia normal & 1. El usuario pulsa sobre una de las personas listadas en la pantalla principal.
\newline 2. Se muestra al usuario la pantalla de detalle de persona, donde se encuentran los datos de la misma y un listado de las sesiones realizadas.
\newline 3. El usuario pulsa sobre el icono de la papelera, situado en la esquina superior izquierda.
\newline 4. Se muestra al usuario un popup de confirmación, donde el usuario puede confirmar o cancelar la acción.
\newline 5. El usuario confirma la acción y el sistema elimina la persona seleccionada y retorna al usuario a la pantalla principal de la aplicación.\\
\hline
Secuencia alternativa & 5a. El usuario cancela la acción. Vuelve al paso 2.\\
\hline
Postcondición & Se elimina la persona seleccionada del sistema exitosamente, así como los datos y resultados asociados a la misma.\\
\hline
Importancia & Vital.\\
\hline
\end{tabular}
\end{center}
\caption{CU03}
\end{table} 

\subsubsection{4.1.2.2.3. Subsistema de gestión de sesiones}

\begin{table}[H]
\label{CU04}
\begin{center}
\begin{tabular}{| l | p{10cm} |}
\hline
CU04 & Comenzar sesión.\\
\hline
Descripción & El usuario pulsa sobre el botón de la pesa situado en la esquina superior derecha de la pantalla principal y la aplicación muestra el listado de personas registradas en el sistema para que el usuario seleccione la persona que comenzará la sesión.\\
\hline
Precondición & Debe existir al menos una persona registrada en la aplicación y el usuario debe haber pulsado sobre el botón de la pesa situado en la esquina superior derecha de la pantalla principal.\\
\hline
Secuencia normal & 1. El usuario pulsa sobre el icono de la pesa situado en la esquina superior derecha de la pantalla principal.
\newline 2. Se muestra la pantalla el listado de personas registradas en la aplicación.\\
\hline
Secuencia alternativa & 2a. Si no hay ninguna persona registrada en la aplicación, se muestra un mensaje indicando que debe existir al menos una persona registrada en el sistema.\\
\hline
Postcondición & Se muestra el listado de personas registradas en la aplicación para que el usuario seleccione la persona que comenzará la sesión.\\
\hline
Importancia & Vital.\\
\hline
\end{tabular}
\end{center}
\caption{CU04}
\end{table}

\begin{table}[H]
\label{CU05}
\begin{center}
\begin{tabular}{| l | p{10cm} |}
\hline
CU05 & Seleccionar persona.\\
\hline
Descripción & El usuario pulsa sobre uno de los usuarios listados y, en caso de que el usuario no tenga ningúna sesión en progreso, se creará la sesión en el sistema para el usuario seleccionado. Por último se mostrará la pantalla de selección de test.\\
\hline
Precondición & El usuario se debe encontrar en la pantalla de selección de usuario y debe existir al menos una persona registrada en la aplicación.\\
\hline
Secuencia normal & 1. El usuario selecciona una de las personas listadas.
\newline 2. Se crea la sesión en el sistema para la persona seleccionada en caso de que el usuario no tuviese ninguna sesión en progreso.
\newline 3. Se muestra la pantalla de selección de test.\\
\hline
Secuencia alternativa & 2a. La persona seleccionada ya tiene una sesión creada y en progreso. Salta al punto 3, mostrándose los resultados obtenidos en aquellos tests que ya hayan sido completados para la sesión que tiene en progreso.\\
\hline
Postcondición & Se crea la sesión en el sistema satisfactoriamente para la persona seleccionada en caso de que no tuviese ninguna sesión en progreso y se muestra al usuario la pantalla de selección de test.\\
\hline
Importancia & Vital.\\
\hline
\end{tabular}
\end{center}
\caption{CU05}
\end{table}

\begin{table}[H]
\label{CU06}
\begin{center}
\begin{tabular}{| l | p{10cm} |}
\hline
CU06 & Seleccionar test.\\
\hline
Descripción & El usuario pulsa sobre uno de los test listados que no estén ya completados y se mostrará la pantalla de realización del test correspondiente.\\
\hline
Precondición & El usuario debe encontrarse en la pantalla de selección de test y la persona seleccionada debe tener una sesión creada y no completada.\\
\hline
Secuencia normal & 1. El usuario selecciona uno de los tests que no estén ya completados.
\newline 2. Se muestra la pantalla de realización del test correspondiente.\\
\hline
Secuencia alternativa & 1a. El usuario selecciona un test que ya estaba completado. Se muestra un mensaje en pantalla indicando que el test ya ha sido realizado y el usuario permanece en la pantalla de selección de test.\\
\hline
Postcondición & Se accede satisfactoriamente a la pantalla de realización del test.\\
\hline
Importancia & Vital.\\
\hline
\end{tabular}
\end{center}
\caption{CU06}
\end{table}

\begin{table}[H]
\label{CU07}
\begin{center}
\begin{tabular}{| l | p{10cm} |}
\hline
CU07 & Ver información de test.\\
\hline
Descripción & El usuario pulsa sobre el icono de información situado a la derecha de cada test en la pantalla de selección de test. El sistema entonces muestra una pantalla con explicaciones sobre cómo realizar el test para el que se está consultando la información.\\
\hline
Precondición & El usuario debe encontrarse en la pantalla de selección de test y la persona seleccionada debe tener una sesión creada y no completada.\\
\hline
Secuencia normal & 1. El usuario pulsa sobre el botón de información que aparece a la derecha de los tests no completados en la pantalla de selección de test.
\newline 2. Se muestra la pantalla de información del test, con las explicaciones pertinentes sobre como realizar el ejercicio.\\
\hline
Secuencia alternativa & No hay.\\
\hline
Postcondición & Se accede satisfactoriamente a la pantalla de información del test.\\
\hline
Importancia & Vital.\\
\hline
\end{tabular}
\end{center}
\caption{CU07}
\end{table}

\begin{table}[H]
\label{CU08}
\begin{center}
\begin{tabular}{| l | p{10cm} |}
\hline
CU08 & Reanudar sesión\\
\hline
Descripción & El usuario pulsa sobre la sesión en progreso que aparece listada en la pantalla de detalle de persona, mostrándose la pantalla de selección de test.\\
\hline
Precondición & El usuario debe encontrarse en la pantalla de detalle de usuario y la persona seleccionada debe tener una sesión creada y no completada.\\
\hline
Secuencia normal & 1. El usuario pulsa sobre la sesión en progreso listada para el usuario del que estamos consultando el detalle.
\newline 2. Se muestra la pantalla de selección de test, diferenciando aquellos que ya estén completados de los pendientes y mostrando los resultados para los que ya están realizados por el usuario para esa sesión.\\
\hline
Secuencia alternativa & No hay.\\
\hline
Postcondición & Se accede satisfactoriamente a la pantalla de selección de test y se muestran los resultados de aquellos que ya estuviesen completados para la sesión, en caso de que los hubiese.\\
\hline
Importancia & Vital.\\
\hline
\end{tabular}
\end{center}
\caption{CU08}
\end{table}

\subsubsection{4.1.2.2.4. Subsistema de resultados y cálculos de estadísticas}

\begin{table}[H]
\label{CU09}
\begin{center}
\begin{tabular}{| l | p{10cm} |}
\hline
CU09 & Ver estadísticas de sesión\\
\hline
Descripción & El usuario pulsa sobre cualquier sesión ya completada de las que aparecen listadas en la pantalla de detalle de persona, mostrándose la pantalla de estadísticas de la sesión.\\
\hline
Precondición & El usuario debe encontrarse en la pantalla de detalle de usuario y la persona seleccionada debe tener al menos una sesión ya completada.\\
\hline
Secuencia normal & 1. El usuario pulsa sobre cualquiera de las sesiones listadas como completadas para el usuario del que estamos consultando el detalle.
\newline 2. Se muestra la pantalla de estadísticas de la sesión completada seleccionada en el paso anterior.\\
\hline
Secuencia alternativa & No hay.\\
\hline
Postcondición & Se accede satisfactoriamente a la pantalla de estadísticas de la sesión completada seleccionada en la pantalla de detalle de usuario.\\
\hline
Importancia & Vital.\\
\hline
\end{tabular}
\end{center}
\caption{CU09}
\end{table}

\begin{table}[H]
\label{CU10}
\begin{center}
\begin{tabular}{| l | p{10cm} |}
\hline
CU10 & Mostrar resultados\\
\hline
Descripción & Cuando el usuario realiza el último test que quede por completar de la sesión, se mostrará al usuario la pantalla de resultados, que mostrará los resultados obtenidos en cada uno de los tests durante la sesión que se acaba de completar.\\
\hline
Precondición & El usuario debe haber completado todos los tests de la sesión que en ese momento tuviese en progreso.\\
\hline
Secuencia normal & 1. El usuario completa el último test que tuviese pendiente de la sesión que tuviese en progreso en el momento de la realización del test indicado.
\newline 2. Una vez obtenido el resultado del último test, se muestra la pantalla de resultados de todos los tests completados para la sesión que recién se ha acabado de completar.\\
\hline
Secuencia alternativa & 1a. El test que se ha completado no era el último y había algún test más pendiente para la sesión actual. Se vuelve a la pantalla de selección de test.\\
\hline
Postcondición & Se muestra satisfactoriamente la pantalla de resultados de la sesión, mostrándose los valores obtenidos para cada uno de los tests que se han completado.\\
\hline
Importancia & Vital.\\
\hline
\end{tabular}
\end{center}
\caption{CU10}
\end{table}

\subsubsection{4.1.2.2.5. Tests de valoración de la condición física}

\begin{table}[H]
\label{CU11}
\begin{center}
\begin{tabular}{| l | p{10cm} |}
\hline
CU11 & Definir duración de test\\
\hline
Descripción & El usuario define el tiempo (en segundos) que durará el test que se va a realizar. Este caso de uso solo aplica para aquellos tests que usan una cuenta atrás.\\
\hline
Precondición & El usuario debe encontrarse en la pantalla de realización de alguno de los siguientes tests: fuerza de brazos, fuerza de piernas o resistencia aeróbica.\\
\hline
Secuencia normal & 1. El usuario pulsa sobre el tiempo predefinido por defecto (30 segundos).
\newline 2. Se abre el teclado numérico del dispositivo móvil.
\newline 3. El usuario borra el valor que ya esté especificado, introduce el número de segundos deseado y acepta.\\
\hline
Secuencia alternativa & No hay.\\
\hline
Postcondición & Se actualiza satisfactoriamente el tiempo de duración de la prueba, mostrándose en la pantalla el número de segundos introducido por el usuario.\\
\hline
Importancia & Vital.\\
\hline
\end{tabular}
\end{center}
\caption{CU11}
\end{table}

\begin{table}[H]
\label{CU12}
\begin{center}
\begin{tabular}{| l | p{10cm} |}
\hline
CU12 & Iniciar test\\
\hline
Descripción & El usuario pulsa sobre el botón START, activándose la detección de movimiento en el dispositivo móvil y comenzando la cuenta atrás. Además aparecen en pantalla los botones STOP y RESET.\\
\hline
Precondición & El usuario debe encontrarse en la pantalla de realización de alguno de los siguientes tests: fuerza de brazos, fuerza de piernas o resistencia aeróbica.\\
\hline
Secuencia normal & 1. El usuario coloca el dispositivo móvil en el brazo o en la pierna (según sea necesario) de la persona que va a realizar el test.
\newline 2. El usuario pulsa sobre el botón START.\\
\hline
Secuencia alternativa & No hay.\\
\hline
Postcondición & Se activa la detección de movimiento en el dispositivo móvil y comienza la cuenta atrás. El botón START desaparece y en su lugar aparecen los botones STOP y RESET.\\
\hline
Importancia & Vital.\\
\hline
\end{tabular}
\end{center}
\caption{CU12}
\end{table}

\begin{table}[H]
\label{CU13}
\begin{center}
\begin{tabular}{| l | p{10cm} |}
\hline
CU13 & Detectar movimiento\\
\hline
Descripción & El dispositivo móvil, colocado en el brazo o en la pierna con el/la que se realizará los ejercicios, detectará el movimiento realizado y contará las repeticiones que se hayan realizado de forma correcta, emitiendo un sonido por cada repetición contabilizada.\\
\hline
Precondición & El usuario debe encontrarse en la pantalla de realización de alguno de los siguientes tests: fuerza de brazos, fuerza de piernas o resistencia aeróbica. Debe tener colocado el móvil en el brazo o en la pierna, según el test a realizar y debe haber pulsado sobre el botón START, encontrándose el test en progreso y sin que haya finalizado la cuenta atrás.\\
\hline
Secuencia normal & 1. La persona que esté realizando el ejercicio y tenga el dispositivo móvil colocado en la extremidad con la que esté realizando la actividad, hace el movimiento requerido por el test mientras la cuenta atrás siga en progreso.\\
\hline
Secuencia alternativa & 1a. El movimiento realizado no es correcto. El sistema no contabiliza la repetición y no emite ningún sonido, manteniéndose inalterado el número de repeticiones que se muestra en pantalla.
\newline 1b. La persona realiza el movimiento cuando la cuenta atrás ha finalizado. El sistema no contabiliza la repetición y no emite ningún sonido, dado que al haber acabado la cuenta atrás el dispositivo móvil ya no detecta el movimiento realizado.\\
\hline
Postcondición & El sistema contabiliza la repetición, aumentando en 1 el número de repeticiones mostradas en pantalla y emitiendo un sonido.\\
\hline
Importancia & Vital.\\
\hline
\end{tabular}
\end{center}
\caption{CU13}
\end{table}

\begin{table}[H]
\label{CU14}
\begin{center}
\begin{tabular}{| l | p{10cm} |}
\hline
CU14 & Guardar resultado\\
\hline
Descripción & El usuario pulsa sobre el botón GUARDAR y el sistema almacena en la base de datos el resultado obtenido en el test para la sesión en progreso y para la persona que lo haya realizado.\\
\hline
Precondición & El usuario debe encontrarse en la pantalla de realización del test. Para que el botón GUARDAR aparezca, el test debe haberse completado. Según el tipo de test, se considerará completado cuando se cumplan una de las siguientes condiciones: que la cuenta atrás haya terminado, que el cronómetro se haya parado por el usuario o que se haya introducido la distancia medida para los tests de flexibilidad.\\
\hline
Secuencia normal & 1. Se finaliza el ejercicio o se completa la información requerida por el test.
\newline 2. El sistema muestra el botón GUARDAR.
\newline 3. El usuario pulsa sobre el botón GUARDAR.
\newline 4. El sistema almacena en la base de datos el resultado del test para la sesión en progreso y la persona que lo haya realizado y lleva a la pantalla de selección de test, donde se muestra el resultado obtenido junto al test recién completado.\\
\hline
Secuencia alternativa & 1a. El test aún no se ha completado. Volver al paso 1.
\newline 4a. El test recién completado era el último que quedaba pendiente de la sesión. En lugar de llevar a la pantalla de selección de test, lleva a la pantalla de resultados de la sesión.\\
\hline
Postcondición & El sistema almacena satisfactoriamente en la base de datos el resultado del test para la sesión en progreso y la persona que lo haya realizado y lleva a la pantalla de selección de test, donde se muestra el resultado recién almacenado.\\
\hline
Importancia & Vital.\\
\hline
\end{tabular}
\end{center}
\caption{CU14}
\end{table}

\begin{table}[H]
\label{CU15}
\begin{center}
\begin{tabular}{| l | p{10cm} |}
\hline
CU15 & Parar test\\
\hline
Descripción & La realización del test se para, de forma que el dispositivo móvil deja de detectar el movimiento y la cuenta atrás se detiene. Se muestra el botón RESET para volver los contadores a 0 si así se desea.\\
\hline
Precondición & El usuario debe encontrarse en la pantalla de realización de alguno de los siguientes tests: fuerza de brazos, fuerza de piernas o resistencia aeróbica y el test debe estar en progreso para que el botón STOP se muestre.\\
\hline
Secuencia normal & 1. El usuario pulsa sobre el botón STOP.\\
\hline
Secuencia alternativa & No hay.\\
\hline
Postcondición & Tanto la cuenta atrás como la detección de movimiento del dispositivo móvil se paran. Se permanece en la pantalla de realización de test, mostrando valores obtenidos hasta el momento y apareciendo el botón RESET.\\
\hline
Importancia & Vital.\\
\hline
\end{tabular}
\end{center}
\caption{CU15}
\end{table}

\begin{table}[H]
\label{CU16}
\begin{center}
\begin{tabular}{| l | p{10cm} |}
\hline
CU16 & Resetear test\\
\hline
Descripción & Restablece todos los contadores y temporizadores a sus valores iniciales, volviéndose a mostrar la pantalla como en su estado inicial.\\
\hline
Precondición & El usuario debe encontrarse en la pantalla de realización de alguno de los siguientes tests: fuerza de brazos, fuerza de piernas o resistencia aeróbica. El test debe estar en progreso o parado (tras haber estado en progreso) para que el botón RESET se muestre.\\
\hline
Secuencia normal & 1. El usuario pulsa sobre el botón RESET.\\
\hline
Secuencia alternativa & No hay.\\
\hline
Postcondición & La cuenta atrás se inicializa y las repeticiones contabilizadas hasta el momento se ponen a 0. La detección de movimiento del dispositivo móvil se deshabilita. Se permanece en la pantalla de realización de test, con todos los contadores mostrando los valores iniciales y apareciendo únicamente el botón START.\\
\hline
Importancia & Vital.\\
\hline
\end{tabular}
\end{center}
\caption{CU16}
\end{table}

\begin{table}[H]
\label{CU17}
\begin{center}
\begin{tabular}{| l | p{10cm} |}
\hline
CU17 & Iniciar cronómetro\\
\hline
Descripción & El usuario pulsa sobre el botón START, activándose el cronómetro. Además aparecen en pantalla los botones STOP y RESET.\\
\hline
Precondición & El usuario debe encontrarse en la pantalla de realización del test de Agilidad y el cronómetro no debe estar funcionando.\\
\hline
Secuencia normal & 1. El usuario pulsa sobre el botón START.\\
\hline
Secuencia alternativa & No hay.\\
\hline
Postcondición & El cronómetro comienza a funcionar y se va mostrando el tiempo transcurrido en pantalla. El botón START desaparece y en su lugar aparecen los botones STOP y RESET.\\
\hline
Importancia & Vital.\\
\hline
\end{tabular}
\end{center}
\caption{CU17}
\end{table}

\begin{table}[H]
\label{CU18}
\begin{center}
\begin{tabular}{| l | p{10cm} |}
\hline
CU18 & Parar cronómetro\\
\hline
Descripción & El cronómetro se para, mostrando en pantalla el tiempo transcurrido hasta el momento. Se muestra el botón RESET para volver los contadores a 0 si así se desea y el botón GUARDAR.\\
\hline
Precondición & El usuario debe encontrarse en la pantalla de realización del test de Agilidad y el cronómetro debe estar funcionando.\\
\hline
Secuencia normal & 1. El usuario pulsa sobre el botón STOP.\\
\hline
Secuencia alternativa & No hay.\\
\hline
Postcondición & Se para el cronómetro y se muestra en pantalla el tiempo registrado hasta el momento. Además aparecen en pantalla el botón RESET, por si se quiere resetear el cronómetro, y el botón GUARDAR, por si se quiere guardar el tiempo obtenido durante la realización del ejercicio.\\
\hline
Importancia & Vital.\\
\hline
\end{tabular}
\end{center}
\caption{CU18}
\end{table}

\begin{table}[H]
\label{CU19}
\begin{center}
\begin{tabular}{| l | p{10cm} |}
\hline
CU19 & Resetear cronómetro\\
\hline
Descripción & Restablece el cronómetro a 0 y se ocultan todos los botones, volviéndose a mostrar la pantalla como en su estado inicial.\\
\hline
Precondición & El usuario debe encontrarse en la pantalla de realización del test de Agilidad y el cronómetro debe estar en funcionamiento o parado (tras haber estado en funcionamiento) para que el botón RESET se muestre.\\
\hline
Secuencia normal & 1. El usuario pulsa sobre el botón RESET.\\
\hline
Secuencia alternativa & No hay.\\
\hline
Postcondición & El cronómetro se inicializa a 0. Se permanece en la pantalla de realización de test pero volviendo a su estado inicial y apareciendo únicamente el botón START.\\
\hline
Importancia & Vital.\\
\hline
\end{tabular}
\end{center}
\caption{CU19}
\end{table}

\begin{table}[H]
\label{CU20}
\begin{center}
\begin{tabular}{| l | p{10cm} |}
\hline
CU20 & Introducir distancia\\
\hline
Descripción & El usuario introduce la medida obtenida tras realizar el test de flexibilidad de brazos o de piernas a la persona que realiza la sesión.\\
\hline
Precondición & El usuario debe encontrarse en la pantalla de realización de alguno de los siguientes tests: flexibilidad de brazos o flexibilidad de piernas.\\
\hline
Secuencia normal & 1. El usuario pulsa sobre el input donde introducirá la distancia en centímetros.
\newline 2. Se abre el teclado numérico del dispositivo móvil.
\newline 3. El usuario introduce la distancia obtenida y acepta.\\
\hline
Secuencia alternativa & No hay.\\
\hline
Postcondición & Se muestra satisfactoriamente en la pantalla la distancia introducida y se muestra el botón GUARDAR para almacenar en el sistema el valor introducido para la sesión y la persona que ha realizado el test.\\
\hline
Importancia & Vital.\\
\hline
\end{tabular}
\end{center}
\caption{CU20}
\end{table}

\subsection{Requisitos no funcionales}

En este apartado se detallarán los requisitos no funcionales del sistema.

\begin{table}[H]
\label{RNF01}
\begin{center}
\begin{tabular}{| l | p{10cm} |}
\hline
RNF01 & Usabilidad.\\
\hline
Descripción & El sistema debe ser usable y disponer de una interfaz intuitiva, adaptable y fácil de manejar para un usuario de nivel básico.\\
\hline
Importancia & Vital.\\
\hline
\end{tabular}
\end{center}
\caption{RNF01}
\end{table} 

\begin{table}[H]
\label{RNF02}
\begin{center}
\begin{tabular}{| l | p{10cm} |}
\hline
RNF02 & Portabilidad.\\
\hline
Descripción & Debe ser posible usar la aplicación en diferentes dispositivos móviles Android, independientemente del tamaño de la pantalla de los mismos, gracias al diseño adaptable empleado en las interfaces de usuario.\\
\hline
Importancia & Vital.\\
\hline
\end{tabular}
\end{center}
\caption{RNF02}
\end{table}

\begin{table}[H]
\label{RNF03}
\begin{center}
\begin{tabular}{| l | p{10cm} |}
\hline
RNF03 & Escabilidad.\\
\hline
Descripción & El sistema debe responder de manera óptima y eficiente a futuras mejoras en el desarrollo de la aplicación, sin que éstas comprometan el estado de la misma. El código debe ser mantenible y fácilmente ampliable para futuras versiones.\\
\hline
Importancia & Vital.\\
\hline
\end{tabular}
\end{center}
\caption{RNF03}
\end{table}

\begin{table}[H]
\label{RNF04}
\begin{center}
\begin{tabular}{| l | p{10cm} |}
\hline
RNF04 & Seguridad\\
\hline
Descripción & Cuando la aplicación necesite hacer uso de algunas características del dispositivo móvil que puedan comprometer la privacidad/seguridad de los datos del usuario, como puede ser la cámara o el acceso de lectura/escritura a los archivos de la memoria del dispositivo, siempre solicitará permiso al usuario antes de hacer uso de las mismas. En caso de que no se conceda el permiso, la aplicación seguirá siendo funcional aunque no se pueda hacer uso de las características para las que no se ha concedido el permiso.\\
\hline
Importancia & Vital.\\
\hline
\end{tabular}
\end{center}
\caption{RNF04}
\end{table} 

\begin{table}[H]
\label{RNF05}
\begin{center}
\begin{tabular}{| l | p{10cm} |}
\hline
RNF05 & Rendimiento\\
\hline
Descripción & El rendimiento de la aplicación de ser tal que permita un desempeño agradable y suave durante el uso de la misma. Los tiempos de respuesta al insertar/recuperar información en la base de datos serán cortos y deberá minimizarse la utilización de recursos cuando sea posible para ahorrar batería.\\
\hline
Importancia & Vital.\\
\hline
\end{tabular}
\end{center}
\caption{RNF05}
\end{table}

\begin{table}[H]
\label{RNF06}
\begin{center}
\begin{tabular}{| l | p{10cm} |}
\hline
RNF06 & Diseño\\
\hline
Descripción & Como se ha indicado en el requisito anterior, minimizar la utilización de recursos y el tiempo de respuesta debe primar sobre cualquier factor. Sin embargo, se debe tener en cuenta que los dispositivos móviles Android (Java) pueden perder el contexto de la aplicación que ejecutan, por lo que el contenido debe ser recuperable. Será de valor añadido que el paquete .APK final tenga el menor tamaño posible para abarcar aquellos teléfonos con menos memoria secundaria.\\
\hline
Importancia & Vital.\\
\hline
\end{tabular}
\end{center}
\caption{RNF06}
\end{table}

\subsection{Reglas de negocio}

El sistema en su totalidad está desarrollado para ser software libre bajo licencia GNU GPL. Ello permite que el código sea accesible a cualquier desarrollador que quiera incrementar su funcionalidad y contribuir a la libre creación de contenidos. Así mismo, las herramientas y tecnologías empleadas también son de libre uso.

\subsection{Requisitos de información}

A continuación se detallarán los requisitos de información, que describen cómo gestiona el sistema la información que se va a almacenar, y son los siguientes:

\begin{table}[H]
\label{RIN01}
\begin{center}
\begin{tabular}{| l | p{10cm} |}
\hline
RIN01 & Información de la persona.\\
\hline
Descripción & El sistema debe almacenar la información sobre las personas mayores registradas en la aplicación.\\
\hline
Datos específicos & DNI de la persona.
\newline Nombre de la persona.
\newline Apellidos de la persona.
\newline Género de la persona.
\newline Fecha de nacimiento de la persona.
\newline Foto asociada a la persona.\\
\hline
Importancia & Vital.\\
\hline
\end{tabular}
\end{center}
\caption{RIN01}
\end{table} 

\begin{table}[H]
\label{RIN02}
\begin{center}
\begin{tabular}{| l | p{10cm} |}
\hline
RIN02 & Información de la sesión.\\
\hline
Descripción & El sistema debe almacenar la información correspondiente a las sesiones de ejercicio que realizan las personas registradas en la aplicación.\\
\hline
Datos específicos & Identificador de la persona que realiza la sesión.
\newline Estado de la sesión.
\newline Fecha de inicio de la sesión.\\
\hline
Importancia & Vital.\\
\hline
\end{tabular}
\end{center}
\caption{RIN02}
\end{table} 

\begin{table}[H]
\label{RIN03}
\begin{center}
\begin{tabular}{| l | p{10cm} |}
\hline
RIN03 & Información del test.\\
\hline
Descripción & El sistema debe almacenar la información correspondiente a los diferentes tests que realizarán las personas y que componen la batería de pruebas Senior Fitness Test (y por tanto cada una de las sesiones).\\
\hline
Datos específicos & Nombre del test.
\newline Descripción del test.\\
\hline
Importancia & Vital.\\
\hline
\end{tabular}
\end{center}
\caption{RIN03}
\end{table} 

\begin{table}[H]
\label{RIN04}
\begin{center}
\begin{tabular}{| l | p{10cm} |}
\hline
RIN04 & Información del resultado.\\
\hline
Descripción & El sistema debe almacenar los resultados que las personas obtienen en cada uno de los tests que componen la batería de pruebas Senior Fitness Test, y que estarán asociados a la sesión que la persona haya realizado en una fecha concreta.\\
\hline
Datos específicos & Identificador de la persona que ha realizado el test.
\newline Identificador de la sesión en la que la persona ha realizado el test.
\newline Identificador del test realizado.
\newline Resultado obtenido en el test.
\newline Fecha de obtención del resultado.\\
\hline
Importancia & Vital.\\
\hline
\end{tabular}
\end{center}
\caption{RIN04}
\end{table} 

\section{Modelo conceptual de datos}

En esta sección se muestra el diagrama conceptual de datos UML de la aplicación, con el que se visualizarán a modo de primer vistazo las clases que posteriormente se diseñarán e implementarán para cubrir las funcionalidades impuestas en el proyecto.\\

En la siguiente imagen podemos ver el diagrama Entidad-Relación que representa las entidades relevantes del sistema de información implementado, así como sus interrelaciones:

\figura{modeloconceptual_er.png}{scale=0.6}{Análisis: Modelo conceptual Entidad-Relación}{modeloconceptual_er}{H}

\section{Modelo de comportamiento del sistema}

Este modelo de comportamiento especifica como debe actuar el sistema. El sistema es el que engloba todos los objetos, y el modelo consta de dos partes:

\begin{itemize}
\item Diagramas de secuencias del sistema: Muestran la secuencia de eventos entre el usuario y el sistema.
\item Contrato de las operaciones del sistema: Describen el efecto que producen las operaciones en el sistema.
\end{itemize}

\subsection{Diagramas de secuencia y contrato de las operaciones del sistema}

A continuación se mostrarán los diagramas de secuencia. No todos los posibles diagramas de secuencia aparecerán, sino que nos centraremos en los más importantes. Además, para evitar contenidos duplicados, se omitirán las operaciones que ya hayan sido explicadas con anterioridad.
















\chapter{Diseño del sistema}
% -*-cap2.tex-*-
% Este fichero es parte de la plantilla LaTeX para
% la realización de Proyectos Final de Carrera, protejido
% bajo los términos de la licencia GFDL.
% Para más información, la licencia completa viene incluida en el
% fichero fdl-1.3.tex

% Copyright (C) 2009 Pablo Recio Quijano 

En este capítulo se describen todos los aspectos relacionados con el diseño del sistema: arquitectura del sistema, patrones de diseño, diseño físico de datos, así como el diseño de la interfaz de usuario.

\section{Arquitectura física del sistema}

A continuación se presentan los elementos hardware que componen la arquitectura física del sistema. Para definirla, se detallarán los componentes a nivel de hardware requeridos para el correcto funcionamiento de la aplicación.\\

Cuando hablamos de un dispositivo móvil o smartphone, nos referimos a los sucesores de los teléfonos móviles sencillos, que han ido evolucionando hasta ser prácticamente ordenadores que caben en nuestros bolsillos o en nuestra palma de la mano, y que además incluyen las funciones básicas de sus antecesores, como pueden ser las llamadas, envío y recibo de sms, mms, etc.\\

En los últimos años los dispositivos móviles han empezado a cobrar importancia debido a su relación entre la potencia y su reducido tamaño, llegando incluso a ser indispensables en ciertos sectores de la sociedad hoy en día.\\

Junto con la aparición de los smartphones se presentaron a su vez los sistemas operativos que éstos dispositivos llevarían integrados como pueden ser Android o IOS, cada uno con características propias, ofreciendo así un entorno de fácil manejo para el usuario final y un sinfín de aplicaciones desarrolladas tanto por empresas como por particulares.\\

Dado que la presente aplicación ha sido desarrollada en Android, los requisitos a nivel de hardware se reducen a los siguientes:

\begin{itemize}
\item Dispositivos: Dispositivo móvil basado en Android.
\item Requisitos mínimos:
\begin{itemize}
\item La versión de Android instalada en el dispositivo debe ser superior o igual a la versión 4.0.3 (ICE\_CREAM\_SANDWICH\_MR1).
\item El dispositivo móvil debe disponer de Acelerómetro y Giroscopio.
\end{itemize}
\end{itemize}

Es conveniente destacar la necesidad de que el dispositivo móvil disponga de Acelerómetro y Giroscopio, dado que la combinación de estos dos sensores hardware son los que permiten que exista el sensor de Gravedad que, a diferencia de los dos sensores anteriormente citados, se trata de un sensor software.

\section{Arquitectura lógica del sistema}

En la arquitectura lógica del sistema se detallan los componentes a nivel de software empleados a lo largo del proyecto, que abarcan tanto al conjunto de aplicaciones, bibliotecas y librerías como al propio software desarrollado para cumplir con los objetivos y requisitos establecidos.

\subsection{Android}

\subsubsection{Introducción}

Android puede entenderse como una plataforma software cuya misión consiste en abstraer el hardware subyacente. Inicialmente fue desarrollado para dispositivos táctiles con recursos limitados, con el fin de facilitar el desarrollo de aplicaciones para dichos dispositivos.\\

La gran diferencia de Android respecto al resto de sistemas operativos para dispositivos móviles es su núcleo basado en GNU/Linux. Esto hace que Android adquiera algunas de las principales características de Linux, convirtiéndose en un software libre, gratuito y multiplataforma. \\

Inicialmente fue desarrollado por Android IC, pero en 2005 fue comprado por Google, y hasta entonces era un sistema operativo muy poco conocido. En el 2007 se fundó la Opend Handset Alliance, una agrupación de empresas de desarrollo de software, hardware y telecomunicaciones con el propósito de avanzar en los estándares abiertos para el desarrollo de software y hardware para dispositivos móviles, y fue entonces cuando se produjo la presentación de Android por parte de Google, liberando gran parte de su código bajo una licencia Apache.\\

El éxito tras la presentación fue escaso, debido a que el sistema operativo se presentó antes de que se comercializara ningún dispositivo que lo incluyese. En la actualidad es el sistema operativo más utilizado.

\subsubsection{Arquitectura}

La arquitectura del sistema Android puede verse como una arquitectura por capas o niveles, de forma que cada nivel puede utilizar servicios ofrecidos por los niveles anteriores y éste, a su vez, proporciona nuevas funciones a los niveles superiores. En la siguiente imagen podemos ver los niveles que componen la arquitectura:

\figura{arquitectura_android.png}{scale=0.9}{Arquitectura Android}{arquitectura_android}{H}

\begin{itemize}
\item \textbf{Aplicaciones:} Constituye el conjunto de aplicaciones presentes en un dispositivo, ya sean las instaladas por el usuario o por defecto, las nativas y las administradas.
\item \textbf{Framework de aplicaciones:} Plataforma de desarrollo que facilita la reutilización de componentes, permite el acceso a los diversos servicios ofrecidos y al hardware de los dispositivos. Los más importantes son:
\begin{itemize}
\item Activity Manager: Conjunto de APIs encargadas de gestionar el ciclo de vida de las aplicaciones.
\item Window Manager: Gestiona las ventanas de una aplicación mediante la librería Surface Manager.
\item Telephone Manager: Compendio de APIs que gestionan las funciones básicas de los teléfonos (llamadas, mensajería, etc).
\item Content Provider: Proporciona los mecanismos necesarios para la comunicación entre aplicaciones.
\item View System: Ofrece los elementos básicos necesarios para la construcción de interfaces.
\item Location Manager: Posibilita a las aplicaciones el acceso a la ubicación del dispositivo.
\item Notification Manager: Permite a las aplicaciones notificar al usuario información asociada a ciertos eventos que ocurran durante la ejecución de una aplicación.
\item XMPP Service: Colección de APIs para el uso de este protocolo de intercambio basado en XML.
\item Resource Manager: Encargado de gestionar todos los elementos que forman parte de una aplicación y son externos al código.
\item Package Manager: Gestor de todos los paquetes instalados en un dispositivo Android y que permite la instalación de nuevos paquetes.
\end{itemize}
\item \textbf{Bibliotecas:} Hacen referencia al conjunto de librerías presentes en Android y proporcionan la mayoría de las características más representativas de esta plataforma. Las principales librerías que podemos encontrar son las siguientes:
\begin{itemize}
\item Surface Manager: Gestión de la pantalla.
\item SQLite: Motor de bases de datos relacionales. Es el motor de bases de datos usado en la presente aplicación.
\item Media Framework: Reproducción de imágenes, vídeo y audio.
\item WebKit: Navegación web.
\item SGL: Gráficos 2D
\item Open GL/ES: Gráficos 3D.
\item FreeType: Renderizado de fuentes.
\item SSL: Comunicación segura mediante sockets.
\item Libcr: Variante optimizada de C.
\end{itemize}
\item \textbf{Android Runtime:} Al mismo nivel que las librerías encontramos el entorno de ejecución, que está constituido por las librerías Java que forman el núcleo del lenguaje y la máquina virtual Dalvik o ART para las versiones más modernas.
\item \textbf{Núcleo Linux:} Se utiliza como una capa de abstracción para el hardware subyacente y por tanto contiene los drivers y controladores necesarios para el correcto funcionamiento del mismo.
\end{itemize}

\subsubsection{Componentes de una aplicación Android}

Los componentes de una aplicación Android son los elementos esenciales que la componen. Cada uno de ellos supone un punto de entrada mediante el cual el sistema puede acceder a la aplicación, aunque no todos suponen puntos de entrada para el usuario.\\

Cada componente es un bloque básico que desempeña un papel específico en el funcionamiento de nuestra aplicación. Existen cuatro tipos diferentes de componentes, cada uno de ellos con un ciclo de vida y propósito único. Los componentes son los que siguen a continuación:

\begin{itemize}
\item \textbf{Actividades o Activities:} Cada actividad o activity respresenta una pantalla de nuestra aplicación. Las actividades trabajan en conjunto para dar una visión coherente de la aplicación, sin embargo, la vida de cada actividad es independiente del resto. El ciclo de vida que sigue una actividad se muestra en la siguiente imagen:
\figura{ciclo_vida_activity.png}{scale=0.9}{Arquitectura Android: Ciclo de vida de una Activity}{ciclo_vida_activity}{H}
\item \textbf{Servicios o Services:} Son componentes que se ejecutan en segundo plano con el fin de realizar operaciones de larga duración o procesos remotos. Los servicios carecen de interfaz y generalmente son llamados por las activities para realizar tareas costosas sin bloquear la interfaz, es decir, se pueden dejar en ejecución en segundo plano o logarlos a una actividad para interactuar con ella. En la siguiente imagen veremos el ciclo de vida de un servicio. En la izquierda está actuando en segundo plano y en la derecha está ligado a una activity:
\figura{ciclo_vida_servicio.png}{scale=0.9}{Arquitectura Android: Ciclo de vida de un Servicio}{ciclo_vida_servicio}{H}
\item \textbf{Proveedor de contenido o Content provider:} El proveedor de contenido gestiona los datos que maneja una aplicación. Controla el acceso a archivos, bases de datos SQLite, etc. Mediante el proveedor de contenidos las aplicaciones pueden consultar y modificar datos de otras aplicaciones siempre que posean los permisos adecuados. La siguiente imagen ilustra cómo una activity puede consultar datos de otra aplicación:
\figura{content_provider.png}{scale=0.9}{Arquitectura Android: Content Provider}{content_provider}{H}
\item \textbf{Receptor de difusiones o Broadcast receiver:} Es el componente encargado de responder a las difusiones o anuncios del sistema (por ejemplo, un anuncio del sistema que indica que queda poca batería). Carece de interfaz, pero debe crear una barra de estado para notificar al usuario cuando detexta una difusión. Una aplicación debe registrar un broadcast receiver para indicar qué difusiones le interesan.
\figura{broadcast_receiver.png}{scale=0.8}{Arquitectura Android: Broadcast Receiver}{broadcast_receiver}{H}
\end{itemize}

\subsubsection{Recursos de una aplicación Android}

Los recursos de una aplicación corresponden con todos los ficheros, imágenes, cadenas de texto, etc, que nuestra aplicación utiliza. Cuando trabajamos desarrollando aplicaciones para Android, las buenas prácticas nos dicen que este tipo de archivos debe mantenerse independientemente al código de la aplicación.\\

Este hecho, conocido como externalización de recursos, permite adaptar un mismo código a dispositivos con diferentes configuraciones y características físicas de una forma prácticamente automática. En la siguiente imagen podemos comprobar cómo quedaría una pantalla que no ha sido adaptada para diferentes dispositivos (parte superior) y una que sí (parte inferior):

\figura{externalizacion_recursos.png}{scale=0.9}{Arquitectura Android: Externalización de recursos}{externalizacion_recursos}{H}

\subsubsection{Manifest de una aplicación Android}

El archivo Manifest es un archivo XML que contiene información esencial para Android acerca de la aplicación. Todas las aplicaciones Android deben tener un archivo AndroidManifest.xml en su directorio raíz.\\

Entre todas las funciones que realiza el archivo manifest, las más significativas son las siguientes:

\begin{itemize}
\item Da nombre al paquete Java de la aplicación.
\item Describe los componentes y qué proceso los hospeda.
\item Contiene la declaración de permisos.
\item Lista las librería necesarias.
\item Declara el nivel mínimo de la API de Android requerido.
\end{itemize}

\subsection{Git}

Git es el software de control de versiones que se ha elegido para el desarrollo del proyecto. Fue desarrollado por Linus Torvalds (creador del sistema operativo Linux) y a día de hoy se usa en grandes proyectos, por ejemplo el propio núcleo de Linux. Como es costumbre en el desarrollo de éste proyecto, Git es software libre distribuido bajo la licencia GPL, lo que lo hace una alternativa libre de usar.\\

Git permite tener el código generado en la implementación del sistema en un repositorio (GitHub) y poder acceder a él remotamente desde cualquier PC, así como poder restaurar versiones anteriores en caso que la actual deba ser reemplazada.\\

Como para cualquier proyecto informático, el uso de un repositorio como GitHub para el alojamiento de la aplicación que nos ocupa se antojó necesario, no sólo por el simple hecho de tener todo el sistema alojado en la nube y disponible para cualquier persona, sino por la seguridad que transmite tener todo el desarrollo en un lugar desde donde sea posible descargarlo en cualquier PC y desarrollarlo en éste, por lo que pudiera suceder a lo largo del desarrollo del proyecto.

\subsection{SQLite}

La plataforma Android proporciona dos herramientas principales para el almacenamiento y consulta de datos estructurados, que ya hemos nombrado en los puntos anteriores:

\begin{itemize}
\item Bases de Datos SQLite.
\item Content providers.
\end{itemize}

En este punto nos centraremos en la primera opción, que es la que se usará en la aplicación implementada para el presente proyecto, y que abarcará todas las tareas relacionadas con el almacenamiento de los datos propios de nuestra aplicación.\\

SQLite es un sistema gestor de base de datos relacional (RDBMS) basado en el Lenguaje Estructurado de Consultas (SQL), tal y como sugiere su nombre. Lo que hace único a SQLite es que se considera una solución embebida. La mayoría de los sistemas de gestión de bases de datos como Oracle, MySQL y SQL Server son procesos de servidor autónomos que se ejecutan independientemente, mientras que SQLite es en realidad una librería que está enlazada dentro de las aplicaciones.\\

Se trata de un motor de bases de datos muy popular en la actualidad por ofrecer características tan interesantes como su pequeño tamaño, no necesitar servidor, precisar poca configuración, ser transaccional y por supuesto ser de código libre.\\

Android incorpora de serie todas las herramientas necesarias para la creación y gestión de bases de datos SQLite, y entre ellas una completa API para llevar a cabo de manera sencilla todas las tareas necesarias.\\

En Android, la forma típica de crear, actualizar y conectar con una base de datos SQLite será a través de una clase auxiliar llamada SQLiteOpenHelper, o para ser más exactos, de una clase propia que derive de ella y que debemos personalizar para adaptarnos a las necesidades concretas de nuestra aplicación.

\section{Patrones de diseño}

Los patrones de diseño se pueden definir como los esqueletos de las soluciones a problemas comunes en el desarrollo de software. Éstos brindan una solución ya probada y documentada a problemas de desarrollo de software que están sujetos a contextos similares.\\

Para que una solución sea considerada un patrón, debe poseer ciertas características. Una de ellas es que debe haber comprobado su efectividad resolviendo problemas similares en ocasiones anteriores. Otra es que debe ser reutilizable, lo que significa que es aplicable a diferentes problemas de diseño en distintas circunstancias.\\

Los patrones de diseño pretenden:

\begin{itemize}
\item Proporcionar catálogos de elementos reusables en el diseño de sistemas software.
\item Evitar la reiteración en la búsqueda de soluciones a problemas ya conocidos y solucionados anteriormente.
\item Formalizar un vocabulario común entre diseñadores.
\item Estandarizar el modo en que se realiza el diseño.
\item Facilitar el aprendizaje de las nuevas generaciones de diseñadores, condensando conocimientos ya existentes.
\end{itemize}

Asimismo, no pretenden:

\begin{itemize}
\item Imponer ciertas alternativas de diseño frente a otras.
\item Eliminar la creatividad inherente al proceso de diseño.
\end{itemize}

Para el presente proyecto se han usado patrones de diseño pertenecientes a diferentes subconjuntos de patrones de diseño. Éstos son el patrón Singleton, perteneciente al subconjunto de patrones de diseño creacionales, y el patrón Adapter, perteneciente al subconjunto de patrones de diseño estructurales. A continuación se detallarán ambos para dar conocimiento de cómo funcionan y cómo resuelven determinados problemas del sistema que los implementan.

\subsection{Patrones de diseño creacionales}

Corresponden a patrones de diseño de software que solucionan problemas de creación de instancias. Nos ayudan a encapsular y abstraer dicha creación.

\subsubsection{Patrón de diseño singleton}

El patrón singleton o patrón de instancia única es un patrón de diseño diseñado para restringir la creación de objetos pertenecientes a una clase o el valor de un tipo a un único objeto. Su intención consiste en garantizar que una clase sólo tenga una instancia y proporcionar un punto de acceso global a ella.\\

El patrón singleton se implementa creando en la clase un método que crea una instancia del objeto sólo si todavía no existe alguna. Para asegurar que la clase no puede ser instanciada nuevamente se regula el alcance del constructor haciéndo que éste sea privado y que sea a través de un método específico la manera de crear la primera y única instancia que se va a usar a lo largo del desarrollo.\\

En el desarrollo del presente proyecto, el uso de este patrón se usa para la clase que se ocupa de la definición de la base de datos, ya que a lo largo del proceso de ejecución del sistema sólo es necesario crear una única instancia de dicha clase.\\

\subsection{Patrones de diseño estructurales}

Corresponden a patrones de diseño de software que solucionan problemas de composición (agregación) de clases y objetos.

\subsubsection{Patrón de diseño adapter}

El patrón adapter, también conocido como wrapper, es un patrón de diseño que se utiliza para transformar una interfaz en otra, de tal modo que una clase que no pudiera utilizar otra clase cualquiera, haga uso de ella a través de una segunda. Dicho de otro modo, convierte la interfaz de una clase en otra interfaz que el cliente espera.\\

Este patrón de diseño permite a clases con interfaces incompatibles trabajar juntas mediante un adaptador intermedio, que se encargará de realizar la conversión de una interfaz a otra.\\

En el desarrollo de nuestra aplicación, el uso de este patrón se antoja necesario debido a los listados que se manejan en las diferentes pantallas de la aplicación, dado que, una vez que hemos obtenido de la base de datos el listado de usuarios, tests o sesiones, necesitaremos convertir dichas colecciones de objetos específicos en una colección de vistas (objeto View), que será sobre la que finalmente se iterará en la capa de presentación para pintar cada elemento en la layout correspondiente.

\section{Diseño físico de datos}

En este apartado se detalla el diseño físico de datos usado por nuestra aplicación.\\

Para que se cumplan con los objetivos y los requisitos impuestos en el proyecto, es necesario implementar una estructura capaz de almacenar la información, de forma que se pueda llevar un control y hacer un estudio de los datos obtenidos durante la realización de los tests que componen la batería de pruebas Senior Fitness Test.\\

La información almacenada en la base de datos consistirá tanto en las personas que realizarán los tests, como las sesiones que realizan y los resultados obtenidos en cada una de las pruebas completadas en dichas sesiones.

\subsection{Base de datos}

A continuación se detalla a través del diagrama Entidad-Relación cómo se estructura la base de datos SQLite diseñada para albergar la información citada en el punto anterior.

\figura{modeloconceptual_er_detail.png}{scale=0.9}{Diagrama Entidad-Relación de la base de datos}{ciclo_vida_servicio}{H}

Los atributos identificadores de cada tabla son enteros a excepción del identificador de persona, que se corresponde con el DNI de la misma. Las fechas de los campos date y birthdate son de tipo String pero parseadas desde tipo Date con el siguiente formato dd/MM/yyyy.\\

En el diagrama no se han representado como atributos de las entidades las claves foráneas, aunque sí aparecen en el fichero Contract de la aplicación, en el que se definen la base de datos y así mismo las tablas del proyecto.

\section{Diseño de la interfaz de usuario}

En este punto se especificarán los elementos que componen las distintas interfaces de usuario que existen para la aplicación, detallándose el contenido de cada pantalla representada.

\section{Uso del dispositivo móvil en la realización de ejercicios}

En este punto se especificará cómo debe usarse el dispositivo móvil en aquellos tests de la batería de pruebas Senior Fitness Test en los que podremos usar el dispositivo para que detecte el movimiento realizado durante los ejercicios, contabilizando de forma autónoma aquellas repeticiones que se realizan de forma satisfactoria.

\section{Diseño de clases UML}

En esta sección se va a mostrar el diseño empleado para el diagrama de clases UML del sistema, que mostrará el diseño de las clases que componen el sistema con sus atributos y métodos necesarios para cumplir con los requisitos y funcionalidades del proyecto expuestas en la etapa de análisis.\\

Dada la naturaleza de las aplicaciones Android, la mayor parte de las clases empleadas en el desarrollo de la aplicación serán activities, y cada una de ellas contendrá la lógica específica de una pantalla de la aplicación.\\

De entre todas las pantallas de la aplicación, hay seis que se corresponden con los tests o pruebas que componen la batería de pruebas Senior Fitness Test (una pantalla para cada test). Al tratarse de pantallas con una finalidad y objetivo similar, hay funcionalidades que comparten entre ellas, por lo que se ha optado por la creación de una activity padre llamada ExerciseActivity de la que extenderán las activities asociadas a cada uno de los tests: fuerza de brazos, fuerza de piernas, resistencia aeróbica, agilidad, flexibilidad de piernas y flexibilidad de brazos, simplificando de esta forma drásticamente el desarrollo de los tests que se vayan implementando sucesivamente tras el primero.\\

\section{Diseño de componentes}

Para el diseño de componentes se realizarán los diseños de los diagramas de secuencia que muestran el flujo de información existente entre el usuario y los componentes del sistema.

\subsection{Diagramas de secuencia}

Como es sabido de los diagramas de secuencia, éstos representan los flujos de comunicación que se dan para un determinado caso de uso del sistema, por lo que, dados los casos de uso desarrollados en el capítulo de análisis de requisitos para la aplicación que nos ocupa, se procederá a diseñarlos en base a estos diagramas.\\

Cabe destacar que no todos los posibles diagramas de secuencia aparecerán, sino que nos centraremos en aquellos casos de uso de la aplicación que se consideren mas relevantes.

\chapter{Implementación del sistema}
% -*-cap2.tex-*-
% Este fichero es parte de la plantilla LaTeX para
% la realización de Proyectos Final de Carrera, protejido
% bajo los términos de la licencia GFDL.
% Para más información, la licencia completa viene incluida en el
% fichero fdl-1.3.tex

% Copyright (C) 2009 Pablo Recio Quijano 

En este capítulo vamos a describir y explicar los detalles de la implementación del sistema software y de la solución propuesta.

\section{Solución propuesta}

Debido a los objetivos y requisitos definidos en las anteriores secciones del documento, ya desde un primer momento se antojaba necesario realizar la implementación de una aplicación móvil, por lo que disponíamos de dos opciones a considerar: hacer una aplicación Android o IOS.\\

Bajo esta disyuntiva, los factores que nos llevaron a tomar la decisión de implementar una aplicación móvil Android fueron los siguientes:

\begin{itemize}
\item Como bien es sabido, la base de Android es Java. Java es uno de los lenguajes más universales y extendidos en el mundo del desarrollo, así que un grandísimo número de desarrolladores lo encontrarán más amigable. Esto, sumado a que desde 2008 me dedico profesionalmente al desarrollo de aplicaciones web con Java, fue un factor determinante a la hora tomar la decisión.
\item Android es un sistema más abierto y "libre" para el desarrollador, permitiéndole tener en una app un control mucho más amplio del terminal que en IOS.
\item La disponibilidad de las herramientas para desarrollar la aplicación también influyó en la decisión. Desarrollar una aplicación IOS requiere hacer uso de un ordenador Mac y de un iPhone para el testeo de la aplicación en el hardware real; herramientas que, teniendo en cuenta que necesitamos hacer uso de los sensores hardware del dispositivo, habría sido necesario adquirir para acometer el desarrollo.
\item Android copa actualmente el 92\% del mercado español \cite{website:cuota}, por lo que por estadística podría hacer uso de la aplicación un público más amplio.
\end{itemize}

Todos estos factores, sumados a la baja curva de aprendizaje necesaria debido principalmente a mi experiencia previa con Java, nos llevaron a seleccionar Android como plataforma de desarrollo.

\section{Entorno de construcción}

Una vez que se ha llegado a la conclusión anterior de implementar la aplicación móvil haciendo uso de Android, se procede a analizar el entorno de construcción que permitirá el desarrollo de todo el sistema.

\subsection{Android Studio}

Para el desarrollo de la aplicación haremos uso del entorno de desarrollo integrado (IDE) Android Studio \cite{website:androidstudio}. Android Studio es el entorno de desarrollo integrado oficial para el desarrollo de aplicaciones para Android y se basa en IntelliJ IDEA.\\

Entre las muchas ventajas que nos ofrece podemos contar con una vista ordenada y modular de los archivos que componen nuestro proyecto. La vista mostrada no se corresponde con la forma original de los archivos en disco, sino que está pensada para optimizar el trabajo.\\

Integrado en nuestro entorno de trabajo, podemos encontrar un sistema para el control de versiones, que incluye soporte para los repositorios más conocidos (Git, Subversion, etc). En nuestro caso se integró con Git, que es el software de control de versiones seleccionado para el presente proyecto, como ya se indicó anteriormente en el documento.\\

La interfaz gráfica está dividida de forma que se muestra una gran cantidad de información de forma eficiente. Además es totalmente personalizable, permitiendo al usuario mostrar y ocultar secciones según se requiera. Android Studio sigue el contexto de trabajo del usuario, mostrando las ventanas con las herramientas que considera oportunas en cada instante.\\

A continuación se muestra un ejemplo de interfaz gráfica:

\figura{androidstudio.png}{scale=0.34}{Interfaz de Android Studio}{androidstudio}{H}

\begin{enumerate}
\item Barra de herramientas: Permite realizar una amplia de gama de acciones: edición, búsqueda, iniciar aplicación, debug, etc. 
\item Barra de navegación: Permite navegar entre los archivos abiertos para facilitar su edición.
\item Ventana de edición: Permite modificar los diferentes archivos.
\item Ventana de herramientas: Proporciona acceso a la gestión del proyecto, control de versiones, consola, etc.
\item Barra de estado: Muestra el estado del proyecto, así como mensajes y avisos.
\end{enumerate}

Para finalizar nuestra visión de Android Studio, finalizaremos hablando de Gradle. Gradle es un sistema abierto que automatiza la compilación del código. Utiliza un grafo dirigido acíclico para determinar en qué orden se pueden ejecutar las tareas. Android Studio incorpora esta tecnología para facilitar la reutilización de código permitiendo generar APKs con distintas funcionalidades partiendo de un mismo proyecto.

\subsection{Android Software Development Kit}

Android SDK o Software Development Kit es un conjunto de herramientas para desarrollar, compilar y depurar aplicaciones para el sistema operativo Android, en definitiva, es una API que además de herramientas para el desarrollo, proporciona soporte técnico, ejemplos y una buena documentación.\\

El SDK incluye un emulador de dispositivos Android, lo que permite probar las aplicaciones de forma rápida y eficiente. Los parámetros de nuestro emulador pueden configurarse de forma que podemos elegir las características que tendrá nuestro dispositivo, como la memoria principal, el tamaño del dispositivo o la versión de Android, lo que permite al programador probar su código en una infinidad de dispositivos sin necesidad de adquirirlos.

\section{Código fuente}

\section{Detección de movimiento}





\chapter{Pruebas y validaciones}
% -*-cap2.tex-*-
% Este fichero es parte de la plantilla LaTeX para
% la realización de Proyectos Final de Carrera, protejido
% bajo los términos de la licencia GFDL.
% Para más información, la licencia completa viene incluida en el
% fichero fdl-1.3.tex

% Copyright (C) 2009 Pablo Recio Quijano 

En este capitulo se presenta el plan de pruebas del sistema, incluyendo los diferentes tipos de pruebas que se han llevado a cabo, ya sean manuales o con ayuda de algún software específico de pruebas para la depuración de la aplicación.

\section{Estrategia}

La estrategia que se siga en el devenir de la ejecución de pruebas sobre el sistema abarcará tanto al alcance de éstas como el objetivo, interpretación y evaluación de los resultados que éstas reflejen.\\

Para el proyecto se han realizado pruebas unitarias, de integración, funcionales y no funcionales. Las dos primeras se basan en las pruebas empleadas de manera automática, mientras que las otras dos se realizan atendiendo a los requerimientos funcionales y no funcionales especificados para el proyecto.

\section{Entorno de pruebas}

En esta sección se procede a especificar las características de los dispositivos empleados para la realización de las pruebas. La aplicación se ha ido depurando haciendo uso del mismo ordenador portátil que se ha empleado para el desarrollo del sistema y sus características son:

\begin{itemize}
\item MacBook Pro (13 pulgadas, mediados 2010)
\item Procesador Intel Core 2 Duo, 2,4 GHz
\item 4 GB RAM
\item HDD 164GB
\item Gráfica NVIDIA GeForce 320M 256 MB
\item Monitor 13 pulgadas
\end{itemize}

En cuanto a las características del dispositivo móvil usado para las pruebas funcionales y no funcionales realizadas sobre el hardware real, son las siguientes:

\begin{itemize}
\item Xiaomi MI5
\item Versión de Android 6.0.1 MXB48T
\item Procesador Quad-core Max 1,80 GHz
\item 3 GB RAM
\item 32 GB de memoria interna
\item Pantalla de 5,15 pulgadas
\end{itemize}

\section{Pruebas unitarias}

Una prueba unitaria es una forma de comprobar el correcto funcionamiento de una unidad o módulo de código, de manera que sirva para asegurar que cada módulo funciona por separado antes de integrarlo con el resto de módulos del sistema.\\

Al dividirse el desarrollo del proyecto en iteraciones como se mencionó en el capítulo 3 Planificación en la sección y se detalló concretamente en la sección 3.4.2. del mismo, se han realizado las pruebas en cada una de las iteraciones antes de continuar con la siguiente.\\

Por tanto, se realizaron pruebas unitarias en cada módulo desarrollado de forma independiente tras cada iteración haciendo uso del depurador de código que lleva integrado Android Studio, y que nos permite ir haciendo debug a la aplicación línea a línea, comprobando que el comportamiento de la lógica de negocio de cada uno de los módulos era el esperado y que los valores de las variables y devueltos por los diferentes métodos y funciones eran consistentes y ajenos a fallos, corrigiéndolos si estos aparecían.

\section{Pruebas de integración}

Una prueba de integración se basa en la comprobación del correcto funcionamiento de un subsistema completo compuesto por varios módulos que, aunque funcionen correctamente por separado, pueden provocar errores a la hora de integrar unos con otros.\\

Al igual que para las pruebas unitarias, las pruebas de integración se han realizado a medida que se avanzaron en las iteraciones del proyecto cuando los módulos de cada iteración se integraban con los anteriormente desarrollados, comprobando que no se produjesen por ejemplo errores de dependencia entre clases.\\

La realización de pruebas de integración se llevó a cabo con las herramientas de debug integradas en Android Studio, tal y como se ha comentado en el punto anterior para el caso de las pruebas unitarias.

\section{Pruebas funcionales}

Las pruebas funcionales se realizan en base a los requerimientos y requisitos establecidos para el desarrollo del proyecto, de modo que será necesario ir comprobando que se cumplen cada una de las funcionalidades especificadas.\\

Se ha establecido por tanto una serie de pruebas concretas para probar que los resultados que proporcione el sistema son los que se esperan de cada funcionalidad, que son las que pasamos a listar a continuación:

\begin{itemize}
\item Alta de persona
\item Listar todas las personas dadas de alta en la aplicación y verificación de los datos mostrados
\item Visualizar el detalle de la persona
\item Eliminar persona
\item Mostrar detalles de sesión realizada
\item Continuar sesión de ejercicios en progreso
\item Iniciar nueva sesión de ejercicios
\item Listar los diferentes tests y sus datos
\item Seleccionar test para su realización
\item Realizar el test Fuerza de piernas
\item Realizar el test Fuerza de brazos
\item Realizar el test Resistencia aeróbica
\item Realizar el test Agilidad
\item Introducir la medida obtenida tras la realización del test Flexibilidad de piernas
\item Introducir la medida obtenida tras la realización del test Flexibilidad de brazos
\item Guardar el resultado obtenido/introducido en los tests
\item Visualizar la pantalla de resultados tras completar todos los tests
\end{itemize}

\section{Pruebas no funcionales}

Las pruebas no funcionales se encargan de verificar que se cumplen los requisitos no funcionales establecidos en el capítulo 4 Análisis de requisitos, en el apartado 4.1.3.\\

En cuanto al ordenador portátil empleado para el desarrollo de la aplicación y al dispositivo móvil usado para las pruebas en el hardware, se ha verificado que han respondido correctamente a las pruebas realizadas a través de ambos.\\

Referente al rendimiento del sistema con respecto a la gestión y persistencia de los datos almacenados, se ha verificado que la información se almacena correctamente a través de la aplicación y se recupera sin incidencias, manteniéndose estable en todo momento el comportamiento de la aplicación.\\

Para finalizar, se realizaron pruebas sobre la aplicación con respecto a la visualización del contenido en dispositivos móviles con pantallas de diferentes tamaños y formatos, corroborando que el diseño responsive empleado en las pantalla de la aplicación Android realizan correctamente su función.


\chapter{Herramientas utilizadas}
% -*-cap2.tex-*-
% Este fichero es parte de la plantilla LaTeX para
% la realización de Proyectos Final de Carrera, protejido
% bajo los términos de la licencia GFDL.
% Para más información, la licencia completa viene incluida en el
% fichero fdl-1.3.tex

% Copyright (C) 2009 Pablo Recio Quijano 

A continuación se listarán las herramientas software que se han usado durante la realización del proyecto, excluyendo aquellas que ya se nombraron y detallaron en los apartados 5.2. Arquitectura lógica del sistema y 6.2. Entorno de construcción.

\section{Cliente del sistema de control de versiones}

Como ya se comentó anteriormente en el presente documento, el software de control de versiones que se ha usado durante la realización del proyecto ha sido Git, que se integra perfectamente con Android Studio.\\

Aun así, también se ha hecho uso del cliente GitHub para la manipulación del código fuente alojado en el repositorio, así como para versionar ficheros a los que no se accedían directamente desde Android Studio, como por ejemplo aquellos correspondientes a la memoria.

\figura{github.png}{scale=0.6}{Logo de GitHub}{github}{H}

\section{Redacción de la memoria}

Para la completa realización de la memoria se ha usado LATEX. Se trata de un sistema de composición de textos, orientado específicamente a la creación de libros, documentos científicos y técnicos que pudiesen contener fórmulas matemáticas.\\

\figura{latex.png}{scale=0.6}{Logo de LATEX}{latex}{H}

LATEX es un sistema de composición de textos que está formado mayoritariamente por órdenes (mácros) construidas a partir de comandos de TEX. LATEX es una herramienta práctica y útil pues, a su facilidad de uso, se une toda la potencia de TEX.

\section{Realización de diagramas}

Para la realización de todos los diagramas necesarios que aparecen a lo largo de toda la memoria se ha usado el creador de diagramas llamado Dia.\\

\figura{dia.png}{scale=0.6}{Logo de Dia}{dia}{H}

Dia es un programa de creación de diagramas en GNU/Linux, MacOS X, Unix y Windows, bajo licencia GPL. Puede ser utilizado para dibujar diferentes tipos de diagramas. Actualmente cuenta con herramientas para dibujar diagramas entidad-relación, diagramas UML, diagramas de flujo, diagramas de red y muchos otros tipos.

\section{Ilustración y retoque}

Para la ilustración y retoque de imágenes se ha usado Adobe Photoshop. Con él se han realizado iconos, botones y edición y retoque de algunos de los recursos usados en la aplicación y en la memoria.

\figura{photoshop.png}{scale=0.1}{Logo de Photoshop}{photoshop}{H}


\chapter{Manual de instalación}
% -*-cap2.tex-*-
% Este fichero es parte de la plantilla LaTeX para
% la realización de Proyectos Final de Carrera, protejido
% bajo los términos de la licencia GFDL.
% Para más información, la licencia completa viene incluida en el
% fichero fdl-1.3.tex

% Copyright (C) 2009 Pablo Recio Quijano 


Dado que la aplicación no se encuentra publicada actualmente en Google Play Store, se ha subido el paquete APK con la versión más reciente de la aplicación al repositorio GitHub que se ha usado durante el desarrollo del proyecto.\\

Para descargarlo, basta con que se acceda a la siguiente URL y se pulse sobre el paquete SeniorFitness.apk:\\

\url{https://github.com/alvarogonzalezalvarez/SeniorFitness/tree/master/apk}\\

Es posible que para la instalación del APK en el dispositivo móvil sea necesario activar la opción que permite la instalación de aplicaciones de origen desconocido, lo cual se puede hacer desde los ajustes del sistema del dispositivo móvil usado.


\chapter{Manual de usuario}
% -*-cap2.tex-*-
% Este fichero es parte de la plantilla LaTeX para
% la realización de Proyectos Final de Carrera, protejido
% bajo los términos de la licencia GFDL.
% Para más información, la licencia completa viene incluida en el
% fichero fdl-1.3.tex

% Copyright (C) 2009 Pablo Recio Quijano 


Este capitulo contiene un manual detallado sobre cómo usar la aplicación desarrollada.

\section{Consideraciones previas}

Esta aplicación requiere que la versión de Android instalada en el dispositivo sea igual o superior a la versión 4.0.3. La versión de Android puede ser consultada desde Ajustes -> Información del teléfono -> Versión de Android.

\section{Primera ejecución}

Al instalar y abrir la aplicación por primera vez y por tanto no haber ninguna persona dada de alta en la aplicación, no será posible hacer uso de ninguna de las funcionalidades de la misma, por lo que el primer paso será dar alta a una persona mayor en la aplicación.

\section{Alta de persona}

Todos los campos de la pantalla deben ser cumplimentados obligatoriamente excepto la foto, que se puede dejar sin informar si así se desea. Una vez introducidos, se debe pulsar sobre el botón GUARDAR para que la persona quede dada de alta en la aplicación y se muestre de nuevo la pantalla inicial, mostrando la persona recién registrada. El texto introducido en el campo DNI debe tener un formato de DNI válido.

\subsection{Agregar foto}

Durante el alta de una persona en el sistema, existe la opción de agregar una foto que quedará asociada a la persona, y que será la que se muestre junto a la información de la misma. Para ello es necesario pulsar sobre el botón AGREGAR FOTO, y seguidamente seleccionar una de las dos opciones que se muestran:

\begin{itemize}
\item Tomar foto: Esta opción abre la cámara del dispositivo y permite tomar una fotografía. Una vez tomada, se abrirá la pantalla de edición de imagen para que se pueda seleccionar la sección de la imagen que interese recortar con una relación de aspecto definida por defecto.
\item Seleccionar foto: Esta opción mostrará el explorador de archivos del dispositivo para que se seleccione la foto que se quiera asociar a la persona que se está dando de alta. Al igual que para la opción anterior, se abrirá la pantalla de edición para que se recorte la sección deseada de la imagen.
\end{itemize}

\section{Empezar o continuar con sesión}

Una vez existe al menos una persona dada de alta en la aplicación, ya será posible realizar una sesión de tests. Para ello, desde la pantalla principal, debe pulsarse sobre el icono de las pesas, situado en la esquina superior derecha de la pantalla. Una vez se haya pulsado sobre dicho icono, aparecerá el listado de personas dadas de alta en la aplicación para que se seleccione aquella que iniciará o continuará la sesión.\\

Al seleccionar la persona, se mostrará el listado de tests asociados a la sesión que la persona tenga en progreso en ese momento. En caso de que no tuviese ninguna sesión en progreso, se creará una nueva y se listarán los tests para la misma (al ser una sesión nueva, todos ellos sin realizar). El siguiente paso será seleccionar el test que se quiera llevar a cabo de la batería de pruebas Senior Fitness Test.

\section{Selección de test}

El siguiente paso consiste en seleccionar el test que se quiera realizar de entre los que no se hayan completado aún.  Al lado de cada test se mostrará un icono de información que explicará en que consiste cada prueba y cómo ha de realizarse. Los tests listados en la aplicación son los siguientes: 

\subsection{Fuerza de piernas}

\subsubsection{En qué consiste}

Número de veces que es capaz de sentarse y levantarse de una silla durante un tiempo definido (30 segundos por defecto) con los brazos en cruz y colocados sobre el pecho.

\subsubsection{Uso de la aplicación}

\begin{enumerate}
\item Definir el tiempo que durará el ejercicio (por defecto 30 segundos). 
\item Colocar el dispositivo móvil sobre el muslo de la persona que realiza la prueba. Debe quedar colocado con la pantalla hacia arriba y de forma que sea legible por la persona cuando ésta se encuentra en el punto de partida del ejercicio, esto es, estando sentada. En la siguiente imagen se ilustra como debe quedar colocado el dispositivo en la pierna:

\figura{dispositivo_movil_piernas.png}{scale=0.6}{Colocación del dispositivo móvil para el test Fuerza de piernas}{dispositivo_movil_piernas_instrucciones}{H}
\item Pulsar sobre el botón START.
\item Realizar el ejercicio hasta que termine la cuenta atrás y el dispositivo móvil emita un doble tono de notificación.
\item Pulsar sobre GUARDAR en caso de que se desee registrar el resultado.
\end{enumerate}

\subsection{Fuerza de brazos}

\subsubsection{En qué consiste}

Número de flexiones de brazo completas, sentado en una silla, que realiza durante un tiempo definido (30 segundos por defecto) sujetando una pesa de 3 libras (2,27 kg) para mujeres y 5 libras (3,63 kg) para hombres.

\subsubsection{Uso de la aplicación}

\begin{enumerate}
\item Definir el tiempo que durará el ejercicio (por defecto 30 segundos). 
\item Colocar el dispositivo móvil sobre el antebrazo de la persona que realiza la prueba. Quedará colocado con la pantalla hacia arriba y de forma que sea legible por la persona cuando ésta se encuentra en el punto de partida del ejercicio, esto es, con el brazo extendido. En la siguiente imagen se ilustra la colocación del dispositivo móvil en el brazo de la persona que realizará el ejercicio:

\figura{dispositivo_movil_brazos.png}{scale=0.6}{Colocación del dispositivo móvil para el test Fuerza de brazos}{dispositivo_movil_brazos_instrucciones}{H}

\item Pulsar sobre el botón START.
\item Realizar el ejercicio hasta que termine la cuenta atrás y el dispositivo móvil emita un doble tono de notificación.
\item Pulsar sobre GUARDAR en caso de que se desee registrar el resultado.
\end{enumerate}

\subsection{Resistencia aeróbica}

\subsubsection{En qué consiste}

Número de veces que levanta la rodilla hasta una altura equivalente al punto medio entre la rótula y la cresta ilíaca durante un tiempo definido (2 minutos por defecto). Se contabiliza una vez por cada ciclo (derecha-izquierda).

\subsubsection{Uso de la aplicación}

\begin{enumerate}
\item Definir el tiempo que durará el ejercicio (por defecto 120 segundos). 
\item Colocar el dispositivo móvil sobre el muslo de la persona que realiza la prueba. Debe quedar colocado con la pantalla hacia arriba y de forma que sea legible por la persona cuando ésta se encuentra en el punto de partida del ejercicio, esto es, estando sentada. En la siguiente imagen se ilustra como debe quedar colocado el dispositivo en la pierna:

\figura{dispositivo_movil_piernas.png}{scale=0.6}{Colocación del dispositivo móvil para el test Fuerza de piernas}{dispositivo_movil_resistencia_instrucciones}{H}
\item Pulsar sobre el botón START.
\item Realizar el ejercicio hasta que termine la cuenta atrás y el dispositivo móvil emita un doble tono de notificación.
\item Pulsar sobre GUARDAR en caso de que se desee registrar el resultado.
\end{enumerate}

\subsection{Flexibilidad de piernas}

\subsubsection{En qué consiste}

Sentado en el borde de una silla, estirar la pierna y las manos intentan alcanzar los dedos del pie que está con una flexión de tobillo de 90 grados. Se mide la distancia entre la punta de los dedos de la mano y la punta del pie (positiva si los dedos de la mano sobrepasan los dedos del pie o negativa si los dedos de la manos no alcanzan a tocar los dedos del pie).

\subsubsection{Uso de la aplicación}

\begin{enumerate}
\item Introducir la aplicación la medida obtenida.
\item Pulsar sobre GUARDAR en caso de que se desee registrar el valor introducido.
\end{enumerate}

\subsection{Flexibilidad de brazos}

\subsubsection{En qué consiste}

Una mano se pasa por encima del mismo hombro y la otra pasa a tocar la parte media de la espalda intentando que ambas manos se toquen. Se mide la distancia entre la punta de los dedos de cada mano (positiva si los dedos de la mano se superponen o negativa si no llegan a tocarse los dedos de la mano).

\subsubsection{Uso de la aplicación}

\begin{enumerate}
\item Introducir la aplicación la medida obtenida.
\item Pulsar sobre GUARDAR en caso de que se desee registrar el valor introducido.
\end{enumerate}

\subsection{Agilidad}

\subsubsection{En qué consiste}

Partiendo de sentado, tiempo que tarda en levantarse caminar hasta un cono situado a 2,44 metros, girar y volver a sentarse.

\subsubsection{Uso de la aplicación}

\begin{enumerate}
\item Pulsar sobre el botón START cuando se vaya a iniciar el ejercicio.
\item Pulsar sobre el botón STOP cuando la persona haya vuelto a sentarse tras completar el ejercicio.
\item Pulsar sobre GUARDAR en caso de que se desee registrar el tiempo obtenido.
\end{enumerate}

\section{Consultar detalle e historial de persona}

Si se desea consultar el perfil de una persona dada de alta en la aplicación, basta con pulsar sobre ella en la pantalla principal. Se abrirá entonces la pantalla que muestra el detalle de la persona, donde se podrán realizar las siguientes acciones:

\begin{itemize}
\item Eliminar persona: Si se desea eliminar de la aplicación a la persona seleccionada, basta con pulsar sobre el icono de la papelera, situado en la esquina superior derecha de la pantalla, y confirmar la acción.
\item Consultar sesiones ya realizadas: Si se desea consultar las estadísticas de una sesión ya realizada, basta con pulsar sobre cualquier sesión completada para que se muestre el detalle de la misma.
\item Continuar sesión en progreso: Si la persona dispone de una sesión en progreso y se pulsa sobre ella, se cargará la pantalla de selección de test y se podrá continuar con la sesión desde el estado en el que se encontraba.
\end{itemize}

\chapter{Conclusiones}
% -*-cap2.tex-*-
% Este fichero es parte de la plantilla LaTeX para
% la realización de Proyectos Final de Carrera, protejido
% bajo los términos de la licencia GFDL.
% Para más información, la licencia completa viene incluida en el
% fichero fdl-1.3.tex

% Copyright (C) 2009 Pablo Recio Quijano 

En este capítulo se detallan los objetivos alcanzados, las enseñanzas obtenidas por parte del desarrollador del proyecto y las posibles mejoras futuras que puedan implementarse en el sistema.

\section{Objetivos alcanzados}



\backmatter % Apéndices, bibliografia ...


\clearpage
\addcontentsline{toc}{chapter}{Bibliografia y referencias}
\bibliographystyle{unsrt}
\bibliography{bibliografia}


\input{fdl-1.3.tex}

\end{document}
