% -*-cap2.tex-*-
% Este fichero es parte de la plantilla LaTeX para
% la realización de Proyectos Final de Carrera, protejido
% bajo los términos de la licencia GFDL.
% Para más información, la licencia completa viene incluida en el
% fichero fdl-1.3.tex

% Copyright (C) 2009 Pablo Recio Quijano 

A continuación se listarán las herramientas software que se han usado durante la realización del proyecto, excluyendo aquellas que ya se nombraron y detallaron en los apartados 5.2. Arquitectura lógica del sistema y 6.2. Entorno de construcción.

\section{Cliente del sistema de control de versiones}

Como ya se comentó anteriormente en el presente documento, el software de control de versiones que se ha usado durante la realización del proyecto ha sido Git, que se integra perfectamente con Android Studio.\\

Aun así, también se ha hecho uso del cliente GitHub para la manipulación del código fuente alojado en el repositorio, así como para versionar ficheros a los que no se accedían directamente desde Android Studio, como por ejemplo aquellos correspondientes a la memoria.

\figura{github.png}{scale=0.6}{Logo de GitHub}{github}{H}

\section{Redacción de la memoria}

Para la completa realización de la memoria se ha usado LATEX. Se trata de un sistema de composición de textos, orientado específicamente a la creación de libros, documentos científicos y técnicos que pudiesen contener fórmulas matemáticas.\\

\figura{latex.png}{scale=0.6}{Logo de LATEX}{latex}{H}

LATEX es un sistema de composición de textos que está formado mayoritariamente por órdenes (mácros) construidas a partir de comandos de TEX. LATEX es una herramienta práctica y útil pues, a su facilidad de uso, se une toda la potencia de TEX.

\section{Realización de diagramas}

Para la realización de todos los diagramas necesarios que aparecen a lo largo de toda la memoria se ha usado el creador de diagramas llamado Dia.\\

\figura{dia.png}{scale=0.6}{Logo de Dia}{dia}{H}

Dia es un programa de creación de diagramas en GNU/Linux, MacOS X, Unix y Windows, bajo licencia GPL. Puede ser utilizado para dibujar diferentes tipos de diagramas. Actualmente cuenta con herramientas para dibujar diagramas entidad-relación, diagramas UML, diagramas de flujo, diagramas de red y muchos otros tipos.

\section{Ilustración y retoque}

Para la ilustración y retoque de imágenes se ha usado Adobe Photoshop. Con él se han realizado iconos, botones y edición y retoque de algunos de los recursos usados en la aplicación y en la memoria.

\figura{photoshop.png}{scale=0.1}{Logo de Photoshop}{photoshop}{H}
