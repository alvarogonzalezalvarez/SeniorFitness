% -*-cap2.tex-*-
% Este fichero es parte de la plantilla LaTeX para
% la realización de Proyectos Final de Carrera, protejido
% bajo los términos de la licencia GFDL.
% Para más información, la licencia completa viene incluida en el
% fichero fdl-1.3.tex

% Copyright (C) 2009 Pablo Recio Quijano 


Este capitulo contiene un manual detallado sobre cómo usar la aplicación desarrollada.

\section{Consideraciones previas}

Esta aplicación requiere que la versión de Android instalada en el dispositivo sea igual o superior a la versión 4.0.3. La versión de Android puede ser consultada desde Ajustes -> Información del teléfono -> Versión de Android.

\section{Primera ejecución}

Al instalar y abrir la aplicación por primera vez y por tanto no haber ninguna persona dada de alta en la aplicación, no será posible hacer uso de ninguna de las funcionalidades de la misma, por lo que el primer paso será dar alta a una persona mayor en la aplicación.

\section{Alta de persona}

Todos los campos de la pantalla deben ser cumplimentados obligatoriamente excepto la foto, que se puede dejar sin informar si así se desea. Una vez introducidos, se debe pulsar sobre el botón GUARDAR para que la persona quede dada de alta en la aplicación y se muestre de nuevo la pantalla inicial, mostrando la persona recién registrada. El texto introducido en el campo DNI debe tener un formato de DNI válido.

\subsection{Agregar foto}

Durante el alta de una persona en el sistema, existe la opción de agregar una foto que quedará asociada a la persona, y que será la que se muestre junto a la información de la misma. Para ello es necesario pulsar sobre el botón AGREGAR FOTO, y seguidamente seleccionar una de las dos opciones que se muestran:

\begin{itemize}
\item Tomar foto: Esta opción abre la cámara del dispositivo y permite tomar una fotografía. Una vez tomada, se abrirá la pantalla de edición de imagen para que se pueda seleccionar la sección de la imagen que interese recortar con una relación de aspecto definida por defecto.
\item Seleccionar foto: Esta opción mostrará el explorador de archivos del dispositivo para que se seleccione la foto que se quiera asociar a la persona que se está dando de alta. Al igual que para la opción anterior, se abrirá la pantalla de edición para que se recorte la sección deseada de la imagen.
\end{itemize}

\section{Empezar o continuar con sesión}

Una vez existe al menos una persona dada de alta en la aplicación, ya será posible realizar una sesión de tests. Para ello, desde la pantalla principal, debe pulsarse sobre el icono de las pesas, situado en la esquina superior derecha de la pantalla. Una vez se haya pulsado sobre dicho icono, aparecerá el listado de personas dadas de alta en la aplicación para que se seleccione aquella que iniciará o continuará la sesión.\\

Al seleccionar la persona, se mostrará el listado de tests asociados a la sesión que la persona tenga en progreso en ese momento. En caso de que no tuviese ninguna sesión en progreso, se creará una nueva y se listarán los tests para la misma (al ser una sesión nueva, todos ellos sin realizar). El siguiente paso será seleccionar el test que se quiera llevar a cabo de la batería de pruebas Senior Fitness Test.

\section{Selección de test}

El siguiente paso consiste en seleccionar el test que se quiera realizar de entre los que no se hayan completado aún.  Al lado de cada test se mostrará un icono de información que explicará en que consiste cada prueba y cómo ha de realizarse. Los tests listados en la aplicación son los siguientes: 

\subsection{Fuerza de piernas}

\subsubsection{En qué consiste}

Número de veces que es capaz de sentarse y levantarse de una silla durante un tiempo definido (30 segundos por defecto) con los brazos en cruz y colocados sobre el pecho.

\subsubsection{Uso de la aplicación}

\begin{enumerate}
\item Definir el tiempo que durará el ejercicio (por defecto 30 segundos). 
\item Colocar el dispositivo móvil sobre el muslo de la persona que realiza la prueba. Debe quedar colocado con la pantalla hacia arriba y de forma que sea legible por la persona cuando ésta se encuentra en el punto de partida del ejercicio, esto es, estando sentada. En la siguiente imagen se ilustra como debe quedar colocado el dispositivo en la pierna:

\figura{dispositivo_movil_piernas.png}{scale=0.6}{Colocación del dispositivo móvil para el test Fuerza de piernas}{dispositivo_movil_piernas_instrucciones}{H}
\item Pulsar sobre el botón START.
\item Realizar el ejercicio hasta que termine la cuenta atrás y el dispositivo móvil emita un doble tono de notificación.
\item Pulsar sobre GUARDAR en caso de que se desee registrar el resultado.
\end{enumerate}

\subsection{Fuerza de brazos}

\subsubsection{En qué consiste}

Número de flexiones de brazo completas, sentado en una silla, que realiza durante un tiempo definido (30 segundos por defecto) sujetando una pesa de 3 libras (2,27 kg) para mujeres y 5 libras (3,63 kg) para hombres.

\subsubsection{Uso de la aplicación}

\begin{enumerate}
\item Definir el tiempo que durará el ejercicio (por defecto 30 segundos). 
\item Colocar el dispositivo móvil sobre el antebrazo de la persona que realiza la prueba. Quedará colocado con la pantalla hacia arriba y de forma que sea legible por la persona cuando ésta se encuentra en el punto de partida del ejercicio, esto es, con el brazo extendido. En la siguiente imagen se ilustra la colocación del dispositivo móvil en el brazo de la persona que realizará el ejercicio:

\figura{dispositivo_movil_brazos.png}{scale=0.6}{Colocación del dispositivo móvil para el test Fuerza de brazos}{dispositivo_movil_brazos_instrucciones}{H}

\item Pulsar sobre el botón START.
\item Realizar el ejercicio hasta que termine la cuenta atrás y el dispositivo móvil emita un doble tono de notificación.
\item Pulsar sobre GUARDAR en caso de que se desee registrar el resultado.
\end{enumerate}

\subsection{Resistencia aeróbica}

\subsubsection{En qué consiste}

Número de veces que levanta la rodilla hasta una altura equivalente al punto medio entre la rótula y la cresta ilíaca durante un tiempo definido (2 minutos por defecto). Se contabiliza una vez por cada ciclo (derecha-izquierda).

\subsubsection{Uso de la aplicación}

\begin{enumerate}
\item Definir el tiempo que durará el ejercicio (por defecto 120 segundos). 
\item Colocar el dispositivo móvil sobre el muslo de la persona que realiza la prueba. Debe quedar colocado con la pantalla hacia arriba y de forma que sea legible por la persona cuando ésta se encuentra en el punto de partida del ejercicio, esto es, estando sentada. En la siguiente imagen se ilustra como debe quedar colocado el dispositivo en la pierna:

\figura{dispositivo_movil_piernas.png}{scale=0.6}{Colocación del dispositivo móvil para el test Fuerza de piernas}{dispositivo_movil_resistencia_instrucciones}{H}
\item Pulsar sobre el botón START.
\item Realizar el ejercicio hasta que termine la cuenta atrás y el dispositivo móvil emita un doble tono de notificación.
\item Pulsar sobre GUARDAR en caso de que se desee registrar el resultado.
\end{enumerate}

\subsection{Flexibilidad de piernas}

\subsubsection{En qué consiste}

Sentado en el borde de una silla, estirar la pierna y las manos intentan alcanzar los dedos del pie que está con una flexión de tobillo de 90 grados. Se mide la distancia entre la punta de los dedos de la mano y la punta del pie (positiva si los dedos de la mano sobrepasan los dedos del pie o negativa si los dedos de la manos no alcanzan a tocar los dedos del pie).

\subsubsection{Uso de la aplicación}

\begin{enumerate}
\item Introducir en la aplicación la medida obtenida.
\item Pulsar sobre GUARDAR en caso de que se desee registrar el valor introducido.
\end{enumerate}

\subsection{Flexibilidad de brazos}

\subsubsection{En qué consiste}

Una mano se pasa por encima del mismo hombro y la otra pasa a tocar la parte media de la espalda intentando que ambas manos se toquen. Se mide la distancia entre la punta de los dedos de cada mano (positiva si los dedos de la mano se superponen o negativa si no llegan a tocarse los dedos de la mano).

\subsubsection{Uso de la aplicación}

\begin{enumerate}
\item Introducir en la aplicación la medida obtenida.
\item Pulsar sobre GUARDAR en caso de que se desee registrar el valor introducido.
\end{enumerate}

\subsection{Agilidad}

\subsubsection{En qué consiste}

Partiendo de sentado, tiempo que tarda en levantarse caminar hasta un cono situado a 2,44 metros, girar y volver a sentarse.

\subsubsection{Uso de la aplicación}

\begin{enumerate}
\item Pulsar sobre el botón START cuando se vaya a iniciar el ejercicio.
\item Pulsar sobre el botón STOP cuando la persona haya vuelto a sentarse tras completar el ejercicio.
\item Pulsar sobre GUARDAR en caso de que se desee registrar el tiempo obtenido.
\end{enumerate}

\section{Consultar detalle e historial de persona}

Si se desea consultar el perfil de una persona dada de alta en la aplicación, basta con pulsar sobre ella en la pantalla principal. Se abrirá entonces la pantalla que muestra el detalle de la persona, donde se podrán realizar las siguientes acciones:

\begin{itemize}
\item Eliminar persona: Si se desea eliminar de la aplicación a la persona seleccionada, basta con pulsar sobre el icono de la papelera, situado en la esquina superior derecha de la pantalla, y confirmar la acción.
\item Consultar sesiones ya realizadas: Si se desea consultar las estadísticas de una sesión ya realizada, basta con pulsar sobre cualquier sesión completada para que se muestre el detalle de la misma.
\item Continuar sesión en progreso: Si la persona dispone de una sesión en progreso y se pulsa sobre ella, se cargará la pantalla de selección de test y se podrá continuar con la sesión desde el estado en el que se encontraba.
\end{itemize}