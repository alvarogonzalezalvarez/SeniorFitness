% -*-cap1.tex-*-
% Este fichero es parte de la plantilla LaTeX para
% la realización de Proyectos Final de Carrera, protejido
% bajo los términos de la licencia GFDL.
% Para más información, la licencia completa viene incluida en el
% fichero fdl-1.3.tex

% Copyright (C) 2009 Pablo Recio Quijano 

A continuación, se describe la motivación del presente proyecto y su alcance, y justificaremos la elección de un proyecto como el presente. También se incluye un glosario de términos y la organización del resto de la presente documentación.

\section{Motivación}

Actualmente, si tomamos como referencia la población española, las personas mayores de 65 años representan el 18,4\% del número total de habitantes (INE, 2016) \cite{website:ine}, y se espera que este porcentaje continue aumentando en las próximas décadas. Por tanto, es esperable encontrarnos cada vez con mayor número de personas mayores y a su vez con más limitaciones físicas, que vendrán dadas principalmente por el irreversible proceso de envejecimiento, y muchas de las cuales pueden paliarse con el ejercicio físico. Sin embargo, para conseguir los máximos beneficios del ejercicio para cada persona es de vital relevancia conocer la condición física del mayor para la correcta prescripción de ejercicio, y es en este punto donde entra en juego la Senior Fitness Test (SFT), una batería de pruebas específicamente diseñada para evaluar la condición física funcional de las personas mayores.\\

Partiendo de esta base y teniendo en cuenta que actualmente es cada vez más frecuente el uso de dispositivos móviles por una amplia gama de usuarios y que el rango de edad de los mismos crece sin detenerse, se determinó que sería interesante desarrollar una aplicación móvil que, con la ayuda de los numerosos sensores de los que disponen los dispositivos móviles de hoy en día, sirviese como herramienta para realizar las diferentes pruebas físicas que componen la SFT, y además diese la posibilidad de registrar y consultar los resultados obtenidos para cada sujeto en las diferentes sesiones realizadas.\\

De esta forma cualquier profesional dentro del sector de Ciencias de la Salud, de la Actividad Física o del Deporte que esté interesado en llevar un seguimiento de la condición física funcional de personas mayores, puede hacer uso de la aplicación para que ésta contabilice de forma autónoma (sin necesidad de tener que contar a ojo ni con la ayuda de un cronómetro) las repeticiones realizadas y los tiempos obtenidos por los sujetos para cada uno de los diferentes ejercicios de la batería de pruebas, además de quedar registrados en el perfil de cada persona para su estudio.\\

Debido a todo lo anteriormente mencionado y a que el aprendizaje de las tecnologías empleadas resulta muy útil de cara al mercado laboral por tratarse de tecnologías punteras en la actualidad, puede justificarse un proyecto como el presente como Proyecto de Fin de Carrera.

\section{Alcance}

El alcance del proyecto consiste en la realización de una aplicación Android que, mediante los sensores que contienen los dispositivos móviles actuales tales como el acelerómetro, el giroscopio o el sensor de gravedad, sea capaz de detectar la correcta realización de los ejercicios en los que consiste la SFT.\\

Para que los sensores del dispositivo móvil sean capaces de realizar las lecturas necesarias para el correcto cálculo del movimiento realizado durante la ejecución del ejercicio, es necesario que el usuario siga las instrucciones dadas por la aplicación para cada uno de los diferentes ejercicios físicos. Por ejemplo, puede ser necesario que el dispositivo móvil se coloque en la extremidad (brazo o pierna) con la que se realizará la actividad.\\

Durante la realización de las pruebas que hagan uso de los sensores comentados, el dispositivo móvil dará feedback sobre la correcta realización de los ejercicios, emitiendo un sonido con cada repetición realizada de forma satisfactoria y en el mismo instante en el que se contabiliza, sin que sea necesario observar la pantalla del dispositivo para conocer si se ha realizado adecuadamente el ejercicio.\\

Una vez finalizado cualquiera de los tests, la aplicación da la posibilidad de almacenar el resultado para la persona mayor que esté realizando la batería de pruebas, dándose por concluida la sesión cuando se haya completado el último test que quede pendiente de realizar de la SFT.\\

El usuario puede consultar en todo momento los resultados obtenidos por cada uno de los sujetos que han ido realizando las pruebas, tanto para la sesión que esté en progreso (con algún test pendiente de finalizar) como para aquellas sesiones que ya se completaron con anterioridad, para las cuales se mostrarán las correspondientes estadísticas.\\

La aplicación es de código libre bajo la licencia GNU GPL así como las técnicas y herramientas empleadas para su desarrollo. Su uso será en dispositivos móviles Android.

\section{Glosario de términos}

Esta sección contiene una lista ordenada alfabéticamente de los principales términos, acrónimos y abreviaturas específicos del dominio del problema:

\begin{itemize}
\item \textbf{Android}: Android es un sistema operativo basado en el núcleo Linux. Fue diseñado principalmente para dispositivos móviles con pantalla táctil, como teléfonos inteligentes, tablets y también para relojes inteligentes, televisores y automóviles.
\item \textbf{API}: La interfaz de programación de aplicaciones es un conjunto de subrutinas, funciones y procedimientos (o métodos, en la programación orientada a objetos) que ofrece cierta biblioteca para ser utilizado por otro software como una capa de abstracción.
\item \textbf{APK}: Un archivo con extensión .apk (Android Application Package, significado en español: Aplicación empaquetada de Android) es un paquete para el sistema operativo Android. Este formato es una variante del formato JAR de Java y se usa para distribuir e instalar componentes empaquetados para la plataforma Android.
\item \textbf{Aplicación}: Es un tipo de programa informático diseñado como herramienta para permitir a un usuario realizar uno o diversos tipos de trabajos.
\item \textbf{Condición física funcional}: Capacidad física para desarrollar actividades normales de la vida diaria de forma segura, con independencia y sin una excesiva fatiga (Rikli y Jones, 2001).
\item \textbf{e-health} / \textbf{eHealth} / \textbf{eSalud}: Es el término con el que se define al conjunto de Tecnologías de la Información y la Comunicación (TICs) que, a modo de herramientas, se emplean en el entorno sanitario en materia de prevención, diagnóstico, tratamiento, seguimiento, así como en la gestión de la salud, ahorrando costes al sistema sanitario y mejorando la eficacia de este.
\item \textbf{Framework}: Es un conjunto estandarizado de conceptos, prácticas y criterios para enfocar un tipo de problemática particular que sirve como referencia, para enfrentar y resolver nuevos problemas de índole similar. Un framework proporciona bibliotecas para acceder a bases de datos, estructuras para plantillas y gestión de sesiones, y con frecuencia facilitan la reutilización de código.
\item \textbf{Git}: Es un software de control de versiones diseñado por Linus Torvalds, pensando en la eficiencia y la confiabilidad del mantenimiento de versiones de aplicaciones cuando éstas tienen un gran número de archivos de código fuente.
\item \textbf{GPL}: La Licencia Pública General de GNU es la licencia de derecho de autor más ampliamente usada en el mundo del software y garantiza a los usuarios finales (personas, organizaciones, compañías) la libertad de usar, estudiar, compartir (copiar) y modificar el software.
\item \textbf{IDE}: Un entorno de desarrollo integrado es una aplicación informática que proporciona servicios integrales para facilitarle al desarrollador o programador el desarrollo de software.
\item \textbf{INE}: El Instituto Nacional de Estadística (INE) es un organismo autónomo de España encargado de la coordinación general de los servicios estadísticos de la Administración General del Estado y la vigilancia, control y supervisión de los procedimientos técnicos de los mismos. Entre los trabajos que realiza, destacan las estadísticas sobre la demografía, economía, y sociedad españolas.
\item \textbf{JAR}: Es un tipo de archivo que permite ejecutar aplicaciones escritas en el lenguaje Java.
\item \textbf{Java}: Java es un lenguaje de programación de propósito general, concurrente, orientado a objetos que fue diseñado específicamente para tener tan pocas dependencias de implementación como fuera posible.
\item \textbf{Prueba} /  \textbf{ Test de valoración de la condición física}: Son las pruebas que se realizan con la finalidad de medir y valorar las diferentes cualidades físicas básicas en sus diferentes facetas. La medición y valoración de estas cualidades nos informa del estado actual del sujeto. Este dato es fundamental para la programación del entrenamiento, ya que nos indica si hay que trabajar de forma específica alguna de ellas, en función de los objetivos a alcanzar, o por el contrario hay que trabajar de forma general.
\item \textbf{Repositorio}: Es un sitio centralizado donde se almacena y mantiene información digital, habitualmente bases de datos o archivos informáticos.
\item \textbf{Senior Fitness Test (SFT)}: Es una batería de pruebas de valoración de la condición física diseñada por Rikli y Jones. Surgió por la necesidad de crear una herramienta que nos permitiese valorar la condición física de los mayores con seguridad así como de forma práctica.
\item \textbf{SQL}: Es un lenguaje específico del dominio que da acceso a un sistema de gestión de bases de datos relacionales que permite especificar diversos tipos de operaciones en ellos.
\item \textbf{SQLite}: SQLite es un motor de bases de datos muy popular en la actualidad por ofrecer características tan interesantes como su pequeño tamaño, no necesitar servidor, precisar poca configuración, ser transaccional y ser de código libre. Android incorpora de serie todas las herramientas necesarias para la creación y gestión de bases de datos SQLite, y entre ellas una completa API para llevar a cabo de manera sencilla todas las tareas necesarias.
\item \textbf{UML}: El Lenguaje Unificado de Modelado es el lenguaje de modelado de sistemas de software más conocido y utilizado en la actualidad. Es un lenguaje gráfico para visualizar, especificar, construir y documentar un Sistema.
\item \textbf{XML}: El Lenguaje de Marcas Extensible es un meta-lenguaje que permite definir lenguajes de marcas desarrollado por el World Wide Web Consortium (W3C) utilizado para almacenar datos en forma legible.
\end{itemize}

\section{Organización del Documento}

La estructura del presente documento es la siguiente:

\begin{itemize}
\item \textbf{Introducción}: Apartado introductorio sobre la motivación para desarrollar este proyecto, su alcance, así como la estructuración de este documento.
\item \textbf{Descripción general}: Descripción más amplia sobre el proyecto, así como todas las características relevantes que tendrá.
\item \textbf{Planificación}: En este apartado se detalla la planificación realizada para llevar a cabo el proyecto, así como las distintas etapas en las que está compuesto el mismo.
\item \textbf{Análisis de requisitos}: En este apartado se detallan los objetivos del sistema, actores del sistema, requisitos funcionales, requisitos no funcionales, reglas de negocio, requisitos de información y se representará el modelo conceptual de datos del proyecto.
\item \textbf{Diseño del sistema}: El diseño del sistema viene dado por una arquitectura general del sistema, patrones de diseño, diseño físico y lógico de datos, el diseño de la interfaz de usuario y los diagramas de clases y de secuencia.
\item \textbf{Implementación del sistema}: En este apartado se detallan los aspectos más importantes sobre la implementación del sistema: entorno de construcción y el código fuente.
\item \textbf{Pruebas y validaciones}: Pruebas realizada a la aplicación, con el fin de comprobar su correcto funcionamiento y cumplimiento de las expectativas.
\item \textbf{Conclusiones}: Conclusiones obtenidas sobre el proyecto desarrollado, visión futura del mismo y experiencia adquirida.
\item \textbf{Herramientas utilizadas}: Herramientas software utilizadas a lo largo del desarrollo del proyecto.
\item \textbf{Manual de instalación}: Apartado donde se indica como realizar la instalación de la aplicación en el sistema.
\item \textbf{Manual de usuario}: Manual de usuario para el correcto uso de la aplicación.
\item \textbf{Bibliografía}: Referencias consultadas durante la realización del proyecto.
\item \textbf{Licencia GNU GFDL}: Texto completo sobre la licencia GNU GFDL en inglés.
\end{itemize}

